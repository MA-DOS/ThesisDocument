\documentclass[a4paper]{article}

\usepackage{url}
\usepackage[all]{nowidow}

\usepackage{xcolor}
\usepackage[colorlinks=true,colorlinks,
linkcolor=purple,
citecolor=purple,
urlcolor=blue,
filecolor=blue]{hyperref}
\usepackage[capitalise]{cleveref}

\usepackage[maxbibnames=99, defernumbers=true, backend=bibtex, citestyle=numeric-comp, url=true, doi=false, isbn=false, giveninits=true, sortcites=true]{biblatex}

\bibliography{references}

\usepackage{authblk}


\title{Energy-aware Co-location of Scientific Workflow Tasks}

\author{Niklas Fomin}


\affil{Distributed and Operating Systems\\Technische Universit\"at Berlin\\Berlin, Germany\\\texttt{niklas.fomin@campus.tu-berlin.de}}


\begin{document}
\maketitle
\begin{abstract}
    Writing a computer science thesis is a considerable challenge for students.
    In this text, we give some tips and structure to write a great thesis.
    We will go over the research process in general, finding a topic, writing an expos\'e, and thesis structure.
    At the end, we include some tips on researching and writing.
\end{abstract}

\section{Introduction}
\label{sec:introduction} 

If you are reading this, you might be about to start a computer science (or \emph{Information Systems Management}, \emph{ICT Innovation}, etc.) thesis at \emph{Technische Universit\"at Berlin}, maybe even at the \emph{Scalable Software Systems} research group.
A thesis, whether it is a bachelor's or master's thesis, is essentially a research task that you complete yourself.
This can be daunting, but you are not alone: Since completing a thesis is necessary to complete a degree at TUB, you can learn from your previous students' experiences.


\cite{10.1145/1341811.1341822}

\printbibliography

\end{document}