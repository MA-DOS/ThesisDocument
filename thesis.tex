\documentclass[a4paper, 11pt]{article}

\usepackage[utf8]{inputenc}
\usepackage[T1]{fontenc}
\usepackage{fix-cm}
\usepackage{textcomp}
\usepackage[a4paper, margin=3cm]{geometry}
\usepackage[titletoc, title]{appendix}
\usepackage{booktabs}   % for \toprule, \midrule, \bottomrule
\usepackage{makecell}   % for \makecell
\usepackage{graphicx}   % for \resizebox
\usepackage{array}      % for >{\raggedright\arraybackslash}p{} column type
\usepackage{float} 
\usepackage{listings}

\usepackage{color}
\usepackage{booktabs}
\usepackage[all]{nowidow}
\usepackage[dvipsnames]{xcolor}
\usepackage[hidelinks]{hyperref}
\usepackage{acronym}
\usepackage{graphicx}
\usepackage{url}
\usepackage{titlesec}
\usepackage{csquotes}
\usepackage{enumitem}
% math and algorithms
\usepackage{amsmath}
\usepackage[linesnumbered,ruled,vlined]{algorithm2e}

% tables
\usepackage{multirow}
\usepackage{booktabs}
\usepackage{array}
\usepackage{tabularx}
\usepackage{amssymb}
\usepackage{verbatimbox}
\usepackage{verbatim}

% general formatting
\usepackage{graphicx}

% xml and code listings
\usepackage{listings}
\usepackage{xcolor}

\lstdefinelanguage{XML}{
  morestring=[b]",
  morecomment=[s]{<!--}{-->},
  stringstyle=\color{teal},
  identifierstyle=\color{black},
  keywordstyle=\color{blue},
  morekeywords={xmlns,version,type,id,name,value}
}

\lstset{
  language=XML,
  basicstyle=\ttfamily\small,
  numbers=left,
  numberstyle=\tiny\color{gray},
  frame=single,
  breaklines=true,
  showstringspaces=false,
  tabsize=2,
  captionpos=b
}

\lstdefinelanguage{YAML}{
    keywords={true, false, null, !!str, !!map, !!seq},
    keywordstyle=\color{blue}\bfseries,
    basicstyle=\ttfamily\footnotesize,
    comment=[l]{\#},
    commentstyle=\color{gray}\itshape,
    stringstyle=\color{red},
    moredelim=[l]{\&},
    moredelim=[l]{*},
    moredelim=**[il][\color{green}]{:},
    morestring=[b]',
    morestring=[b]"
}

\lstdefinelanguage{C++}{
    language=C++,
    morekeywords={
        alignas, alignof, and, and_eq, asm, atomic_cancel, atomic_commit, atomic_noexcept,
        auto, bitand, bitor, break, case, catch, class, compl, concept, consteval, constexpr,
        constinit, const_cast, continue, co_await, co_return, co_yield, decltype, default,
        delete, do, dynamic_cast, else, enum, explicit, export, extern, false, for, friend,
        goto, if, import, inline, module, mutable, namespace, new, noexcept, not, not_eq,
        nullptr, operator, or, or_eq, private, protected, public, reflexpr, register,
        reinterpret_cast, requires, return, sizeof, static, static_assert, static_cast,
        struct, switch, synchronized, template, this, thread_local, throw, true, try, typedef,
        typeid, typename, union, using, virtual, volatile, while, xor, xor_eq
    },
    sensitive=true,
    morecomment=[l]{//},
    morecomment=[s]{/*}{*/},
    morestring=[b]",
    morestring=[b]',
    moredelim=**[is][\color{blue}]{@}{@},
    keywordstyle=\color{blue}\bfseries,
    commentstyle=\color{gray}\itshape,
    stringstyle=\color{red},
    basicstyle=\ttfamily\footnotesize,
    numbers=left,
    numberstyle=\tiny\color{gray},
    stepnumber=1,
    numbersep=10pt,
    showspaces=false,
    showstringspaces=false,
    showtabs=false,
    frame=single,
    tabsize=4,
    captionpos=b,
    breaklines=true,
    breakatwhitespace=true,
    escapeinside={\%*}{*)}
}

\usepackage{transparent}
\usepackage{eso-pic}
\usepackage[section]{placeins}
\usepackage{setspace}
\usepackage{parskip}

\usepackage{subcaption}

\renewcommand\thefigure{%
\thesection.\arabic{figure}}
\renewcommand\thesubfigure{%
\thesection.\arabic{figure}.\arabic{subfigure}}
\renewcommand\thetable{%
\thesection.\arabic{table}}

\usepackage[main=english, ngerman]{babel}

% we use the cleveref package to refer to figures, sections, etc.
% instead of "Figure~\ref{fig:example}", write only "\cref{fig:example}" and the word "Figure" (or table, etc) will be inserted normally
\usepackage[noabbrev,capitalise]{cleveref}

\usepackage[
    maxbibnames=99,
    style=numeric,
    url=false,
    backend=bibtex8,
    sortcites=true,
]{biblatex}
\bibliography{refs}
\DeclareFieldFormat[online]{urldate}{Last accessed: #1}
\DeclareFieldFormat{eprint}{arXiv: \href{https://arxiv.org/abs/#1}{#1}}
\DeclareFieldFormat[report]{title}{``#1''}


\newcommand{\projectTitle}{Energy-aware Co-location of Scientific Workflow Tasks}
\newcommand{\thesisType}{Master's Thesis}
\newcommand{\authors}{Niklas Fomin}
\newcommand{\matrikel}{464308}
\newcommand{\authorEmail}{\href{mailto:niklas.fomin@campus.tu-berlin.de}{niklas.fomin@campus.tu-berlin.de}}
\newcommand{\examinera}{Prof.~Dr.~habil.~Odej Kao}
\newcommand{\examinerb}{Prof.~Dr. Volker Markl}
\newcommand{\supervisor}{Jonathan Bader}

\newcommand{\projectYear}{2025}
\newcommand{\facultyName}{Fakultät Elektrotechnik und Informatik}
\newcommand{\departmentName}{Fachgebiet Distributed and Operating Systems}
\newcommand{\myparagraph}[1]{\paragraph{#1}\mbox{}\\}

\begin{document}
\input{front.tex}
\clearpage
\pagestyle{plain}

\newpage

\section*{KI-Disclaimer}
Hiermit versichere ich, dass ich in der vorliegenden Arbeit ChatGPT 5, Version 1.2025.168 und DeepL Translator for Mac ausschließlich zur Verbesserung der sprachlichen Qualität verwendet habe. Die inhaltlichen Leistungen sind meine eigenen.

Die Satzung zur Sicherung guter wissenschaftlicher Praxis an der TU Berlin vom 8. März 2017\footnote{\url{https://www.static.tu.berlin/fileadmin/www/10000060/FSC/Promotion___Habilitation/Dokumente/Grundsaetze_gute_wissenschaftliche_Praxis_2017.pdf}} habe ich zur Kenntnis genommen.

\vskip 2cm
\begin{tabular}{@{}p{.5in}p{4in}@{}}
     & \hrulefill                              \\
     & (Unterschrift) \authors, Berlin, \today \\
\end{tabular}


\newpage

\section*{Acknowledgements}
I thank all of them for the support....
I would like to first and foremost thank my wife Jenny who not only supported me during the process of writing this thesis but guided me through a time of personal and professional change. I am thankful for a family that supported my decisions and gave me a place to rest and recalibrate. Through my friends I was able to endure this process.
Special thanks to my supervisor Jonathan Bader for his caring guidance, that exceeded the scope of conducting this thesis.



\newpage
\section*{Abstract}

The combined contribution of processing units and memory devices to a data center's total energy consumption typically ranges between 70\% and 80\%. When scientific workflows are run in HPC data centers due to their parallelized execution multiple tasks can be co-located. Co-location can occur either on the same physical node or within the same virtual container which leads to sharing of the underlying compute and memory resources. A high level of resource sharing slows down the processing duration and increases the energy consumption compared to when executing them exclusively since they don't interfer with each other. Therefore, if possible workflow tasks with a high shared resource profile should be allocated to distinct compute resources to avoid contention, thus avoiding slowdown. In this thesis we present \textit{ShaReComp}, a multi-staged approach that characterizes tasks in terms of temporal signatures and applies dissimilarity-based clustering to group tasks with complementary resource usage. We integrate \textit{ShaReComp} into simulated workflow execution where we implement 4 co-location aware scheduling algorithms alongside co-location baselines, that we call \textit{ShaRiff}. Across nine real-world scientific workflows, \textit{ShaReComp} generally achieves higher intra-cluster dissimilarity than random clustering, confirming its ability to group tasks with complementary resource usage. When integrated into \textit{ShaRiff} scheduling simulation, this approach reduced overall energy consumption by up to 18\% and makespan by up to 15\% compared to random co-location, demonstrating the effectiveness of informed, energy-aware task consolidation.

% GERMAN
\clearpage
\begin{otherlanguage}{ngerman}
    \section*{Kurzfassung}
    Der kombinierte Anteil von Prozessoren und Speichergeräten am Gesamtenergieverbrauch eines Rechenzentrums liegt in der Regel zwischen 70\% und 80\%. Wenn Datenanalyse-Workflows in HPC-Rechenzentren gestartet werden, können aufgrund ihres parallelisierten Ablaufs mehrere Analyse-Jobs gemeinsam ausgeführt werden. Die Zusammenlegung der Jobs kann entweder auf demselben physischen Server oder innerhalb desselben virtuellen Containers erfolgen, was zu einer gemeinsamen Nutzung der zugrunde liegenden Rechen- und Speicherressourcen führt. Eine hohe gemeinsame Nutzung von Ressourcen verlangsamt die Verarbeitungsdauer und erhöht den Energieverbrauch im Vergleich zur exklusiven Ausführung, da sich die Jobs gegenseitig beeinträchtigen. Daher sollten Workflow-Jobs mit einer hohen gemeinsamen Ressourcennutzung nach Möglichkeit verschiedenen Rechenressourcen zugewiesen werden, um Konflikte und damit Verlangsamungen zu vermeiden. In dieser Arbeit stellen wir \textit{ShaReComp} vor, einen mehrstufigen Ansatz, der Workflow-Jobs anhand zeitlicher Verhaltens-Signaturen charakterisiert und eine auf Unähnlichkeit basierende Clusterbildung anwendet, um Jobs mit komplementärer Ressourcennutzung zu gruppieren. Wir implementieren die Umsetzung der Job Ko-lokations Methode \textit{ShaReComp} in eine simulierte Workflow-Umgebung, in der wir neben algorithmischen Vergleichsgrundlagen vier ko-lokations-bewusste Scheduling-Algorithmen entwerfen, die wir \textit{ShaRiff} nennen. In neun realen Datenanalyse-Workflows erzielt \textit{ShaReComp} insgesamt eine höhere Intra-Cluster-Unähnlichkeit als zufälliges Clustering, was die rechnerische Fähigkeit bestätigt, Workflow-Jobs mit komplementärer Ressourcennutzung zu gruppieren. Bei der Integration in die \textit{ShaRiff}-Simulation reduziert dieser Ansatz den Gesamtenergieverbrauch um bis zu 18\% und die Gesamtausführungszeit um bis zu 15\% im Vergleich zur zufälligen Ko-lokation, was die Wirksamkeit einer informierten, energie-effizienten Job-Konsolidierung demonstriert.
\end{otherlanguage}

\clearpage

\tableofcontents
\listoftables
\listoffigures
\listofalgorithms
\clearpage

\section{Introduction}
\label{cha:introduction}

% TODO: Put significant graphic in here for problem motivation
\subsection{Problem Motivation \& Description}
\label{subse:problem_motivation_description}
% Energy-aware Workflow Execution, Thamsen, Ch. 1,3,5
% TODO: Add a well descriptive intuition example on the problem motivation just like in the MA's
The carbon footprint of information and communication technologies (ICT) continues to increase, despite the urgent imperative to decarbonize society and remain within planetary boundaries [citation]. A key driver of this trend is the exponential growth in data collection, storage, and processing across scientific disciplines [citation]. Scientific Workflow Management Systems (SWMSs) have become essential tools for managing this complexity, enabling researchers to exploit computing clusters for large-scale data analysis in domains such as remote sensing, astronomy, and bioinformatics [citation]. However, workflows executed through SWMSs are often long-running, resource-intensive, and computationally demanding, which translates into high energy consumption and substantial greenhouse gas emissions [citation]. Techniques to mitigate these impacts include energy-efficient code generation for workflow tasks and energy-aware scheduling strategies [citation]. Nevertheless, practitioners face challenges in assessing which approaches are most applicable, effective, and feasible to implement without excessive effort \cite{thamsen2025energyawareworkflowexecutionoverview}.

Scientific workflows have become a central paradigm for automating computational workloads on parallel and distributed platforms. With the rapid growth in data volumes and processing requirements over the last two decades, workflow applications have grown in complexity, while computing infrastructures have advanced in processing capacity and workload management capabilities. A critical component of this evolution is energy management, where scheduling and resource provisioning strategies are designed to maximize throughput while reducing or constraining energy consumption [citation]. In recent years, energy management in scientific workflows has gained particular importance, not only because of rising energy costs and sustainability concerns but also due to the pivotal role workflows play in enabling major scientific breakthroughs \cite{Coleman_2021}.

% A review on the decarbonization of high-performance computing centers, CA Silva
High-performance computing (HPC) refers to the pursuit of maximizing computational capabilities through advanced technologies, methodologies, and applications, enabling the solution of complex scientific and societal problems [citation]. HPC systems consist of large numbers of interconnected compute and storage nodes, often numbering in the thousands, and rely on job schedulers to allocate resources, manage queues, and monitor execution. While these infrastructures provide the backbone for highly demanding applications, their intensive power draw—arising not only from computation but also from networking, cooling, and auxiliary equipment—makes them major consumers of electricity and significant contributors to climate change [citation]. The demand for HPC continues to rise across both public and private sectors, fueled by emerging computationally intensive domains such as artificial intelligence, Internet of Things, cryptocurrencies, and 5G networks [citation]. The transition from petascale to exascale performance has amplified sustainability concerns, as operational costs approach parity with capital investment and energy efficiency becomes a limiting factor [citation]. Recent exascale systems exemplify this trend: while achieving unprecedented performance, they consume tens of megawatts of power and highlight the urgency of energy-aware design. To increase efficiency, modern HPC architectures increasingly integrate heterogeneous hardware, combining multicore CPUs with specialized accelerators such as GPUs, enabling more operations per second but also driving higher power densities at the node and rack level. This creates additional challenges for energy management and cooling, compounded by scheduling constraints and the demand for near-continuous system availability. Addressing these issues is essential to ensure that future HPC developments meet performance goals without exacerbating their environmental footprint \cite{Silva_2024}.

% FONDA 2
As scientific workflows grow in scale and complexity, coarse models of resource usage are no longer sufficient for ensuring efficient and sustainable execution. Tasks within data-driven workflows often appear in multiple instances and may vary substantially depending on input data, parameters, or execution environments. This results in highly dynamic patterns of resource demand, energy consumption, and carbon emissions, which fluctuate during runtime rather than remaining constant. At the same time, the carbon intensity of the underlying infrastructure changes over time, reflecting variations in energy availability and network conditions across sites. To address these challenges, there is a need for fine-grained, time-dependent task models that capture detailed resource usage profiles, incorporate infrastructure energy characteristics, and adapt dynamically during execution. Such models would enable more accurate task-to-machine mappings, informed scheduling decisions across multiple sites, and strategies for co-locating complementary tasks to reduce interference, energy waste, and communication overhead. Developing these models requires continuous monitoring of tasks, predictive techniques to handle incomplete prior knowledge, and adaptive mechanisms that respond in real time to evolving infrastructure conditions. Ultimately, this approach provides the foundation for carbon-aware workflow execution, where tasks are scheduled and allocated in ways that minimize energy consumption and associated emissions while maintaining performance and scalability.
Task mapping addresses the challenge of assigning tasks to machines in a way that not only satisfies resource requirements but also minimizes energy consumption and carbon emissions. Traditional approaches largely focused on matching resource demands with machine capabilities, but they overlooked the variability in energy efficiency and carbon intensity over time. By leveraging fine-grained task models and infrastructure profiles, mappings can be extended to consider spatio-temporal variations in energy supply and demand. A key component of this is task co-location: strategically placing tasks on the same machine, within the same cluster, or in proximity across sites. When executed with awareness of complementary resource usage patterns, co-location reduces interference, avoids idle resource blocking, and lowers communication overhead, thereby improving both performance and energy efficiency. Together, advanced task mapping and co-location strategies provide a pathway toward reducing the carbon footprint of computational workflows while maintaining scalability and reliability.

% pHPCe: a hybrid power conservation approach for containerized HPC environment
Reducing power consumption while maintaining acceptable levels of performance remains a central challenge in containerized High Performance Computing (HPC). Existing approaches to Dynamic Power Management (DPM) typically rely on profile-guided prediction techniques to balance energy efficiency and computational throughput. However, in multi-tenant containerized HPC environments, this balance becomes significantly more complex due to heterogeneous user demands and contention over shared resources. These factors increase the difficulty of accurately predicting and managing power-performance trade-offs. Furthermore, while containerization frameworks such as Docker are widely adopted, they were not originally designed with HPC workloads in mind, leading to limited research and insufficient software-level mechanisms for monitoring and controlling power consumption in such environments \cite{Kuity_2023}.

Task clustering is a technique aimed at improving workflow efficiency by aggregating fine-grained computational tasks into larger units, commonly referred to as jobs. This aggregation reduces the overhead associated with scheduling numerous small tasks individually, which in turn lowers energy consumption and shortens the overall makespan of the workflow. By combining coarse-grained and fine-grained tasks into clustered jobs, resource utilization becomes more balanced, and the performance of workflow execution can be significantly enhanced \cite{Saadi_2023}.

This paper focuses on effective energy and resource costs management for scientific workflows. Our work is driven by the observation that tasks of a given workflow can have substantially different resource requirements, and even the resource requirements of a particular task can vary over time. Mapping each workflow task to a different server can be energy inefficient, as servers with a very low load also consume more than 50\% of the peak power [8]. Thus, server consolidation, i.e. allowing workflow tasks to be consolidated onto a smaller number of servers, can be a promising approach for reducing resource and energy costs.

We apply consolidation to tasks of a scientific workflow, with the goal of minimizing the total power consumption and resource costs, without a substantial degradation in performance. Particularly, the consolidation is performed on a single workflow with multiple tasks. Effective consolidation, however, poses several challenges. First, we must carefully decide which workloads can be combined together, as the workload resource usage, performance, and power consumption are not additive. Interference of combined workloads, particularly those hosted in virtualized machines, and the resulting power consumption and application performance need to be carefully understood. Second, due to the time-varying resource requirements, resources should be provisioned at runtime among consolidated workloads.

We have developed pSciMapper, a power-aware consolidation framework to perform consolidation to scientific workflow tasks in virtualized environments. We first study how the resource usage impacts the total power consumption, particularly taking virtualization into account. Then, we investigate the correlation between workloads with different resource usage profiles, and how power and performance are impacted by their interference. Our algorithm derives from the insights gained from these experiments. We first summarize the key temporal features of the CPU, memory, disk, and network activity associated with each workflow task. Based on this profile, consolidation is viewed as a hierarchical clustering problem, with a distance metric capturing the interference between the tasks. We also consider the possibility that the servers may be heterogeneous, and an optimization method is used to map each consolidated set (cluster) onto a server. As an enhancement to our static method, we also perform time varying resource provisioning at runtime \cite{5644899}.

\subsection{Research Question \& Core Contributions}
\label{subse:research_question_core_contributions}

The central questions this thesis seeks to address are:

% TODO: refine research questions
\begin{enumerate}[label=RQ\arabic*]
    \item How can fine-grained, time-dependent models of workflow tasks be developed to capture fluctuating patterns of computational resource usage and energy consumption during execution?
    \item How can the co-location of workflow tasks be modeled so that their interference is minimized and the resulting shared usage of resources leads to lower overall energy consumption and carbon emissions while performance is maintained?
    \item How can co-location models and time-dependent task characterizations be integrated into resource management systems and workflow scheduling frameworks to enable adaptive, energy-aware execution at scale?
\end{enumerate}

The resulting core contributions of this thesis are:
\begin{itemize}
    \item An architectural mapping that defines which layers of workflow execution need to be monitored and systematically assigns suitable data exporters to them
    \item The implementation of a monitoring client, capable of serving the relevant monitoring layers for scientific workflow execution which collects fine-grained, time-dependent resource usage data for workflow tasks during execution that was used for data collection on running 10 nf-core pipelines. 
    \item An analysis of co-location effects on the core and node-level that are later on used for implementing the proposed co-location approach.
    \item The design and implementation of a novel co-location approach that utilizes time series data to compute for any given set of tasks clusters where the resource usage patterns are complementary.
    \item The application of 2 multivariate, statistical learning methods on the time series data of the workflow tasks: Kernel Canonical Correlation Analysis (KCCA) and Random Forest Regressor (RFR).
    \item The development of an evaluation test-bed in the WRENCH simulation framework that integrates the proposed co-location approach and allows for the simulation of scientific workflow execution with and without co-location.
    \item The evaluation of the proposed co-location approach using 10 nf-core workflows, demonstrating its effectiveness in reducing energy consumption  while maintaining performance.
\end{itemize}

\subsection{Structure of the Thesis}
\label{subse:structure_of_the_thesis}
The remainder of this thesis is structured as follows: Chapter 2 presents the fundamentals of this work, covering scientific workflow systems, the co-location problem, cluster resource management, machine learning and more. Chapter 3 introduces related work for monitoring scientific workflows, modeling the co-location of tasks and enabling energy-awareness through minimizing resource contention. In addition, the main state-of-the-art approaches for the co-location problem are presented. Furthermore, Chapter 4 depicts the approach to the problem in both a theoretical and practical manner and ultimately defines requirements for the realization of the approach. The corresponding implementation is elaborated in Chapter 5. Moreover, Chapter 6 evaluates the models compared to the state-of-the-art approaches using a simulation. Chapter 7 interprets the key findings and discusses some limitations of this work. Finally, Chapter 8 concludes this thesis and gives an outlook on the impact of this work for the future.
\clearpage
\section{Background}
\label{cha:background}

This chapter provides the necessary background for the thesis. Section 2.1 introduces the domain of High-Performance Computing (HPC), with a focus on modern computational hardware (2.1.1) and virtualization technologies (2.1.2). Section 2.2 then discusses scientific workflows as a representative HPC application, beginning with a definition of scientific workflows (2.2.1), followed by their management systems (2.2.2), and concluding with workflow tasks as their smallest management unit (2.2.3). Section 2.3 introduces the domain of energy-aware computing, while Section 2.4 addresses workflow monitoring, including monitoring targets (2.4.1), resource monitoring (2.4.2), energy monitoring (2.4.3), and task characterization using monitoring data (2.4.4). Section 2.5 presents the co-location problem, which constitutes the core of this thesis. It motivates the problem through resource contention (2.5.1), discusses its relationship to workflow scheduling and task mapping (2.5.2), and provides a general overview with the guiding boundaries that shape the definition of workflow task co-location used throughout this work. Section 2.6 introduces the role of machine learning in scientific workflow processing, focusing on its application in resource management (2.6.1) and the theoretical background of the models applied in this thesis. This includes Kernel Canonical Correlation Analysis (KCCA) (2.6.2.1), Random Forest Regression (2.6.2.2), Linear Regression (2.6.2.3), and Agglomerative Clustering for task clustering (2.6.2.4). Finally, Section 2.7 outlines the evaluation methodology, with an emphasis on simulation approaches and a description of the WRENCH framework for simulating distributed computing environments (2.7.1).

\subsection{High Performance Computing}
\label{sec:background_hpc}
% TODO: Include citation HPC book, Chapter 1.1
High-Performance Computing (HPC) encompasses a collection of interrelated disciplines that together aim to maximize computational capability at the limits of current technology, methodology, and application. At its core, HPC relies on specialized electronic digital machines, commonly referred to as supercomputers, to execute a wide variety of computational problems or workloads at the highest possible speed. The process of running such workloads on supercomputers is often called supercomputing and is synonymous with HPC. The fundamental purpose of HPC is to address questions that cannot be adequately solved through theory, empiricism, or conventional commercial computing systems. The scope of problems tackled by supercomputers extends beyond traditional scientific and engineering applications to include challenges in socioeconomics, the natural sciences, large-scale data management, and machine learning. An HPC application refers both to the problem being solved and to the body of code, or ordered instructions, that define its computational solution.

What distinguishes HPC systems from conventional computers is their organization, interconnectivity, and scale. Here, scale refers to the degree of both physical and logical parallelism: the replication of essential physical components such as processors and memory banks, and the partitioning of tasks into units that can be executed simultaneously. While parallelism exists in consumer devices like laptops with multicore processors, HPC systems exploit it on a vastly larger scale, structured across multiple hierarchical levels. Their supporting software is designed to orchestrate and manage operations at this level of complexity, ensuring efficient execution across thousands of interconnected components.

\paragraph{Modern HPC Hardware}
\label{sec:background_hpc_hardware}
% Include graphic inspired from Chapter 1.1, page 7 of HPC book.
High performance computer architecture determines how very fast computers are formed and function. High performance computing (HPC) architecture is not specifically about the lowest-level technologies and circuit design, but is heavily influenced by them and how they can be most effectively employed in supercomputers. HPC architecture is the organization and functionality of its constituent components and the logical instruction set architecture (ISA) it presents to computer programs that run on supercomputers. HPC architecture exploits its enabling technologies to minimize time to solution, maximize throughput of operation, and serve the class of computations associated with large, usually numeric-intensive, applications. In recent years supercomputers have been applied to data-intensive problems as well, popularly referred to as “big data” or “graph analytics”. For either class of high-end applications, HPC architecture is created to overcome the principal sources of performance degradation, including starvation, latency, overheads, and delays due to contention. It must facilitate reliability and minimize energy consumption within the scope of performance and data requirements. Cost is also a factor, affecting market size and ultimate value to domain scientists and other user communities. Finally, architecture shares in combination with the many other layers of the total HPC system the need to make application programming by end users as easy as possible.

HPC system performance depends on the speed of its components, with processor clock rate being a key factor. A major challenge arises from mismatches in cycle times across technologies, such as fast processor cores versus much slower DRAM. To bridge this gap, modern architectures use memory hierarchies that combine high-capacity DRAM with faster SRAM caches. Performance is also shaped by communication speed, measured in bandwidth and latency, which vary with technology and distance. Ultimately, HPC architecture seeks to balance computation, memory, and communication speeds while optimizing cost, power, and usability to maximize application performance.
Efficiency in HPC refers to how effectively system components are utilized when executing a workload. A common metric is floating-point efficiency, defined as the ratio of sustained floating-point performance to the theoretical peak, both measured in FLOPs. While once meaningful in an era when floating-point operations were costly, this measure has become less representative as data movement and memory access now dominate in terms of time, energy, and die space. Nevertheless, FLOP-based efficiency remains the most widely reported measure.

Power consumption is a critical factor in HPC, as processors, memory, interconnects, and I/O devices all require electricity, and the resulting heat must be removed to avoid failure. Processor sockets alone may consume 80–200 W, and cooling adds significant overhead, sometimes exceeding 20\% of total power use. Air cooling suffices for smaller systems, but high-density, high-performance systems increasingly rely on liquid cooling to achieve higher packing density and performance. Modern processors further support power management through dynamic voltage and frequency scaling, variable core activation, and thermal monitoring. These mechanisms enable a balance between power consumption and performance, guided by software that can set or adjust configurations at runtime based on workload demands.

The multiprocessor class of parallel computer represents the dominant architecture in contemporary supercomputing. Broadly defined, it consists of multiple independent processors, each with its own instruction control, interconnected through a communication network and coordinated to execute a single workload. Three main configurations of multiprocessor systems exist: symmetric multiprocessors (SMPs), massively parallel processors (MPPs), and commodity clusters. The distinction lies in memory organization. SMPs use a shared memory model accessible by all processors, MPPs assign private memory to each processor, and cluster-based designs combine both approaches by grouping processors into nodes that share memory locally while maintaining separation across nodes. Modern multicore systems often follow this hybrid structure, balancing performance, scalability, and complexity.
% TODO: Maybe include graphic that shows overview

A shared-memory multiprocessor consists of multiple processors with direct hardware access to a common main memory. This architecture allows any processor to read or write data produced by another, requiring an interconnection network that ensures cache coherence across all processors. Cache coherence guarantees correctness by keeping local caches consistent, often implemented through protocols such as MESI, where writes are detected and other caches are updated or invalidated accordingly.

Shared-memory multiprocessors are commonly divided into two categories based on memory access times. In symmetric multiprocessors (SMPs), all processors can access any memory block in equal time, known as uniform memory access (UMA). While contention for memory banks can still cause delays, the system provides equal opportunity to all processors. SMPs are widely used in enterprise servers, workstations, and multicore laptops, and often serve as nodes within larger parallel systems.

Nonuniform memory access (NUMA) architectures extend shared-memory designs by allowing all processors to access the full memory space but with different access times depending on locality. NUMA leverages fast local memory channels alongside slower global interconnects, enabling greater scalability than SMPs. However, this places additional responsibility on software developers to optimize data placement in order to achieve high performance. NUMA systems emerged in the late 20th century and remain a key design for scaling shared-memory multiprocessors.

% TODO: Add NUMA graphic
% TODO: Add AMD stuff

\subsubsection{Virtualization in HPC}
\label{sec:background_hpc_virtualization}
% TODO: Cite paper: Survey on Virtualization of HPC, M. Shao IEEE
With the growing demand for High-Performance Computing (HPC), hardware resources have continued to expand in scale and complexity, with increasingly intricate interconnections between system components. A central challenge lies in maximizing both performance and resource utilization. Virtualization technologies offer a means to improve resource utilization in HPC environments, but often at the cost of performance overhead. Multi-user usage scenarios, combined with the stringent performance requirements of HPC workloads, place high demands on virtualization approaches. Furthermore, the highly customized nature of HPC systems complicates the integration of virtualization solutions. This work examines operating system-level and application-level virtualization within HPC, outlines the current limitations, and discusses potential directions for future developments.

Virtualization at the operating system level has been widely explored through technologies such as VMware, Xen, LXC, and KVM, each offering different trade-offs between usability, security, and performance. VMware, one of the earliest and most established solutions, allows multiple operating systems to run simultaneously on a single host, offering strong isolation, security, and ease of configuration. However, the performance overhead introduced by full virtual machines is too high for HPC workloads, which demand near-native computational speed. Xen, developed as an open-source virtual machine monitor, improved performance compared to VMware but required operating system modifications, making it less developer-friendly and leading to reduced long-term adoption. More recent solutions such as LXC and KVM integrate virtualization more closely with the Linux kernel, reducing overhead compared to traditional VMs. While these operating system–level containers are lighter than VMware or Xen, they still introduce performance penalties, with KVM in particular showing unacceptable overhead for CPU-intensive HPC applications. Moreover, KVM requires hardware-level virtualization support, limiting its applicability.

In multi-tenant HPC environments, the dynamic sharing of hardware resources introduces complex power–performance relationships, as concurrently executed tasks exhibit varying levels of power consumption across different resources. Users increasingly expect the flexibility of cloud-like environments, where execution conditions resemble their native setups without significant performance loss. Container-based HPC environments address this demand by isolating groups of processes into separate applications that run densely on available hardware threads. \cite{Kuity_2023}.

Virtualization enables the provisioning of resources to multiple users on the same physical machine with the goal of maximizing utilization. Depending on the abstraction layer, different virtualization technologies provide distinct benefits. Containers, which virtualize at the operating system level, offer lightweight isolation and have become widely adopted for deploying microservices. They effectively bridge the gap between development and production by supporting continuous integration and deployment pipelines.
Containers leverage the host operating system kernel and typically incur far less management and runtime overhead compared to virtual machines. Prior studies have shown that container performance often approaches native execution, particularly for CPU- and memory-intensive workloads, whereas VMs tend to suffer greater degradation for memory, disk, and network-intensive applications. In comparative analyses, Docker containers have demonstrated near-native efficiency for CPU and memory tasks but exhibit performance bottlenecks in certain networking and storage configurations. Research has also highlighted scalability challenges, including increased startup times as the number of containers grows, as well as interference effects when multiple containers share disk- or network-intensive workloads on the same host. These findings underline the fact that performance in containerized environments is strongly dependent on workload characteristics and resource contention patterns. Despite advances such as workload-aware brokering systems aimed at improving energy efficiency and utilization, limited attention has been paid to how workload nature and interference influence consolidation decisions \cite{8397647}.

% TODO: Cite paper: Survey on Virtualization of HPC, M. Shao IEEE
Application-level virtualization operates above the kernel layer, sharing both the kernel and underlying hardware. This approach is most prominently represented by containers, with Docker introducing a transformative model of packaging and deployment. By encapsulating application dependencies into images and leveraging ecosystems such as Docker Hub, Docker enables rapid and portable application deployment. However, in HPC environments, Docker raises serious concerns. Security risks emerge from its daemon-based execution model, which runs with root privileges, and its resource allocation mechanism conflicts with HPC scheduling managers like Slurm, PBS, or SGE. These schedulers typically rely on cgroups to enforce job-level resource limits, but such constraints are ineffective when bypassed by Docker’s daemon. Furthermore, Docker lacks robust support for MPI and multi-node collaboration, making it unsuitable for large-scale HPC workloads.

% TODO: Cite paper: Survey on Virtualization of HPC, M. Shao IEEE
To address these issues, container solutions such as Singularity and Shifter were developed specifically with HPC requirements in mind. Singularity allows users to build and test applications in local environments and then deploy them seamlessly on HPC systems. Unlike Docker, it does not rely on a persistent daemon, thus resolving security and resource management concerns. Singularity also supports multiple container image formats, including compatibility with Docker images, and integrates efficiently with HPC resource managers. Its design ensures that containers incur minimal overhead, as virtualization occurs only at the application level with a shared kernel, making it particularly well-suited for HPC contexts where performance efficiency is critical.

% TODO: Containerization graphic
% TODO: Virtual Machine graphic

\subsection{Scientific Workflows}
\label{sec:background_workflows}

% TODO: Include more stuff from 
% \cite{Scientific worflows: past, present and future}
\subsubsection{Scientific Workflow Management Systems}
\label{sec:background_workflows_swms}
Scientific Workflow Management Systems (SWMSs) enable the composition of complex workflow applications by connecting individual data processing tasks. These tasks, often treated as black boxes, can represent arbitrary programs whose internal logic is abstracted away from the workflow system. The resulting workflows are typically modeled as directed acyclic graphs (DAGs), where channels define the dependencies between tasks: the output of one task serves as the input for one or more downstream tasks. This abstraction allows users to design scalable, reproducible workflows while managing execution complexity across diverse computational environments \cite{thamsen2025energyawareworkflowexecutionoverview}.

Scientific Workflow Management Systems (SWMSs) provide a simplified interface for specifying input and output data, enabling domain scientists to integrate and reuse existing scripts and applications without rewriting code or engaging with complex big data APIs. This abstraction lowers the entry barrier for developing and executing large-scale workflows. In parallel, cloud computing has become an increasingly popular execution environment, complementing or replacing traditional HPC clusters. Clouds offer advantages such as elastic scalability, flexible pay-as-you-go pricing models, and access to a wide range of heterogeneous hardware resources, making them attractive for diverse workflow workloads \cite{Bader_2022}.

% TODO: Add graphic of SWMS architecture in general

\subsubsection{Examples of Scientific Workflows}
\label{sec:background_workflows_examples}
% TODO: Add graphic of pipeline 
Scientific workflows are compositions of sequential and con- current data processing tasks, whose order is determined by data interdependencies [4]. A task is the basic data process- ing component of a scientific workflow, consuming data from input files or previous tasks and producing data for follow- up tasks or output files (see Figure 1). A scientific workflow is usually specified in the form of a directed, acyclic graph (DAG), in which individual tasks are represented as nodes. Scientific workflows exist at different levels of abstraction: abstract, concrete, and physical. An abstract workflow mod- els data flow as a concatenation of conceptual processing steps. Assigning actual methods to abstract tasks results in a concrete workflow. If this mapping is performed auto- matically, it is called workflow planning [5]. To execute a concrete workflow, input data and processing tasks have to be assigned to physical compute resources. In the context of scientific workflows, this assignment is called scheduling and results in a physical and executable workflow [6]. Low-level batch scripts are a typical example of physical workflows \cite{Bux2013}.

% TODO: Add section about HPC and Scientific Workflows with reference to earlier section
\subsubsection{Scientific Workflow Tasks}
\label{sec:background_workflows_examples}
Workflow applications are typically executed per input, or per set of related inputs, enabling coarse-grained data parallelism at the application level. Multiple tasks can run concurrently if independent inputs are processed by the same workflow, and parallelism also arises within a single input when workflow graphs fork, allowing downstream tasks to execute simultaneously. Conversely, many workflows begin with parallel execution of different tasks whose outputs are later joined, synchronizing multiple workflow paths.

Due to their complexity and scale, workflows often consist of large numbers of tasks with multiple parallel paths and are executed on clusters—collections of interconnected compute nodes managed as a single system. Scientific Workflow Management Systems (SWMSs) submit ready-to-run tasks to cluster resource managers such as Slurm or Kubernetes, which allocate resources according to user-defined requirements like CPU cores and memory. Task communication is typically implemented via persistent cluster storage systems (e.g., Ceph, HDFS, or NFS), where intermediate data is written and read between tasks. This approach ensures flexible scheduling, fault tolerance through restartable intermediate states, and simplified execution management.

While SWMSs usually treat tasks as black boxes, opening these tasks for optimization can significantly improve performance. In particular, adapting code to the available hardware, such as optimizing for specific CPUs or accelerators, can reduce runtime. Since certain code segments disproportionately affect overall task execution time, focusing on these hotspots offers an effective strategy for workflow-level performance improvement \cite{thamsen2025energyawareworkflowexecutionoverview}.

% TODO: Add graphic of task 
% TODO: Mention task disclaimer for writing ease


\subsection{Energy-Aware Computing}
\label{sec:background_energyawarecomputing}
% TODO: Include more examples of the relevancy of energy-aware computing
\cite{thamsen2025energyawareworkflowexecutionoverview}
Power consumption in computing systems can broadly be divided into static and dynamic components. Static power is drawn even when components are idle, often reaching up to half of peak power, while dynamic power depends on utilization and increases with workload. This inefficiency has motivated the concept of energy-proportional computing, in which energy use per operation scales directly with utilization. Hardware trends and techniques such as Dynamic Voltage and Frequency Scaling (DVFS) have improved proportionality, while workload consolidation further increases efficiency by concentrating tasks on fewer resources.

Beyond compute, data centers also consume energy for cooling, lighting, and auxiliary operations. This is measured by Power Usage Effectiveness (PUE), which expresses the ratio between total data center energy use and the share consumed by IT equipment. While leading hyperscale centers achieve PUE values close to 1.1, the global average remains higher, and the metric has known limitations, including sensitivity to workload characteristics and the inability to capture whole-system trade-offs.

The climate impact of energy use further depends on carbon intensity, i.e., the emissions associated with each unit of consumed electricity. This varies geographically and temporally with the mix of fossil, nuclear, and renewable generation sources. As such, carbon intensity can fluctuate substantially over hours and regions, meaning that both location and timing of computation influence emissions.

Breaking down component-level energy use, CPUs typically dominate server consumption, accounting for 40–66\% of total power, with significant static and dynamic contributions. Memory is the second largest consumer, drawing a smaller but relatively stable share, while storage devices show limited dynamic variation and modest overall consumption. Networking equipment consumes comparatively little, with low dynamic variation, though attributing its energy use to specific workloads is non-trivial.

A geoscience workflow (FORCE) was executed on a commodity cluster using 21 nodes for 315 minutes to process 304 GB of satellite images. The hardware setup led to an estimated operational energy use of 15.8 kWh, factoring in average data center overhead (PUE 1.61). Given Germany’s 2021 grid carbon intensity of 439 gCO₂e/kWh, this corresponds to about 6.95 kgCO₂e emissions per run, or 20.8 kgCO₂e for the three repetitions used to report median results—roughly equal to driving 53 miles in a gasoline car. Beyond operational emissions, the embodied carbon footprint of the hardware was considered. With an estimated 1200 kgCO₂e per node over its lifetime and 0.00599\% of lifetime usage during the experiment, the workflow’s share amounts to about 1.5 kgCO₂e per run, or 4.5 kgCO₂e for three executions. Thus, both operational and embodied emissions contribute to the total footprint of the workflow evaluation.

\subsection{Monitoring Scientific Workflows}
\label{sec:background_monitoring}

\subsubsection{Performance Monitoring}
\label{sec:background_monitoring_performance}
% TODO: Insert citation for HPC Book
Performance monitoring is a critical stage in application development, extending beyond functional correctness and validation of results. Even after thorough testing with diverse datasets and computational modes, hidden inefficiencies may remain that limit the application’s ability to fully exploit the underlying hardware. This is especially significant in parallel computing, where inefficiencies are amplified across many processor cores, leading not only to longer runtimes but also to higher computational costs, as users are often charged proportionally to aggregate machine time. Monitoring helps ensure that performance is not degraded by preventable factors, for example by checking whether computation times match processor capabilities or whether communication delays align with message sizes and network bandwidth. Instrumentation of code segments can provide such insights, though even basic measurements introduce latency and overhead that may distort results or even alter execution flow. To reduce this, statistical sampling is often employed, recording snapshots of program state at intervals rather than logging every event. Sampling periods can be tuned to balance accuracy against intrusiveness. Hardware-based approaches offer another alternative: modern CPUs provide dedicated performance counters for events such as instruction retirement, cache misses, or branches. These counters operate transparently, incurring virtually no overhead, though they are limited to a predefined set of measurable events. Together, these techniques form the basis of effective performance monitoring, allowing developers to identify and address bottlenecks without unduly interfering with execution.

\subsubsection{Monitoring Layers}
\label{sec:background_monitoring_layers}
While low-level tools can provide detailed insights into system resource usage, linking these measurements to higher-level abstractions such as workflow tasks remains challenging. In large-scale scientific workflows executed on distributed environments like clusters or clouds, tasks may be scheduled on arbitrary nodes and may even share resources with other tasks. As a result, locally observed traces of CPU, memory, or I/O usage cannot be directly attributed to specific tasks. To bridge this gap, resource usage profiles from compute nodes must be correlated with metadata such as workflow logs or job orchestrator information (e.g., container management data). Establishing this information chain enables the classification of observed behavior at the task level, highlighting inefficient or resource-intensive components of the workflow that offer the greatest potential for optimization.

Existing monitoring frameworks focus primarily on processes or systems and generally lack the ability to map resource usage back to workflow tasks, especially in multi-tenant or distributed environments. This limits the ability to isolate the contribution of individual tasks to overall resource consumption. Moreover, metrics natively provided by workflow management systems are often coarse-grained, capturing only summary statistics for task lifetimes. To obtain fine-grained insights into task-level behavior, additional instrumentation and mapping strategies are required to connect low-level monitoring data with workflow abstractions \cite{Witzke2024}.

% TODO: Insert adapted graphic of monitoring layers, inspired by Bader.
The proposed architectural blueprint for workflow monitoring is structured into four layers: the resource manager, the workflow, the machine, and the task layer. These layers represent logical distinctions within a scientific workflow execution environment, each focusing on a different monitoring subject and retrieving metrics from lower layers as needed. Unlike user-centric designs, this approach emphasizes system components rather than interaction flows. Higher layers provide increasingly abstract views, relying only on selected metrics from underlying layers while, in theory, being able to access all. For instance, at the resource manager level, systems such as Slurm, Kubernetes, or HTCondor only require aggregate metrics like task resource consumption or available machine resources, without depending on fine-grained traces such as syscalls. The workflow layer captures metrics tied to the workflow specification, while the machine layer delivers detailed reports of node-level resource usage. At the lowest level, the task layer focuses on fine-grained task execution metrics. Together, this hierarchy balances abstraction and granularity, ensuring that relevant monitoring information is exposed at the appropriate level to support both workflow execution and performance optimization \cite{Bader_2022}.

The monitoring architecture is structured into four interdependent layers that capture different levels of abstraction within a workflow execution environment. At the top, the resource manager layer supervises the cluster, orchestrates task assignments, and provides coarse-grained monitoring. It contributes aggregated information such as active workflows, running and queued tasks, node health status, and distributed file system states. To enable correct task placement, it draws on workflow-level identifiers and DAG structures, machine-level metrics like available cores or memory, and task-level resource usage and execution status. The workflow layer refines this view by capturing execution semantics, such as dependencies between tasks expressed as DAGs, workflow progress, runtime statistics, makespan, and error reports. The machine layer focuses on per-node monitoring, reporting both general metrics (e.g., CPU and memory utilization) and fine-grained hardware characteristics such as architecture, clock rates, disk partitions, and virtualization context. At the lowest level, the task layer provides the most detailed insights, including logs, execution times, time-series resource usage, and kernel-level traces such as system calls or I/O activity. These fine-grained metrics are essential for diagnosing failures and identifying performance bottlenecks. Together, the four layers form a hierarchical structure where higher layers abstract and summarize information from lower ones, enabling a comprehensive and scalable approach to workflow monitoring \cite{Bader_2022}.

Scientific workflow management systems (SWMSs) provide varying degrees of built-in monitoring support, which is often complemented by external tools. Pegasus integrates tightly with HTCondor as its resource manager and submits monitoring data directly into a relational database, enabling real-time querying and visualization with external tools such as ElasticSearch or Grafana. Nextflow, originally developed for bioinformatics workflows, has since expanded into other domains and produces monitoring reports once workflow executions are complete. Airflow, with its strong integration into Python, exports selected metrics to StatsD, from where they can be forwarded to systems such as Prometheus for monitoring. Snakemake, also Python-based, supports report generation after execution and offers live monitoring through its Panoptes service, which streams data to an external server accessible via API. Argo, designed natively for Kubernetes, provides a web-based interface where users can access reports and logs of previous executions alongside live monitoring features. Despite these capabilities, most resource managers do not natively handle workflow-level submissions. Consequently, SWMSs typically manage task dependencies and submit tasks sequentially as their requirements are satisfied, with notable exceptions such as HTCondor’s DAGMan meta-scheduler and Slurm’s built-in dependency mechanisms \cite{Bader_2022}.

\subsubsection{Resource Monitoring}
\label{sec:background_monitoring_resource}
% TODO: Insert citation for ebpf book
Monitoring scientific workflows requires tools that can capture and relate information across different components of a computing system, since no single perspective provides sufficient detail for effective optimization. Low-level system monitors can reveal CPU, memory, or I/O usage but lack awareness of which workflow tasks generate that load, while workflow management systems expose task-level metrics that are often too coarse to identify inefficiencies. Similarly, resource managers like Slurm or Kubernetes aggregate machine usage but obscure fine-grained behavior. To close these gaps, specialized tools must be employed at each system layer—resource manager, workflow, machine, and task—and their outputs correlated. This layered approach ensures that high-level abstractions such as workflows can be linked to detailed system traces, allowing bottlenecks to be identified, failures diagnosed, and task placements optimized with respect to both performance and energy consumption. Without combining tools across layers, monitoring remains fragmented, leaving critical inefficiencies hidden in complex, distributed execution environments. In the following, we briefly discuss tools and approaches for monitoring each of these layers in detail while section % reference to section approach for monitoring
will go into much more detail.

Building on the need for layered monitoring, it is essential to introduce some fundamental terms and techniques that describe how monitoring data is collected and analyzed across system components. Central among these is tracing, which captures fine-grained event-based records such as system calls, I/O operations, or network packets. Tracing provides detailed raw data, often with high volume, that can either be post-processed into summaries or analyzed on the fly using programmatic tracers such as those enabled by the Berkeley Packet Filter (BPF). BPF allows small programs to run directly in the kernel, making it possible to process events in real time and reduce the overhead of storing and analyzing massive trace logs. In contrast, sampling tools collect subsets of data at regular intervals to create coarse-grained performance profiles. While sampling introduces less overhead than tracing, it provides only partial insights and can miss important events. Together with fixed hardware or software counters, tracing and sampling form the backbone of observability—the practice of understanding system behavior through passive observation rather than active benchmarking. Observability thus provides the conceptual umbrella under which monitoring tools operate, ranging from low-level system probes to workflow-aware abstractions. 
The BPF ecosystem provides several user-friendly front ends for tracing, most notably BCC (BPF Compiler Collection) and bpftrace. BCC was the first higher-level framework, offering C-based kernel programming support alongside Python, Lua, and C++ interfaces, and it introduced the libbcc and libbpf libraries that remain central to BPF instrumentation. It also provides a large collection of ready-to-use tools for performance analysis and troubleshooting. bpftrace, in contrast, offers a concise, domain-specific language that makes it well suited for one-liners and short custom scripts, while BCC is better for complex programs and daemons. A lighter-weight option, ply, is also being developed for embedded Linux environments. These frameworks, maintained under the Linux Foundation’s IO Visor project, collectively form what is often referred to as the BPF tracing ecosystem.These distinctions are crucial, as they shape how different tools are applied at the task, machine, workflow, and resource manager layers, and they form the basis for the more concrete monitoring techniques discussed in the following sections.

% TODO: Insert overview graph with monitoring targets on OS-level

Building on this, monitoring CPU usage fundamentally relies on understanding how time is distributed across these execution modes and how the scheduler allocates processor resources among competing tasks. Metrics such as user time, system time, and idle time provide the basis for identifying imbalances, while additional indicators like context switches, interrupt handling, and run queue lengths reveal how efficiently the scheduler manages concurrency. Since background kernel activities and hardware interrupts can consume significant CPU cycles outside of explicit user processes, distinguishing their contribution is essential for accurate analysis. Together, these metrics allow researchers and practitioners to detect inefficiencies such as excessive kernel overhead, overloaded cores, or unfair task distribution—insights that are critical when analyzing the performance of scientific workflows running on shared or large-scale computing infrastructures. A practical strategy for CPU performance analysis begins with verifying that a workload is running and truly CPU-bound, which can be confirmed through utilization metrics and run queue latency. Once established, usage should be quantified across processes, modes, and CPUs to identify hotspots such as excessive system time or uneven load distribution. Additional steps include measuring time spent in interrupts, exploring hardware performance counters like instructions per cycle (IPC), and using specialized BPF tools for deeper insights into stalls, cache behavior, or kernel overhead. This structured approach helps progressively narrow down the root causes of performance issues.
In close relation to CPU monitoring, effective memory performance analysis requires tracking how memory behavior influences compute efficiency. Since CPUs frequently stall waiting for data, metrics such as page faults, cache misses, and swap activity can directly translate into wasted cycles. A systematic strategy begins with checking whether the out-of-memory (OOM) killer has been invoked, as this signals critical memory pressure. From there, swap usage and I/O activity should be examined, since heavy swapping almost always leads to severe slowdowns. System-wide free memory and cache usage provide a high-level view of available resources, while per-process metrics help identify applications with excessive resident set sizes (RSS). Monitoring page fault rates and correlating stack traces can reveal which tasks or files drive memory pressure, while tracing allocation calls such as brk() and mmap() offers a complementary perspective on memory growth. At the hardware level, performance counters measuring cache misses and memory accesses give insights into where CPUs are stalling on memory I/O. Together, these monitoring steps build a detailed picture of how memory usage interacts with CPU performance, helping to uncover bottlenecks and inefficiencies that limit overall throughput.

Following CPU and memory, file system monitoring should focus on workload behavior at the logical I/O layer and its interaction with caches and the underlying devices. Key signals include operation mix and rates (reads, writes, opens/closes, metadata ops such as rename/unlink), latency distributions for I/O (including tails), and the balance of synchronous versus asynchronous writes. Cache effectiveness is central: track page-cache hit ratios over time, read-ahead usefulness (sequential vs random access), volumes of dirty pages and write-back activity, and directory/inode cache hit/miss rates. Attribute I/O to files and processes to spot hot files and short-lived file churn, and examine I/O size distributions to detect pathologically small requests. Relate logical I/O to physical I/O to assess whether caching is working or the workload is spilling to disk; include error rates and filesystem-specific events (e.g., journaling) for completeness. Finally, watch capacity and fragmentation risk (very high fill levels), mount options that affect performance semantics (e.g., atime, sync modes), and any lock contention visible in the file system path. A compact table or time-series dashboard for these metrics is helpful for diagnosis and comparison across workloads [cite].

In storage subsystems, background monitoring should characterize I/O where it matters for delivered performance and capacity planning rather than enumerate specific tools. At minimum, track request latency as a distribution (including tails) and decompose it into time queued in the operating system versus service time at the device; sustained high queue time indicates saturation regardless of nominal utilization. Measure throughput and IOPS per device together with queue depth to relate load to response, and record I/O size histograms and access locality (sequential vs. random) to explain latency modes. Distinguish operation classes—reads vs. writes, synchronous vs. asynchronous, metadata, readahead, flush/discard—since policies and devices handle them differently and they can interfere (e.g., reads stalling behind large write bursts). Attribute I/O to processes/containers and, where applicable, to workflow tasks so that noisy neighbors and hot spots can be isolated. Track error and timeout rates at the device interface to separate performance issues from reliability faults. Contextual signals such as filesystem cache hit ratio (logical vs. physical I/O), write-back pressure, and device fill level complement block-layer metrics and help explain shifts in latency or throughput. Segment all measurements per device and scheduling policy, and analyze them over time to identify persistent contention, bimodal behavior, and regressions that warrant tuning, task remapping, or rescheduling.

Containers are a lightweight virtualization technology that isolate applications while sharing the same host operating system kernel. Their implementation in Linux builds on two core mechanisms: namespaces, which provide isolation by restricting the view of system resources such as processes, file systems, and networks; and control groups (cgroups), which regulate resource usage, including CPU, memory, and I/O. Container runtimes such as Docker or orchestration frameworks like Kubernetes configure and combine these mechanisms to provide isolated execution environments. Two versions of cgroups exist in the kernel. Version 1 is still widely used in production systems, while version 2 addresses several shortcomings, including inconsistent hierarchies and limited composability, and is expected to become the default for containerized workloads in the near future.

From a performance monitoring perspective, containers introduce challenges that extend beyond those encountered in traditional multi-application systems. First, cgroups may impose software limits on CPU, memory, or disk usage that can constrain workloads before physical hardware limits are reached. Detecting such throttling requires monitoring metrics that are not visible through standard process- or system-wide tools. Second, containers can suffer from resource contention in multi-tenant environments, where “noisy neighbors” consume disproportionate shares of the available resources, leading to unpredictable performance degradation for co-located containers. This is particularly critical in Kubernetes deployments, where hundreds of containers may share a host and rely on fair scheduling enforced by the kernel and the container runtime.

A further complication lies in attribution. The Linux kernel itself does not assign a global container identifier. Instead, containers are represented as a combination of namespaces and cgroups, which complicates mapping low-level kernel events back to the higher-level abstraction of a specific container. While some workarounds exist—such as deriving identifiers from PID or network namespaces, or from cgroup paths—this mapping is non-trivial and runtime-specific. Consequently, many traditional monitoring tools remain container-unaware, often reporting host-level metrics even when executed inside containers. This mismatch can obscure important performance characteristics, such as CPU scheduling delays or memory pressure within a specific container, and may hinder the diagnosis of resource bottlenecks.

To address these issues, container monitoring must operate at multiple levels of abstraction. Metrics must capture both coarse-grained resource usage across cgroups and fine-grained kernel events such as scheduling latencies, memory allocation faults, and block I/O delays. At the same time, the collected data needs to be attributed correctly to the corresponding container environment in order to identify interference, diagnose bottlenecks, and evaluate orchestration policies. These requirements have motivated the use of extended monitoring techniques such as Berkeley Packet Filter (BPF) tracing, which can instrument kernel paths at runtime and associate low-level events with container-related constructs such as namespaces and cgroups.

Hypervisors enable hardware virtualization by abstracting physical resources and presenting them as fully isolated virtual machines, each running its own kernel. Two common configurations exist: bare-metal hypervisors, such as Xen, which schedule guest virtual CPUs directly on physical processors, and host-based hypervisors, such as KVM, which rely on a host kernel for scheduling. Both models often involve additional I/O handling layers, for example QEMU, which introduce latency but can be optimized through techniques such as shared memory transports and paravirtualized device drivers. Over time, virtualization efficiency has improved significantly with processor extensions like Intel VT-x and AMD-V, paravirtualization interfaces for hypercalls, and device-level virtualization (e.g., SR-IOV). Modern platforms, such as AWS Nitro, minimize software overhead by offloading core hypervisor functionality to dedicated hardware components.

Monitoring hypervisors poses challenges similar to containers but with a different focus. Since each VM runs its own kernel, guest-level monitoring tools can operate normally, but they cannot always observe the virtualization overheads introduced by the hypervisor. Key concerns include the frequency and latency of hypercalls in paravirtualized environments, the impact of hypervisor callbacks on application scheduling, and the amount of stolen CPU time—periods when a guest’s vCPU is preempted by the hypervisor. From the host perspective, additional events such as VM exits provide insight into how often guests trigger traps to the hypervisor, for example on I/O instructions or privileged operations, and how long these exits last. Attribution is further complicated because exit reasons vary widely across workloads and hypervisor implementations.

Effective analysis therefore requires combining guest-side monitoring of virtualized resources with host-side tracing of hypervisor interactions. Guest instrumentation can measure hypercall behavior, detect high callback overheads, or quantify CPU steal time, while host tracing can expose exit patterns, QEMU-induced I/O delays, and hypervisor scheduling effects. As with containers, BPF-based tracing has become particularly valuable in this domain, since it can capture low-level kernel events with high resolution, linking them to hypervisor operations. With the shift toward hardware-assisted virtualization, fewer hypervisor-specific events remain visible to the guest, making host-level tracing and cross-layer correlation increasingly central to performance monitoring of virtualized environments.

\subsubsection{Monitoring Energy Consumption}
\label{sec:background_monitoring_energy}
Energy monitoring of servers combines hardware-aware measurement with system-level attribution to quantify operational emissions and guide optimizations. At its core is a decomposition of power into static (idle) and dynamic (utilization-dependent) components; modern platforms aim for energy-proportional behavior via DVFS and workload consolidation, yet static draw can still be a large fraction of peak power. Component contributions vary—CPUs frequently dominate (tens to ~150 W per socket depending on model and load), memory adds a steadier baseline (few watts per 8 GB, modest dynamic range), storage ranges from low-swing HDDs to more energy-proportional SSDs, and network switches exhibit very low dynamic variability—so monitoring must separate baseline from incremental use. Measurements should be translated into environmental impact using facility Power Usage Effectiveness (PUE) and grid carbon intensity (CI), acknowledging that CI varies by region and time. Practically, on-node electrical telemetry is obtained from Intel RAPL via MSRs and OS interfaces (powercap, perf events), with user-space polling or eBPF-assisted collection; higher-level tools (e.g., CodeCarbon, PowerAPI, Scaphandre) build on these to attribute energy to processes. For multi-tenant hosts, thread/container-level attribution benefits from correlating RAPL energy with performance counters across context switches (e.g., BPF-based approaches such as DEEP-mon), enabling per-thread, per-container, and per-host aggregation with negligible overhead. A robust methodology therefore (i) establishes the static baseline and a CPU model (e.g., ACPS/frequency-aware) to avoid double counting, (ii) profiles disks and NICs under controlled I/O sizes/rates while flushing caches and subtracting baseline+CPU, and (iii) scales results through PUE and CI for carbon accounting, yielding repeatable, architecture-agnostic estimates suitable for power-aware scheduling, capping, and capacity planning.
% TODO: Cite doctoral thesis on RAPL
% TODO: Include schematic overview from doctoral thesis, Figure 1.1, 2.3 and 2.4
% Also mention that this is not a substantial part of the thesis but it rather builds on that
Improving the energy efficiency of high-performance computing (HPC) and data center systems requires accurate and fine-grained monitoring of power consumption at the level of individual servers and their components. Traditional approaches based on external wattmeters or hardware sensors are costly, intrusive, and often lack the accuracy and granularity required for detailed analysis, particularly in heterogeneous and highly dynamic workloads. This has motivated the adoption of hardware-assisted mechanisms such as Intel’s Running Average Power Limit (RAPL), which since the Sandy Bridge architecture has provided direct, low-overhead measurements of energy consumption across CPU cores, the processor package, DRAM, and sometimes integrated GPUs. RAPL enables real-time power monitoring that can be integrated into scheduling, optimization, and power modeling frameworks, offering a practical alternative to coarse external meters or performance-counter-based estimation models. While RAPL has been widely used due to its accuracy, availability, and negligible overhead, its limitations in granularity and the interpretation of certain domains remain open research challenges. In HPC environments, where maximizing performance per watt is critical, RAPL has become a central tool for attributing energy costs to workloads, profiling applications, and guiding energy-aware scheduling, but ongoing work is needed to refine its role in comprehensive system-level energy modeling and management.

Beyond general breakdowns of server energy consumption, the focus in high-performance computing has shifted towards how to obtain accurate and fine-grained measurements without relying on costly or intrusive external hardware. Traditional approaches such as wattmeters or chassis-level sensors remain valuable for coarse measurements, but they lack the granularity to attribute consumption to individual subsystems or workloads. Performance monitoring counters and operating system statistics have therefore been increasingly combined with modeling techniques to approximate component-level power use, though such models often struggle with accuracy under fluctuating workloads. This has motivated the adoption of hardware-integrated interfaces, most notably Intel’s Running Average Power Limit (RAPL), which provides direct access to energy consumption data for CPU packages, cores, and memory domains. In contrast to external metering, RAPL enables low-overhead, programmatic, and high-resolution measurements, making it a widely used foundation for power modeling and energy-aware scheduling in HPC systems.

The Running Average Power Limit (RAPL) interface, first introduced with Intel’s Sandy Bridge architecture, provides a hardware-based mechanism to both measure and limit energy consumption across different CPU domains. Its primary purpose is twofold: offering fine-grained, high-frequency energy measurements and enabling power capping to manage thermal output. RAPL exposes several power domains depending on the processor generation, such as the processor package (PKG), which accounts for the entire socket including cores and uncore components, PP0 for cores, PP1 for integrated GPUs, DRAM for attached memory, and PSys in newer architectures like Skylake for system-level monitoring of the SoC. Among these, the PKG domain is universally available, while support for others varies across server and desktop models. Each domain provides cumulative energy consumption values via model-specific registers (MSRs), updated at millisecond resolution and expressed in architecture-specific energy units. These values can be accessed directly through the MSR driver in Linux or via higher-level interfaces such as sysfs, perf, or the PAPI library, offering flexibility in integrating RAPL into monitoring and profiling workflows. With its combination of accuracy, low overhead, and broad accessibility, RAPL has become a central mechanism for energy measurement and modeling in high-performance and data center computing.
% Include RAPL schema, but mention drawbacks due to AMD architecture

% End with deep-mon and also mention other tools that will be discussed later in the approach/implementation section
With the increasing adoption of containerization in both cloud and HPC environments, fine-grained energy attribution has become a crucial requirement for efficient workload management. While Intel’s RAPL interface provides accurate measurements of CPU and memory energy consumption at high granularity, its integration into container-level monitoring fills an important gap left by earlier VM- and node-centric approaches. Scientific workflows, which often run as complex pipelines of heterogeneous tasks within containers, particularly benefit from this capability, as it enables precise accounting of energy usage per workflow component. This allows power-aware schedulers and orchestrators to balance performance and energy efficiency, identify hotspots, and optimize resource allocation across distributed systems. Tools such as DEEP-mon build on RAPL by combining kernel-level event tracing with container-level aggregation, offering low-overhead monitoring that can attribute energy costs down to threads and containers. Such capabilities are essential to advance sustainable HPC by enabling detailed profiling of workflow execution and supporting energy-aware scheduling decisions in containerized infrastructures.


\subsection{The Co-location Problem}
\label{sec:background_colocation}
% Begin with the origin of co-location being in operting systems scheduling
% Mention that there are many levels of co-location as in FONDA, VMs, containers and in resource managers
% TODO: Cite Barry Linnerts slides from OS or find the direct source
The problem of scientific workflow task co-location can be traced back to classical operating system scheduling, where the core challenge lies in allocating activities to functional units in both time and space. Traditionally, scheduling in operating systems refers to assigning processes or threads to processors, but this problem appears at multiple levels of granularity, from whole programs to fine-grained instruction streams executed within superscalar architectures. On multiprocessor and multicore systems, the scheduler must not only decide when to run a task but also where, since shared caches, memory bandwidth, and execution units create interdependencies between colocated workloads. Early work in operating systems introduced coscheduling or gang scheduling, where groups of related tasks are executed simultaneously to reduce synchronization delays, exploit data locality, and minimize contention for shared resources. These concepts are directly relevant to modern scientific workflows, where multiple interdependent tasks are often colocated on the same nodes or cores in high-performance computing environments. The performance and energy efficiency of workflows are therefore closely tied to scheduling decisions, as co-located tasks may either benefit from shared resource usage or suffer from interference, making scheduling a fundamental problem for efficient workflow execution.
In high-performance computing (HPC), task co-location poses unique challenges due to the complex interplay of shared resources in modern multi-core architectures. Processors share on-chip structures such as last-level caches, memory controllers, and interconnects, as well as off-chip memory bandwidth, creating significant opportunities for contention when multiple applications run concurrently. This contention can result in severe performance degradation, making it critical to understand and predict the impact of co-location on application efficiency. A naïve approach of exhaustively profiling all potential co-locations is infeasible in practice, given the enormous number of possible workload combinations. Instead, research has focused on predictive methodologies that use application-level indicators, such as cache usage or memory access patterns, to estimate interference effects. This has led to classification schemes that characterize applications both by their sensitivity to co-located workloads and by their capacity to disrupt others. 
Building on the challenges of co-location in multi-core systems, the problem becomes even more pronounced when considering scientific workflows, which consist of heterogeneous tasks with highly variable and dynamic resource requirements. Assigning each task to a separate server leads to poor energy efficiency, as servers continue to draw a substantial fraction of their peak power even under low utilization. Server consolidation thus emerges as a promising strategy, where multiple workflow tasks are mapped onto fewer servers to reduce total power consumption and resource costs. In this context, the terms consolidation and co-location can be used interchangeably, as both refer to the placement of multiple tasks onto the same physical resources with the aim of improving efficiency. However, consolidation is far from trivial: the resource usage of colocated tasks is not additive, and interference effects can significantly impact both power consumption and application performance. Furthermore, the temporal variation in workflow task demands requires runtime-aware provisioning strategies to avoid resource contention and performance degradation. These complexities underline the need for methodologies that can accurately predict and manage co-location trade-offs, paving the way for addressing the specific challenges of scientific workflow task consolidation in HPC and cloud environments \cite{5644899}

\subsubsection{Resource Contention and Interference}
\label{sec:background_colocation_interference}
% Distiniguish a little bit between node-level contention and core-contention, but put the methods to the related work section and wove into the approach part as the reasoning for the design decisions.
% TODO: Make a sketch based on Terrible Twins Paper Figure 1
The fundamental motivation for addressing co-location in HPC and cloud environments lies in the problem of resource contention. When processes execute on different cores of the same server, they inevitably share hardware resources such as caches, buses, memory, and storage. This sharing often leads to interference, slowing down execution compared to scenarios where tasks have exclusive access to these resources. In extreme cases, memory traffic contention has been shown to cause super-linear slowdowns, where execution times more than double, making sequential execution more efficient than poorly chosen co-schedules. For large-scale systems hosting thousands of tasks, this contention complicates job scheduling, as schedulers must avoid placing workloads that compete heavily for the same resources. Without accurate estimates of resource usage or slowdown potential, scheduling decisions risk becoming guesswork, reducing overall efficiency. Importantly, poor scheduling is not limited to high-resource applications: even pairing computationally bound tasks that do not interfere can be suboptimal if it prevents beneficial co-scheduling with memory-bound applications. This highlights that effective co-location must avoid both high-contention pairings and missed opportunities for complementary workload placement. Addressing these challenges requires systematic strategies that recognize and mitigate interference, providing the key motivation for studying co-location in the context of scientific workflow tasks \cite{inproceedings}.

In HPC environments, where users submit batch jobs to multi-core compute nodes, efficient resource utilization is critical for balancing throughput, makespan, and job duration. However, parallel applications often fail to fully exploit all allocated cores due to bottlenecks in shared resources such as memory or I/O bandwidth, leading to inefficiencies and longer runtimes. Co-allocating multiple applications on the same nodes has been explored as a strategy to reduce makespan and improve overall system throughput, yet it remains uncommon in production systems because contention for shared resources like last-level caches or memory controllers can increase individual job durations. One approach to mitigate these drawbacks is process mapping, where processes of parallel applications are carefully assigned to specific cores to minimize communication costs and interference. With the growing complexity of HPC architectures, featuring deep memory hierarchies and high core densities, process mapping has become increasingly important. Still, existing solutions often focus solely on single-application performance and rely on costly profiling runs, limiting their applicability in co-located, real-world production workloads. This creates a clear need for methodologies that can address co-location and mapping challenges jointly, enabling more efficient use of HPC resources without sacrificing fairness or performance \cite{10.1007/978-3-031-48803-0_31}.

\subsubsection{Scientific Workflow Scheduling \& Task Mapping}
\label{sec:background_colocation_scheduling}
In this section, we describe two widely-used energy-aware workflow scheduling algorithms that leverage the traditional power consumption model for making scheduling decisions described in the previous section. We then evaluate the energy consumption for schedules computed by these two algorithms using both the traditional and the realistic models. We do so by using a simulator that can simulate the power consumption of a workflow execution on a compute platform for either model. We perform these simulations based on real-world execution traces of three I/O-intensive workflow applications. The specific scheduling problem that these algorithms aim to solve is as follows.
Scheduling Problem Statement. Consider a workflow that consist of single-threaded tasks. This workflow must be executed on a cloud platform that comprises homogeneous, multi-core compute nodes. Initially, all compute nodes are powered off. A compute node can be powered on at any time. Virtual machine (VM) instances can be created at any time on a node that is powered on. Each VM instance is started for an integral number of hours. After this time expires, the VM is shutdown. A node is automatically powered off if it is not running any VM instance. The cores on a node are never oversubscribed (i.e., a node runs at most as many VM instances as it has cores). A VM runs a single workflow task at a time, which runs uninterrupted from its start until its completion. The metrics to minimize are the workflow execution time, or makespan, and the total energy consumption of the workflow execution \cite{HosseiniShirvani2024}.

n this section, we select several representative scheduling algorithms from each category per their performance and impact. Based on the scheduling approaches, we introduce four types of scheduling algorithms:
Task-based (scheduling task-by-task): In the literature, they are also called list-based scheduling. In these algorithms, tasks are ordered based on some priority ranking and then a scheduling decision is made for each task in that order. Path-based (scheduling path-by-path): In these algorithms, a workflow is partitioned into paths based on some criteria and then a scheduling decision is made for each path.
BoT-based (scheduling BoT-by-BoT): In these algorithms, a workflow is partitioned into BoTs (Bag of Tasks) such that each BoT is a set of tasks that have no data dependencies among them, and a scheduling decision is made for each BoT.
Workflow-based (scheduling workflow-by-workflow): In these algorithms, all tasks in a workflow are simultaneously scheduled and then the schedule for each task is improved using some procedure to improve the quality of the global workflow schedule \cite{a survery}.

Node-level and core-level co-location represent two distinct layers of contention within HPC systems. At the node level, tasks share off-chip resources such as memory bandwidth, network interfaces, and I/O subsystems, where heavy communication or I/O-intensive jobs can interfere with one another and degrade performance. At the core level, co-located tasks contend for on-chip resources like private and shared caches, pipelines, and execution units, which can lead to latency, cache thrashing, or reduced instruction throughput. These co-location challenges differ fundamentally from scheduling and task mapping: scheduling determines when tasks execute and mapping decides on which resource they run, while co-location focuses on how tasks interact when placed together on the same hardware. Even though all three may share the objective of improving throughput and minimizing energy consumption, co-location directly addresses the resource interference between tasks and therefore requires strategies that go beyond scheduling order or resource assignment. The challenge lies in predicting and managing these interference effects so that consolidation for energy savings does not undermine overall performance, which makes co-location a critical but separate consideration within workflow execution strategies.
The co-location problem in scientific workflow execution arises directly from the heterogeneous ways tasks can currently be placed on HPC and cloud infrastructures. Tasks may be co-located on the same physical node, executed exclusively on a node, distributed across multiple nodes inside containers, or deployed in virtual machines where either one task runs per VM or multiple VMs are consolidated onto the same host. Each of these deployment choices introduces different levels of contention: cores may compete for shared last-level caches, memory bandwidth, or interconnect capacity, while nodes may compete for I/O channels, network interfaces, or access to distributed storage. Because workflows are executed in these diverse environments, the challenge of co-location becomes embedded into scheduling and task mapping decisions—not as a separate process, but as a cross-cutting concern that determines how efficiently tasks can share resources without incurring performance degradation or excessive energy costs.
% TODO: Make a cool schema about co-location, task mapping and scheduling that showcases a couple of scenarios, motivates them and then shows the focus of this work
% Also bring up the correlation between contention and energy efficiency and make reference to the approach section.
% Sketch Based on terrible twins and Performance impact of resource contention in multicore systems
In the context of scientific workflows, the relation between resource contention and energy efficiency becomes particularly critical, as under-utilized or poorly consolidated resources can lead to significant power waste. Since servers consume a large fraction of their peak power even at low utilization, mapping each workflow task to a separate node often results in inefficiency. Consolidating tasks onto fewer servers can mitigate this by increasing utilization and reducing overall energy consumption. However, consolidation also raises the challenge of interference, as colocated tasks may contend over CPU, memory, disk, or network resources, potentially degrading performance and offsetting the energy savings. This trade-off highlights why the task co-location problem cannot be ignored: energy-efficient workflow execution requires balancing resource consolidation to reduce idle power draw with careful awareness of contention patterns to avoid excessive slowdowns. In this sense, energy efficiency and performance are inherently tied to how tasks are colocated, making it essential to embed power-awareness into workflow scheduling and mapping strategies \cite{5644899} \cite{Lee2012}.

\subsection{Machine Learning in Scientific Workflow Computing}
\label{sec:background_ml}


\subsubsection{Intelligent Resource Management}
\label{sec:background_ml_resourcemanagement}
% TODO: Intro sentence on resource managers in scientific workflows
While traditional HPC systems often avoid colocating applications on the same node due to unpredictable interference effects, the potential benefits of improved throughput and energy efficiency make the problem worth revisiting. When colocated tasks are bottlenecked by different resources, resource utilization can be increased without altering application code, opening opportunities for more efficient execution. Machine learning offers a promising avenue to address the complexity of this problem, as it can capture non-trivial relationships between hardware performance monitoring counters and the resulting performance degradation under colocation. Unlike rule-based heuristics, machine learning models can generalize across diverse applications and workloads, enabling predictive insights into slowdown and contention effects. Although training and inference may be computationally demanding, practical optimizations have shown that machine learning can be applied with manageable overhead, making it a viable tool to guide scheduling and task placement decisions for scientific workflows in HPC environments. Resource colocation in multi-core HPC systems inherently introduces contention for shared on- and off-chip resources such as caches, memory controllers, and interconnects, which can significantly affect application performance. Traditional heuristic-based scheduling approaches, for example those relying only on LLC miss rates, have shown limited success as they often fail to capture the complex and non-linear slowdown effects caused by colocated applications. To overcome these limitations, recent research has turned to machine learning techniques that exploit performance monitoring counters (PMCs) to predict degradation effects more accurately. By training models offline on representative workloads and deploying them in scheduling decisions at runtime, such approaches enable degradation-aware colocation strategies that minimize makespan and improve resource utilization. This integration of predictive modeling into workload management represents a significant step towards intelligent, energy-efficient execution of HPC workloads, where scheduling decisions are not only resource-aware but also contention-sensitive, directly addressing the challenges posed by colocating diverse scientific applications \cite{ZACARIAS2021125}.

When executing scientific workflows on large-scale computing infrastructures, researchers are required to define task-level resource limits, such as execution time or memory usage, to ensure that tasks complete successfully under the control of cluster resource managers. However, these estimates are often inaccurate, as resource demands can vary significantly across workflow tasks and between different input datasets, leading either to task failures when limits are underestimated or to inefficient overprovisioning when limits are set too conservatively. Overprovisioning, while preventing failures, reduces cluster parallelism and throughput, as excess resources are reserved but left unused, while incorrect runtime estimates can distort scheduling decisions and degrade overall system efficiency. To address this, recent research has explored workflow task performance prediction as a means to automate the estimation of runtime, memory, and other resource needs. Machine learning plays a central role in these efforts, with approaches ranging from offline models trained on historical execution data, to online models that adapt dynamically during workflow execution, to profiling-based methods applied before execution. Performance prediction models can integrate into both workflow management systems and resource managers, enabling more informed scheduling, efficient resource utilization, and improved energy- and cost-aware computing. This establishes task-level performance prediction as a crucial foundation for advancing scientific workflow execution towards higher reliability, efficiency, and sustainability \cite{bader2025predictingperformancescientificworkflows}.

\subsubsection{Utilized Machine Learning Algorithms in this Thesis}
\label{sec:background_ml_algorithms}
% Summary sentence about what models are used and what for and where else in the thesis it will be more elaborated on them.

\paragraph{Linear Regression}
\label{sec:background_ml_lr}
% TODO: Add formal example notation and add citation for sklearn documentation.
% Sketch and notation from our MPDS Paper.
Linear regression is a fundamental statistical method used to model the relationship between one or more explanatory variables and a continuous response variable. It estimates coefficients by minimizing the residual sum of squares, providing an optimal linear fit between predictors and outcomes. The model assumes linearity, independence of errors, homoscedasticity, and normally distributed residuals. Ordinary Least Squares (OLS) is the most common approach, where the coefficients are computed analytically from the design matrix. However, when predictors are highly correlated, multicollinearity can occur, making the coefficient estimates unstable and sensitive to noise in the data. Despite this limitation, linear regression remains widely used due to its interpretability, computational efficiency, and ability to serve as a baseline for more complex regression techniques.

% Notation


\paragraph{Kernel Canonical Correlation Analysis}
\label{sec:background_ml_kcca}
% TODO: Add formal example notation and maybe some graphs from cca zoo that show something.
% Add example from pSciMapper and adapt slightly. Can be reused for the approach part.
Canonical Correlation Analysis (CCA) is a statistical technique designed to identify and maximize correlations between two sets of variables. Unlike Principal Component Analysis, which focuses on variance within a single dataset, CCA aims to find linear projections of two datasets such that their correlation is maximized. The result is a set of canonical variables that capture the strongest relationships between the two domains. Kernel Canonical Correlation Analysis (KCCA) extends this idea by applying kernel methods, allowing the detection of nonlinear relationships. In KCCA, the data are implicitly mapped into high-dimensional feature spaces through kernel functions, and correlations are then maximized in that transformed space. This enables KCCA to capture more complex dependencies than linear CCA, making it particularly powerful in settings where relationships between datasets are not strictly linear \cite{1202783}.
Kernel Canonical Correlation Analysis (KCCA) works by taking kernel matrices of two datasets and solving a generalized eigenvalue problem to identify projections that are maximally correlated. Intuitively, this means that KCCA maps both datasets into high-dimensional feature spaces defined by kernel functions and then finds directions in these spaces where the correlation between the two datasets is strongest. In practice, one dataset can represent resource usage metrics while the other represents performance or power measurements. KCCA then identifies correlated structures—such as clusters of similar usage patterns and their corresponding performance or energy behaviors—revealing how variations in resource usage align with variations in system-level outcomes. This makes KCCA particularly suitable for analyzing complex, nonlinear relationships in workflow execution, where resource usage and energy efficiency are intertwined \cite{5644899}.
Ensemble methods improve predictive performance by combining the outputs of multiple base estimators, thereby reducing variance and increasing robustness compared to using a single model. Among the most widely used ensemble approaches are gradient-boosted trees and random forests, both of which rely on decision trees as their fundamental building blocks. Random forests in particular construct a large number of decision trees, each trained on a bootstrap sample of the data and using random subsets of features at split points. This injection of randomness ensures that the trees are diverse, and their errors are less correlated. Predictions are then aggregated, typically by averaging in regression or majority voting in classification, which cancels out some of the individual errors and leads to more stable and accurate results. Intuitively, while a single decision tree may overfit or be highly sensitive to small changes in the data, combining many such trees smooths out these instabilities. The difference between classification and regression in random forests lies in the aggregation step: for classification, the predicted class is determined by majority vote (or probability averaging across trees), while in regression the final prediction is the mean of all tree outputs, yielding a continuous value. This simple but powerful approach has made random forests one of the most effective and robust methods for both supervised learning tasks.

\paragraph{Random Forest Regression}
\label{sec:background_ml_rfr}
% Add citation from sklearn documentation
A Random Forest Regressor is an ensemble learning method that builds multiple decision tree regressors on random subsets of the training data and combines their predictions through averaging. This approach reduces variance compared to a single decision tree, improving predictive accuracy and robustness against overfitting. Each tree is constructed using the best possible splits of the features, while the randomness introduced through bootstrap sampling and feature selection ensures diversity among the trees. An additional advantage is the native handling of missing values: during training, the algorithm learns how to direct samples with missing entries at each split, and during prediction, such samples are consistently routed based on the learned strategy. This makes Random Forest a flexible and powerful method for regression tasks with heterogeneous and potentially incomplete data.

Ensemble methods improve predictive performance by combining the outputs of multiple base estimators, thereby reducing variance and increasing robustness compared to using a single model. Among the most widely used ensemble approaches are gradient-boosted trees and random forests, both of which rely on decision trees as their fundamental building blocks. Random forests in particular construct a large number of decision trees, each trained on a bootstrap sample of the data and using random subsets of features at split points. This injection of randomness ensures that the trees are diverse, and their errors are less correlated. Predictions are then aggregated, typically by averaging in regression or majority voting in classification, which cancels out some of the individual errors and leads to more stable and accurate results. Intuitively, while a single decision tree may overfit or be highly sensitive to small changes in the data, combining many such trees smooths out these instabilities. The difference between classification and regression in random forests lies in the aggregation step: for classification, the predicted class is determined by majority vote (or probability averaging across trees), while in regression the final prediction is the mean of all tree outputs, yielding a continuous value. This simple but powerful approach has made random forests one of the most effective and robust methods for both supervised learning tasks \cite{Breiman2001RandomF}.

Ensemble methods improve predictive performance by combining the outputs of multiple base estimators, thereby reducing variance and increasing robustness compared to using a single model. Among the most widely used ensemble approaches are gradient-boosted trees and random forests, both of which rely on decision trees as their fundamental building blocks. Random forests in particular construct a large number of decision trees, each trained on a bootstrap sample of the data and using random subsets of features at split points. This injection of randomness ensures that the trees are diverse, and their errors are less correlated. Predictions are then aggregated, typically by averaging in regression or majority voting in classification, which cancels out some of the individual errors and leads to more stable and accurate results. Intuitively, while a single decision tree may overfit or be highly sensitive to small changes in the data, combining many such trees smooths out these instabilities. The difference between classification and regression in random forests lies in the aggregation step: for classification, the predicted class is determined by majority vote (or probability averaging across trees), while in regression the final prediction is the mean of all tree outputs, yielding a continuous value. This simple but powerful approach has made random forests one of the most effective and robust methods for both supervised learning tasks.

\paragraph{Agglomerative Clustering}
\label{sec:background_ml_ac}

Hierarchical clustering is a family of algorithms that group data into nested clusters, represented as a tree-like structure called a dendrogram. In this hierarchy, each data point starts as its own cluster, and clusters are successively merged until all points form a single cluster. The commonly used agglomerative approach follows this bottom-up process, with the merging strategy determined by a chosen linkage criterion. Ward linkage minimizes the total variance within clusters, producing compact and homogeneous groups, while complete linkage minimizes the maximum distance between points in different clusters, emphasizing tight cluster boundaries. Average linkage instead considers the mean distance across all points between clusters, and single linkage focuses on the minimum distance, often resulting in elongated or chain-like clusters. Although flexible, agglomerative clustering can be computationally expensive without additional constraints, as it evaluates all possible merges at each step.

\subsubsection{Simulating Distributed Computing}
\label{sec:background_simulation}
% TODO: See if a wrench graphic would make sense here.
Scientific workflows have become essential in modern research across many scientific domains, supported by Workflow Management Systems (WMSs) that automate resource selection, data management, and task scheduling to optimize performance metrics such as latency, throughput, reliability, or energy consumption. Despite significant engineering progress in WMS design, many fundamental challenges remain unresolved, as theoretical approaches often rely on assumptions that fail to hold in production infrastructures. As a result, most improvements in WMS algorithms and architectures are evaluated experimentally on specific platforms, workflows, and implementations, which makes systematic evaluation and fair comparisons difficult. Addressing this limitation, WRENCH was introduced as a simulation framework that provides accurate, scalable, and expressive experimentation capabilities. Built on SimGrid, WRENCH abstracts away the complexity of simulating distributed infrastructures while preserving realistic models of computation, communication, storage, and failures. It provides a Developer API for implementing simulated WMSs and a User API for creating simulators that run workflows on simulated platforms with minimal code. Through these abstractions, WRENCH allows researchers to test scheduling, resource allocation, and fault-tolerance strategies at scale without the prohibitive cost of real-world deployments. Importantly, WRENCH supports a wide range of distributed computing scenarios, including cloud, cluster, and HPC environments, enabling reproducible and comparative studies of WMS design choices. By lowering the barrier to simulating complex workflows and providing visualization and debugging tools, WRENCH facilitates the systematic exploration of workflow scheduling, performance prediction, and energy-aware computing strategies in controlled yet realistic settings \cite{wrench}.






\clearpage
\section{Related Work}
\label{cha:relatedwork}
The following section introduces related research that has informed and inspired the development of this thesis. It reviews selected works addressing key themes such as workflow monitoring, performance modeling, and scheduling in distributed environments. Much prior research has focused on co-scheduling in high-performance computing (HPC) at the operating system level, on contention management between batch workloads and long-running latency-critical services in cloud data centers, or on multi-objective optimization and metaheuristic approaches for job placement. In contrast, this thesis takes a novel perspective by combining machine learningbased scheduling with online consolidation techniques, explicitly targeting scientific workflows in HPC cluster environments. By embedding this adaptive scheduling logic within a simulation framework that captures resource contention and task-level dynamics, the work extends beyond existing literature and, to the best of our knowledge, represents the first integrated exploration of learning-driven, contention-aware scheduling for workflow execution in simulated HPC systems.

\subsection{Monitoring of Scientific Workflows}
\label{sec:relatedwork_monitoring_scientific_workflows}
Bader et al. present a conceptual framework for advanced monitoring in scientific workflow environments, emphasizing the heterogeneity of metrics and abstraction layers involved in large-scale workflow execution. Their work identifies four distinct monitoring layers—spanning from infrastructure-level telemetry to workflow-level behavior—and positions these as an architectural blueprint for integrating and correlating performance observations across distributed systems. The authors analyze several state-of-the-art workflow management systems to assess their monitoring capabilities and to highlight existing fragmentation in metric collection and interpretation. Their approach is primarily descriptive and infrastructural, focusing on how to organize and harmonize monitoring responsibilities across system components.
In contrast, the work presented in this thesis goes beyond the architectural structuring of monitoring layers and instead operationalizes monitoring data for performance modeling and scheduling decisions. While Bader et al. focus on multi-layer metric aggregation, the present work uses collected measurements—such as runtime and power consumption under co-location—to infer quantitative models of interference and task affinity. Thus, where their contribution establishes the foundation for integrated observability in workflow systems, this thesis builds upon that idea by applying the monitored data directly to predictive, learning-driven scheduling mechanisms that adapt resource allocation based on empirically derived contention profiles \cite{Bader_2022}.

Witzke et al. address the persistent challenge of connecting low-level monitoring data with high-level workflow semantics in distributed scientific computing environments. Their work focuses on tracing I/O behavior across heterogeneous compute nodes, where workflow tasks may execute concurrently and share resources. By integrating system-level telemetry with metadata extracted from workflow logs and container orchestration frameworks such as Kubernetes, the authors propose a methodology to attribute observed resource consumption—particularly I/O activity—to specific workflow tasks. This enables a task-level analysis of performance bottlenecks and inefficiencies, offering valuable insights for optimization. In contrast to their work, which centers on establishing a traceability chain between metrics and logical workflow components, the approach presented in this thesis operates at a higher abstraction level by using modeled and predicted resource interactions to inform scheduling decisions dynamically. Rather than correlating observed traces post-execution, it leverages learned contention patterns and predictive models to influence online task placement, thus turning monitoring insights into proactive scheduling intelligence \cite{Witzke2024}.

\subsection{Characterization of Scientific Workflow Tasks}
\label{sec:relatedwork_characterization_scientific_workflow_tasks}
% Add a couple more examples here at the end for general HPC characterization
Characterizing scientific workflow applications has been a central topic in understanding performance variability and optimization opportunities in large-scale computing environments. Early foundational work by Juve et al. systematically profiled diverse scientific workflows from domains such as astronomy, bioinformatics, and seismic modeling, revealing that despite differences in scientific goals, workflows often exhibit recurring structural and computational patterns. Their study demonstrated that a few dominant job types typically account for most of the total runtime and I/O activity, and that inefficiencies often arise from repeated data access or imbalanced task configurations. This characterization work established the importance of profiling-based insights as a prerequisite for improving workflow scheduling and system utilization.

Subsequent research extended this line of inquiry toward predictive and analytical modeling. Bader et al. conducted a broad survey of workflow task runtime prediction methods, categorizing models by their statistical or machine learning approach, training mode (offline, online, or pre-execution), and level of infrastructure awareness. Their synthesis demonstrated that accurate task-level performance prediction requires both application features and system state information, including heterogeneity and GPU support. In the context of this thesis, such predictive modeling is regarded as a necessary step toward understanding not just isolated task behavior but also inter-task interference when workflows are executed under shared resource conditions.
A related direction emerged from Zhu et al., who developed power- and performance-aware workflow consolidation models based on kernel canonical correlation analysis. Their work introduced temporal signatures to represent CPU, memory, and I/O behavior as time-series features and correlated them with task runtime and power consumption. While their focus remained on offline analysis and consolidation, this approach inspired the feature representation adopted in this thesis, where resource usage patterns are used to model contention and energy impact during online workflow scheduling.
Complementing these modeling efforts, Brondolin et al. introduced DEEP-Mon, a monitoring framework capable of attributing fine-grained power consumption to containerized workloads with negligible system overhead. This kind of container-level observability provides the instrumentation backbone necessary for empirically characterizing workflow behavior, as it enables associating power and performance traces with specific workflow tasks and execution contexts.
Collectively, these works advanced the field from static profiling toward analytical and predictive characterization of workflow behavior. However, they generally focused on offline measurement, individual task modeling, or coarse-grained consolidation in data center contexts.

\subsection{Task Co-location in Scientific Workflows}
\label{sec:relatedwork_task_colocation_scientific_workflows}
% Mention a couple core-level or os-level co-location things here.

Recent research on task co-location has increasingly focused on balancing performance isolation with resource efficiency, particularly in heterogeneous or multi-tenant computing environments. While the motivation—to improve utilization through concurrent execution—is common across contexts, the techniques differ widely depending on whether the target system is a cloud, containerized service infrastructure, or HPC cluster.

Task co-location has become a central strategy for improving resource utilization and energy efficiency in large-scale distributed systems. Early work by Zhu et al. introduced pSciMapper, a consolidation framework designed for scientific workflows in virtualized environments. Their approach treated consolidation as a hierarchical clustering problem, using kernel canonical correlation analysis to model interference between CPU, memory, disk, and network resource profiles. By correlating resource requirements with task runtime and power consumption, pSciMapper achieved significant power savings with minimal slowdown. However, it operated primarily in an offline mode—optimizing workflow placement decisions before execution—and focused on static virtualized environments rather than dynamic HPC workflows.

More recent research has expanded co-location techniques toward data- and container-level optimization. WOW (Workflow-Aware Data Movement and Task Scheduling) by Lehmann et al. proposed a coupled scheduling mechanism that simultaneously steers data movement and task placement to minimize I/O latency and network congestion during workflow execution. Implemented in Nextflow and deployed on Kubernetes, WOW demonstrated how co-location of data and tasks can drastically improve makespan for dynamic scientific workflows. This approach, however, primarily addresses data movement and distributed file system bottlenecks, rather than the interplay of compute, memory, and I/O contention among co-located workflow tasks.

Complementary to WOW, CoLoc by Renner et al. explored distributed data and container co-location in analytic frameworks such as Spark and Flink. By pre-aligning file placement and container scheduling on Hadoop YARN and HDFS, CoLoc improved data locality and reduced network overhead, resulting in execution time reductions of up to 35\%. This work demonstrated the benefits of cross-layer scheduling coordination but focused on recurring, data-intensive jobs rather than the complex, dependency-driven task graphs typical of scientific workflows.

In cloud and service-oriented systems, studies such as OLPart (Chen et al., 2023) and the Interference-Aware Container Orchestration framework (Li et al., 2023) address the challenge of resource partitioning under performance interference. OLPart employs online learning via contextual multi-armed bandits to dynamically partition CPU and memory resources between latency-critical and best-effort jobs, using performance counters to infer sensitivity to contention. Its strength lies in operating without prior workload knowledge, adapting to runtime behavior. Similarly, Li et al. propose a machine learningdriven orchestration approach that introduces scheduling latency as a more accurate interference metric for mixed workloads. Their system predicts host-level interference and adjusts resource allocation in real time to preserve QoS while maximizing utilization. Both works demonstrate how adaptive learning mechanisms can manage contention in containerized or service-oriented environments but are inherently focused on online services rather than scientific workflows with inter-task dependencies.

In the HPC domain, co-location has traditionally been explored through hardware-level modeling and scheduling heuristics. Zacarias et al. present an intelligent resource manager that predicts performance degradation between co-located HPC workloads using performance monitoring counters (PMCs) as model features. The scheduler then selects application mixes that minimize degradation while improving node utilization. This approach achieves measurable improvements over conventional batch scheduling, but it remains focused on pairwise job co-location at the job-manager level and lacks integration with workflow-level task orchestration. Complementary to this, Álvarez et al. introduce nOS-V, a system-wide scheduler enabling fine-grained co-execution of HPC applications at the task level. By dynamically managing shared node resources and bypassing traditional oversubscription mechanisms, nOS-V demonstrates notable gains in throughput and resource efficiency, though it does not incorporate predictive or learning-based adaptation.

A broad body of research has addressed the challenge of performance interference in multi-tenant and consolidated environments, where multiple tasks, processes, or virtual machines share physical resources. Much of this work stems from efforts to optimize system throughput, energy efficiency, and fairness in cloud and HPC contexts, yet the focus typically lies at the level of system services or job managers rather than workflow-specific task scheduling.

Early work by Dwyer et al. pioneered the use of machine learning for online interference estimation on multi-core processors. Their model predicts performance degradation in real time without instrumenting workloads, helping operators in data centers and HPC clusters make informed consolidation decisions. This approach was notable for applying learning-based methods to shared-resource contention but focused primarily on thread- and process-level interference rather than workflow-level orchestration. Complementary to this, Breitbart et al. analyzed co-scheduling of memory-bound and compute-bound applications in supercomputing environments. Their AutoPin+ tool automatically determines suitable combinations of applications to maximize throughput and energy efficiency, demonstrating up to 28\% runtime reduction. However, their strategy remains reactive and application-type specific, without predictive or adaptive capabilities.
Further refinement came from the work of Alves and Drummond, who developed a multivariate model for predicting cross-application interference in virtualized HPC environments. By considering simultaneous access patterns to shared caches, DRAM, and network interfaces, they achieved high accuracy in predicting interference effects across workloads. This work, along with their later Interference-aware Virtual Machine Placement Problem (IVMPP) formulation, demonstrated that combining interference modeling with optimization can effectively reduce degradation while minimizing physical machine usage. Nonetheless, these methods were still anchored in offline modeling and VM placement optimization rather than dynamic or online scheduling.
In data center research, adaptive resource management has evolved toward online learning and multi-objective optimization. The Orchid framework (Chen et al.) and OLPart system introduced contextual multi-armed bandit algorithms for real-time resource partitioning among co-located jobs. These approaches enable systems to autonomously explore and learn optimal resource splits between latency-critical and best-effort workloads, achieving improved throughput and fairness without prior workload knowledge. Similar efforts, such as ScalCCon by Li et al., addressed scalability challenges in correlation-aware VM consolidation through hierarchical clustering, improving both consolidation speed and performance predictability for large-scale infrastructures.
Finally, Sampaio and Barbosa proposed an interference- and power-aware scheduling mechanism incorporating a slowdown estimator that predicts task completion under noisy runtime conditions. Their simulations based on Google Cloud traces demonstrate effective SLA compliance with energy cost reductions, bridging the gap between consolidation efficiency and QoS guarantees.

\subsection{Energy Awareness in Scientific Workflow Execution}
\label{sec:relatedwork_energy_awareness_scientific_workflow_execution}
Related Work: Energy Awareness

Energy measurement at fine granularity has been a recurring challenge across HPC and cloud environments. Raffin and Trystram dissect software-based CPU energy metering with a critical analysis of RAPL mechanisms and an exploration of eBPF as a low-overhead path for accurate, resilient measurements. Their Rust implementation emphasizes pitfalls in timing and counter handling, providing a blueprint for building correct energy profilers on modern x86 platforms. Complementing measurement, Arjona Aroca et al. conduct a component-level characterization of server energy, showing super-linear CPU power behavior and offering validated models that keep estimation error below 5\% for real analytics workloads, while also identifying efficient operating points for NICs and disks. Warade et al. bring these concerns to containers, empirically breaking down the energy footprint of common Dockerized workloads to motivate energy-aware container development and deployment practices.

A parallel line of work focuses on forecasting and control. Algarni et al. evaluate classical time-series models—AR, ARIMA, and ETS—for predicting per-container power, demonstrating that method choice should be tuned to workload/container characteristics, with ETS frequently achieving the lowest MAPE. At the orchestration level, Kuity and Peddoju propose pHPCe, a containerized HPC framework that couples an online LSTM with rolling updates to predict power/performance and a contention-aware power-cap selection mechanism, yielding double-digit power savings at low overhead. In cloud consolidation, Abdessamia et al. apply Binary Gravitational Search for VM placement in heterogeneous data centers, reporting substantial energy savings over first/Best/Worst-fit and particle-swarm baselines. Kumar et al. similarly address green consolidation with task mapping and sleep-state strategies, arguing for policies that explicitly consider both active and idle energy draw to reduce total consumption in multi-tenant clouds.

Energy-aware scheduling for scientific workflows has been studied from both modeling and algorithmic angles. Silva et al. analyze two production I/O-intensive workflows on power-instrumented platforms and show that power is not linearly tied to CPU load; I/O—and even waiting for I/O—materially impacts energy. They introduce an I/O-aware power model that, when integrated into simulation, improves accuracy by orders of magnitude relative to traditional CPU-only models. Building on the importance of accurate models for policy quality, Coleman et al. evaluate popular energy-aware workflow schedulers under this improved model and find that CPU-linearity assumptions can underestimate power by up to 360\% on I/O-heavy workloads; a simple I/O-balancing scheduler guided by the accurate model produces more attractive energymakespan tradeoffs. From an optimization perspective, Durillo, Nae, and Prodan extend HEFT into MOHEFT, a Pareto-based multi-objective scheduler that captures empirical energy behavior on heterogeneous systems, achieving up to 34.5\% energy reductions with marginal makespan impact. Mohammadzadeh et al. pursue metaheuristics, hybridizing Ant Lion Optimizer with Sineosine and chaos-enhanced exploration in WorkflowSim to jointly minimize makespan, cost, and energy for scientific workflows. Choudhary et al. propose a framework that combines task clustering, partial critical path sub-deadlines, and DVFS on compute nodes to reduce transmission and execution energy across several benchmark workflows. Saadi et al. revisit the interplay between clustering and scheduling choices, showing that vertical clustering paired with MaxMin scheduling tends to save energy by lowering makespan, though sensitivity to workflow structure remains.

Energy awareness often intersects with interference management and QoS. Blagodurov and Fedorova advocate contention-aware HPC scheduling, arguing for cluster schedulers that reason about shared resource bottlenecks (e.g., caches, memory controllers) to avoid pathological slowdowns that also waste energy. Sampaio and Barbosa explicitly tie consolidation, slowdown, and power via a slowdown estimator feeding an interference- and power-aware scheduler; simulations based on cloud traces suggest SLA compliance with ~12\% cost reductions. Finally, several lines of work underscore that accurate performance/energy prediction is a precondition for effective energy-aware orchestration: from measurement robustness (Raffin and Trystram) and component characterization (Arjona Aroca et al.), to container-level prediction (Algarni et al., Warade et al.), to model-informed workflow scheduling (Silva et al., Coleman et al., Durillo et al.). Together, these studies advance the state of practice in measuring, modeling, and optimizing energy across the stack—from microarchitectural counters and container runtimes up to workflow-level schedulers and consolidation controllers.
\clearpage
\section{Approach}
\label{cha:approach}
% TODO: Add references to the sections and prior chapters.
% Section: OK
The following section systematically outlines the methodological approach of this thesis. While concrete implementation details and technology-specific decisions are discussed in the subsequent chapter, this section focuses on establishing the theoretical foundation that builds upon the concepts introduced in the background. The proposed research problem is addressed through a threefold decomposition. First, a data collection phase captures detailed execution metrics from scientific workflow runs. Second, this collected data undergoes in-depth analysis to identify relevant performance characteristics and relationships. Finally, the insights derived from this analysis are employed in the simulation and algorithmic modeling phase, where various co-location strategies are evaluated to study their effects on performance and energy efficiency.

The structure of this chapter is organized as follows. After defining the problem statement, a general overview of the methodological approach adopted in this work is presented. Subsequently, a set of assumptions is introduced to clearly delimit the scope, applicability, and limitations of the proposed approach. The chapter then begins with the discussion of online scientific workflow task monitoring, detailing how execution data is collected and structured for further analysis. This is followed by an in-depth explanation of the data analysis phase, focusing on matching task entities from different monitoring sources and leveraging this unified data for statistical exploration. The analysis section begins with embedding methodologies and supervised learning, encompassing data preprocessing and predictive modeling techniques. Thereafter, the focus shifts to unsupervised learning, specifically addressing task clustering as applied in this work. The chapter concludes with a detailed presentation of the theoretical approach to the simulation environment, outlining the main conceptual framework, heuristic design principles for scientific workflow scheduling, the definition of baseline algorithms, and the algorithms devised to implement the novel online co-location strategies developed in this thesis.

\subsection{Central Objective}
\label{sec:problem_statement}
% Section: OK
The central objective of this work is inspired by the design proposed in \cite{5644899}, where task co-location is formulated as a consolidation and clustering problem within a virtualized computing environment. The goal is to consolidate workflow tasks—subject to the structural constraints imposed by the workflows Directed Acyclic Graph (DAG)—onto virtualized machines in such a way that their resource usage profiles complement each other, thereby achieving energy-aware execution. While \cite{5644899} approaches this problem statically, determining task clusters and node assignments prior to workflow execution, this thesis extends the problem to a dynamic, online setting.
In contrast to the static mapping-based approach, this work integrates co-location directly into the task mapping and scheduling process, arguing that co-location and scheduling are inherently interdependent and should be addressed in a unified way rather than separate optimization problems. Consequently, the formulated problem becomes an online co-location problem, where workflow tasks must be characterized before execution in order to enable contention-aware co-location decisions at runtime. The co-location in this context operates at the virtual container level, specifically focusing on virtual machines hosted on physical servers, while contention effects between virtual machines themselves are considered beyond the scope of this thesis.

% Approach Overview Figure
\begin{figure}[htbp]
    \centering
    \includegraphics[width=\textwidth]{/home/nfomin3/dev/ThesisDocument/fig/04/04-overview.pdf} % Adjust the path if needed
    \caption{Overview of the approach consisting of 6 steps.}
    \label{fig:approach_overview}
\end{figure}

\subsubsection{Assumptions}
\label{sec:assumptions}
% Section: OK
To clearly outline the scope, boundaries, and methodological constraints of this work, the following guiding assumptions were defined:

\begin{enumerate}
    \item \textbf{Monitoring Configuration Limits}: A maximum of 80 monitoring features per workflow task is imposed to ensure manageable execution times and allow for statistical evaluation across varying monitoring configurations. This restriction underscores the need for future work to investigate the influence of monitoring data quality and dimensionality on predictive performance.
    \item \textbf{Monitoring Data Coverage}: Short-lived tasks (typically under one second) are only partially captured or occasionally missed due to system load and sampling intervals exceeding one second.
    \item \textbf{Offline Data Analysis}: All data preprocessing, model training, hyperparameter tuning, and fitting are performed offline after workflow execution. The resulting trained models are then transformed into a suitable format for integration into the simulation environment. Nevertheless remains the co-locaiton approach online during simulation accessing the data gathered during prior execution.
    \item Simulation Environment and Platform Equivalence: The simulated platform is assumed to approximate the physical execution environment. It is expected that the overall behavior observed in simulation aligns with real-world execution trends.
    \item Simulation Capabilities and Contention Modeling: The WRENCH framework currently supports simulation of memory contention by limiting per-Host memory consumption, where exceeding the limit results in extended task execution times. Similarly, CPU contention is modeled through proportional increases in task runtime. Other low-level contention effects like cache interference, interconnect bottlenecks, or I/O contention are not modeled in this iteration.
    \item Energy Model Assumptions: The energy model provided by SimGrid is assumed to realistically approximate energy consumption variations when tasks with differing resource usage profiles are colocated on the same virtual machine. The impact on energy efficiency is attributed to CPU utilization behavior and derived from the platform description where different load levels map to consumed energy amount.
    \item Evaluation Focus: Co-location efficiency is evaluated primarily through per-host energy consumption over time, total workflow energy usage, and overall makespan, which serve as the main indicators of effective virtual machine co-location.
\end{enumerate}

\subsection{Online Task Monitoring}
\label{sec:online_task_monitoring}

The online task monitoring approach builds on a hierarchical monitoring architecture that tries to captures a wide spectrum of metrics relevant to the execution of scientific workflows. The design is inspired by \cite{Bader_2022} targeting a four-layered structure consisting of the Resource Manager, Workflow, Machine, and Task layers. Each layer represents a distinct abstraction level in the workflow execution environment and provides valuable insights into different aspects of performance and resource utilization.

Starting with the Resource Manager layer a coarse-grained view related to cluster or partition status, resource allocation, and job management is provided. The Workflow layer operates on the logical workflow representation, maintaining execution progress, task dependencies, and overall runtime statistics. Below, the Machine layer captures system-level performance data such as CPU, memory, and storage utilization, as well as hardware-specific configurations. At the lowest level, the Task layer delivers fine-grained, time-resolved monitoring data, including per-task resource consumption, low-level kernel metrics, and execution traces.
\cite{Bader_2022}.

The following section provides a brief overview of the data sources and monitoring technologies that were ultimately assigned to the four layers of interest and thus selected for the approach. While the accompanying table presents a broader view of the different tools and data collection options that were tested the text below focuses only on the specific data sources that proved to be most reliable and effective within the implemented monitoring framework.

% TODO: Refine contents of the table here.
\begin{table}[htbp]
    \centering
    \setlength{\tabcolsep}{6pt} % adjust horizontal padding
    \renewcommand{\arraystretch}{1.2} % adjust vertical spacing
    \small % reduce overall text size to fit page
    \resizebox{\textwidth}{!}{
        \begin{tabular}{lccccccc}
            \toprule
            \textbf{Monitoring Features}        &
            \multicolumn{7}{c}{\textbf{Data Sources}}                                                                                                                                                                                  \\
            \cmidrule(lr){2-8}
                                                & \textbf{cAdvisor} & \textbf{ebpf-energy-exporter} & \textbf{docker-activity} & \textbf{scaphandre} & \textbf{slurm-exporter} & \textbf{process-exporter} & \textbf{cgroups-exporter} \\
            \midrule
            \multicolumn{8}{l}{\textbf{Resource Manager}}                                                                                                                                                                              \\[3pt]
            Infrastructure status               & x                 &                               &                          &                     &                         &                                                       \\
            File system status                  & x                 &                               &                          &                     &                         &                           &                           \\
            Running workflows                   & x                 &                               &                          &                     &                         &                           &                           \\
            \midrule
            \multicolumn{8}{l}{\textbf{Workflow}}                                                                                                                                                                                      \\[3pt]
            Status                              & x                 & x                             &                          &                     &                         &                           &                           \\
            Workflow specification              &                   & x                             &                          &                     &                         &                           &                           \\
            Graphical representation            &                   & x                             &                          &                     &                         &                           &                           \\
            Workflow ID                         & x                 & x                             &                          &                     &                         &                           &                           \\
            Execution report                    &                   & x                             &                          &                     &                         &                           &                           \\
            Previous executions                 &                   & x                             &                          &                     &                         &                           &                           \\
            \midrule
            \multicolumn{8}{l}{\textbf{Machine}}                                                                                                                                                                                       \\[3pt]
            Status                              &                   &                               & x                        & x                   &                         &                           &                           \\
            Machine type                        &                   &                               & x                        & x                   &                         &                           &                           \\
            Hardware specification              &                   &                               & x                        & x                   &                         &                           &                           \\
            Available resources                 &                   &                               & x                        & x                   &                         &                           &                           \\
            Used resources                      &                   &                               & x                        & x                   &                         &                           &                           \\
            \midrule
            \multicolumn{8}{l}{\textbf{Task}}                                                                                                                                                                                          \\[3pt]
            Task status                         &                   &                               &                          & x                   &                         &                           &                           \\
            Requested resources                 &                   &                               &                          & x                   &                         &                           &                           \\
            Consumed resources                  &                   &                               &                          & x                   &                         &                           &                           \\
            Resource consumption for code parts &                   &                               &                          & x                   &                         &                           &                           \\
            Task ID                             &                   &                               &                          & x                   &                         &                           &                           \\
            Application logs                    &                   &                               &                          & x                   &                         &                           &                           \\
            Task duration                       &                   &                               &                          & x                   &                         &                           &                           \\
            Low-level task metrics              &                   &                               &                          & x                   &                         &                           &                           \\
            Fault diagnosis                     &                   &                               &                          & x                   &                         &                           &                           \\
            \bottomrule
        \end{tabular}
    }
    \caption{Technologies and their capabilities evaluated against a 4-layered Monitoring Framework for Scientific Workflows.}
    \label{tab:monitoring-features}
\end{table}

Although monitoring plays a central role in enabling the presented methodology, the main objective of this thesis is not to evaluate the performance of monitoring tools or to develop new low-level monitoring frameworks for scientific workflows. Therefore, a detailed technical assessment of these tools is not provided here and will not be revisited in Chapter 6. Instead, a concise justification for the distinction between the tools that were experimentally explored and are shown in \ref{tab:monitoring-features} and those that were finally chosen and used for actual data collection within the evaluation setup.

As shown in the \ref{tab:monitoring-features}, the monitoring approach in this work was built around Prometheus as the chosen time-series database. Consequently, all monitoring tools were evaluated based on their interoperability with Prometheus, specifically their ability to function as exporters. The table lists the technologies that were examined for this purpose.

Since the main objective was to gain insight into workflow tasks executed as containers, the focus was on capturing detailed information at the container level and above. However, container-level data can sometimes result in non-identifiable entities, as container names may represent metadata that cannot be directly linked to specific workflow tasks or workflow instances. Therefore, the SMWS's built-in tracing capabilities remained valuable, as they provide a reliable mapping between workflow logic and system-level execution data.

Before describing the specific tools that were ultimately used for monitoring, it is important to briefly outline why certain exporters were excluded from further use. The Slurm exporter, a custom implementation, was originally intended to retrieve job-to-process mappings from the resource manager. However, equivalent information could also be obtained directly from the workflow engines tracing mechanism, and because Slurm's daemons introduced considerable overhead, the exporter was not used in later stages. The process exporter was also omitted due to difficulties in collecting and aggregating all process IDs associated with a single container. In many cases, only the initial shell process of a workflow task was captured, while its spawned subprocesses remained untracked.

Scaphandre, despite appearing promising at first, showed limited compatibility with AMD hardware, which was used in this work. It also struggled to capture short-lived tasks from the workflow management system, leading to incomplete datasets. Docker Activity, another RAPL-based exporter, performed reasonably well but was still constrained by its dependence on Intel power counters, which are only partially supported on AMD CPUs.

In the end, the most consistent and comprehensive results were achieved using cAdvisor and a custom eBPF-based exporter named ebpf-monitor, a modified fork of DEEP-mon \cite{8425477}. These two tools demonstrated the best stability and coverage across different task durations and resource types and were therefore selected for the final monitoring setup.

The following table presents the final selection of data sources that were retained for the monitoring setup, along with the specific metrics enabled for each source.
% TODO: Throw out the exporters that i did not end up using from the table.
% Table with Tool Overview and metrics
\begin{table}[htbp]
    \centering
    \setlength{\tabcolsep}{6pt}
    \renewcommand{\arraystretch}{1.2}
    \small
    \caption{Preliminary simulation results for a subset of workflows showing the overall improvement for both runtime and energy consumption compared to the average of 2 naive baselines without co-location.}
    \label{tab:workflow_results}
    % \resizebox{\textwidth}{!}{%
    \resizebox{\textwidth}{0.46\textheight}{%
        \begin{tabular}{
            >{\raggedright\arraybackslash}p{2.7cm}
            >{\raggedright\arraybackslash}p{3.3cm}
            >{\raggedright\arraybackslash}p{7cm}
            >{\raggedright\arraybackslash}p{2.7cm}
            }
            \toprule
            \textbf{Software Tool} & \textbf{Primary Focus}                  & \textbf{Used Metrics}                               & \textbf{Comment}                                                           \\
            \midrule
            nextflow               & Scientific Workflow Engine              & trace file summary                                  & Used for WfFormat generation and matching containers to nextflow processes \\
            \midrule
            cAdvisor               & Container Performance Monitor           & \makecell[l]{container\_memory\_working\_set\_bytes                                                                              \\container\_memory\_usage\_bytes\\container\_memory\_rss\\container\_fs\_reads\_bytes\_total\\container\_fs\_writes\_bytes\_total\\container\_fs\_io\_current} & 11191 \\
            \midrule
            ebpf-monitor           & Container Energy \& Performance Monitor & \makecell[l]{container\_memory\_working\_set\_bytes                                                                              \\container\_memory\_usage\_bytes\\container\_memory\_rss\\container\_fs\_reads\_bytes\_total\\container\_fs\_writes\_bytes\_total\\container\_fs\_io\_current\\container\_mem\_rss\\container\_mem\_pss\\container\_mem\_uss\\container\_kb\_r\\container\_kb\_w\\container\_num\_reads\\container\_disk\_avg\_lat\\container\_num\_writes\\container\_cycles\\container\_cpu\_usage\\container\_cache\_misses\\container\_cache\_refs\\container\_weighted\_cycles\\container\_power\\container\_instruction\_retired} & 39216.4 \\
            \bottomrule
        \end{tabular}%
    }
\end{table}

% Description of cadvisor, ebpf-energy-monitor
cAdvisor is an open-source daemon for monitoring resource usage and performance of containers. It continuously discovers containers via Linux cgroups under the path /sys/fs/cgroup. Once started, cAdvisor subscribes to create/delete events in the cgroup filesystem, converts them to internal add/remove events, and configures per-container handlers. Metrics originate from machine-level facts parsed from /proc and /sys directories and most prominently container and process usage collected at cgroup boundaries. In practice, cAdvisor provides low-overhead, per-container telemetry.%\cite{Tol}.

The ebpf-monitor is based on the DEEP-Mon system, which is per-thread power attribution method to translate coarse-grained hardware power measurements into fine-grained, thread-level energy estimates by exploiting hardware performance counters. The Intel RAPL interface provides power readings per processor package or core, but it cannot distinguish how much of that energy was consumed by each thread. DEEP-Mon bridges this gap by observing how actively each thread uses the processor during each sampling interval. It does so by monitoring the number of unhalted core cycles—a counter that measures how long a core spends executing instructions rather than idling. Since power consumption correlates almost linearly with unhalted core cycles, the fraction of total cycles attributed to each thread provides a reasonable proxy for its share of energy usage.
In essence, DEEP-mon first computes the weighted cycles for each thread—combining its active cycles when alone with its proportionally reduced cycles when co-running. These weighted cycles determine how much of the total core-level RAPL energy should be assigned to that thread. The final per-thread power estimate is then derived by distributing the total measured power of each socket proportionally to the weighted cycle counts of all threads running on that socket. This approach allows DEEP-mon to infer realistic thread-level power usage even in systems with simultaneous multithreading and time-shared workloads, without modifying the scheduler or requiring any application-specific instrumentation. The DEEP-mon tool was modified in this work to export container-level metrics directly to Prometheus.

Based on this identification of relevant metrics and the according technologies, we now introduce the monitoring algorithm built on top of the selected exporter stack and the resulting overview of the needed components for implementing a suitable monitoring client that can efficiently behave like specified in \ref{tab:monitoring_layers}. It outlines how these components were combined to realize the monitoring logic in this work, while the technical implementation details are described in \ref{cha:implementation}.

First, we outline the core components that form the monitoring client and describe the strategy used to retrieve raw data from the monitored system.
\begin{figure}[htbp]
    \centering
    \includegraphics[width=\textwidth]{/home/nfomin3/dev/ThesisDocument/fig/04/04-monitoring.pdf} % Adjust the path if needed
    \caption{The monitoring client alongside the needed system components.}
    \label{fig:monitoring_client}
\end{figure}

\ref{fig:monitoring_client} and \ref{alg:monitoring} show the principle behavior of the monitoring client. The event listener continuously observes container lifecycle events—such as start and termination—emitted by the runtime environment. Once an event is detected, the data query interface dynamically formulates PromQL queries based on the configuration and metadata of the affected container. These queries are sent to Prometheus to retrieve fine-grained time-series data. The retrieved raw metrics are then passed to the aggregation module, which aligns and consolidates them into a unified record per container.

\begin{algorithm}[H]
    \caption{Event-Driven Monitoring and Metric Aggregation Framework}
    \label{alg:monitoring}
    \KwIn{Configuration $C$ defining monitoring targets}
    \KwOut{Aggregated container-level time-series metrics for each Nextflow process}

    \BlankLine
    initialize\_monitoring(C)\\
    init\_event\_listener()\\
    \While{true}{
        event $\gets$ wait\_for\_container\_event()\\
        meta $\gets$ extract\_metadata(event)\\
        \If{event.type = START}{
            register\_process(meta.id, meta.workflow\_label)
        }
        \If{event.type = TERMINATE}{
            metrics $\gets$ \{\}\\
            \ForEach{target $\in$ C.targets}{
                query $\gets$ build\_promql(target, meta.id, meta.time\_window)\\
                metrics[target] $\gets$ execute\_prometheus\_query(query)
            }
            data $\gets$ aggregate(metrics, meta.workflow\_label)\\
            store(data)
        }
    }
\end{algorithm}

% Algorithm Description
The design introduced in \ref{alg:monitoring} reduces monitoring overhead and ensures that only relevant data are collected when workflow tasks actually execute. When a container start event is received, the client extracts metadata such as container ID, associated workflow label, and start timestamp, then registers the process in an internal mapping to maintain correspondence between container identifiers and workflow tasks. Upon receiving a termination event, the client triggers a targeted data collection phase. The monitoring client allows for dynamic configuration of the metrics mentioned in \ref{tab:monitoring-features}, enabling data analysis specifically scoped to the user's needs.

\subsection{Modelling Co-location of Workflow Task Behavior using Raw Time-Series Data}
\label{sec:data_analysis}
The next section describes the approach used to analyze the data collected through the monitoring client. The focus lies on how the raw metrics retrieved from Prometheus and other exporters were preprocessed, structured, and aligned to support both supervised and unsupervised learning tasks.

% Paragraph about generic preprocessing of raw data
\begin{enumerate}
    \item Parsing of raw time-series CSV files.
    \item Alignment of timestamps across different sources.
    \item Merging of per-task data into unified records.
    \item Matching Entities between execution layer and workflow layer
\end{enumerate}

% Add paragraphs for the enumeration steps?
The first phase of the data analysis focuses on entity matching of heterogeneous monitoring data sources into a unified representation of each executed workflow task. During workflow execution, diverse monitoring tools and system components produce data at different abstraction levels.

% TODO: Add visualization of the data formats
The matching process begins with container lifecycle records, which provide information on task identifiers, container names, process hashes, and working directories. These files serve as the primary link between workflow-level entities and low-level monitoring data. Each monitoring source is traversed recursively to locate and load the associated time-series data. These data files are organized hierarchically under source-specific directories and often contain multiple interleaved measurements from different tasks. To facilitate task-level analysis, these aggregated time-series datasets need to be split into per-task CSV files.
Each per-task dataset is enriched with contextual metadata. The first enrichment phase attaches the correct working directory to every container trace, allowing file system references.
The next stage introduces workflow-level semantics by matching the per-task monitoring data with the workflows own metadata. Using the exported trace file by the SWMS, the analysis extracts task names and their corresponding working directories. These entries are matched against container-level records from the monitoring data to identify which monitored container corresponds to which logical task in the workflow. The resulting mapping is used to update all per-task monitoring files by appending the correct SWMS process name. This concludes the step of entity matching and preparing the collected data for further statistical analysis.

\subsubsection{Preprocessing Raw Time-Series Data for Task-Clustering}
\label{sec:data_preprocessing_predictive}

\begin{enumerate}
    \item Peak-pattern construction. For every task and monitored workload type, we first derive a peak time series: the raw per-second resource signal is resampled into three-second buckets and the maximum per bucket is retained. When two peak series must be compared, we truncate both to the shorter length so that correlation is computed on aligned vectors without padding artifacts.
    \item Workload-type affinity. Different resource domains interfere to different degrees such as CPU vs CPU peaks are typically more contentious than CPU vs file I/O. We encode this with an empirical affinity score defined in the prior section between workload types. High affinity means higher potential interference when peaks align; low affinity reflects benign coexistence.
\end{enumerate}

\subsubsection{Task Clustering for Contention-Aware Consolidation}
\label{sec:task_clustering}

Using the temporal signature feature representation of the monitored tasks, we establish the basis for energy-aware task co-location. Co-location is treated as a consolidation problem, formulated through a clustering approach. In this context, clustering groups tasks based on a defined distance measure that captures their similarity. To incorporate energy awareness into this objective we extend the distance definition to account for interference between workloads. This is achieved by conducting separate workload experiments designed to measure resource contention and interaction effects. From these experiments, affinity scores are derived to quantify how different workloads influence each other when co-located. These affinity scores are then integrated into the clustering distance function, ensuring that both performance and energy interactions guide the co-location decisions.

Consolidation is formulated as a clustering problem with an important modification: rather than grouping tasks that are similar, the goal is to cluster tasks with complementary resource usage patterns to minimize contention during co-location. Building on the previously established notion of affinity—which quantifies how strongly workloads interfere when sharing resources—the clustering process now incorporates this measure directly into its distance metric. The task task distance increases when two tasks exert pressure on the same resources simultaneously, indicating potential contention, and decreases when their resource usage peaks complement one another.
Algorithm \ref{alg:task_distance_clustering} summarizes the main stages of the task consolidation procedure used in this work. The approach begins by computing pairwise similarities between task signatures, where each signature represents a multi-dimensional profile of resource usage over time. The similarity computation incorporates resource affinity weights derived from the contention experiments, ensuring that tasks are compared not only by magnitude but also by their impact on shared resources such as CPU, memory, and I/O. This yields a resource-aware similarity matrix that reflects how well tasks can coexist on the same node without interference.
Next, a percentile-based threshold is applied to the similarity matrix. This adaptive rule determines when clustering should stop by selecting only the most suitable task pairs according to the defined percentile. In this way, the algorithm avoids arbitrary distance cutoffs and instead adapts to the underlying distribution of similarity values.
Finally, agglomerative clustering is performed using a specified linkage criterion, such as average or complete linkage, to iteratively merge tasks into clusters. Each resulting cluster represents a potential consolidation group, meaning a set of tasks that can be executed together within a single virtual machine due to complementary or non-overlapping resource demands. The output clusters thus provide a structured view of optimal co-location candidates for subsequent scheduling and energy-aware execution strategies.

% Algorithm for Task Consolidation in ShaReComp
\begin{algorithm}[H]
    \caption{ShaReComp - Task Consolidation Algorithm}
    \label{alg:task_distance_clustering}
    \SetKwFunction{Sim}{compute\_workload\_affinity}
    \SetKwFunction{Sim}{compute\_similarity}
    \SetKwFunction{Thresh}{percentile\_threshold}
    \SetKwFunction{Merge}{agglomerative\_merge}

    \KwIn{task signatures $\mathrm{sig}$, affinity weights $w$, percentile $\tau$, linkage}
    \KwOut{clusters $\mathcal{C}$}

    $S \gets$ \Sim($\mathrm{sig}$, $w$) \tcp*[r]{resource-aware similarity}
    $\theta \gets$ \Thresh($S$, $\tau$) \tcp*[r]{percentile-based stopping rule}
    $\mathcal{C} \gets$ \Merge($S$, $\theta$, linkage) \tcp*[r]{agglomerative clustering}

    \textbf{return } $\mathcal{C}$
\end{algorithm}

Next, we introduce the subsequent steps necessary to fullfill \ref{alg:task_distance_clustering}, focusing on how the task data needs to be refined.

% Workload experiments
\paragraph{Measuring Resource Interference of Co-located Benchmarks}
\label{sec:measuring_resource_contention}

The measurement of resource contention follows a two-stage protocol that first establishes isolated baselines for each workload class and then repeats the same workloads under controlled co-location. In the baseline stage, CPU-bound, memory-bound, and file-I/O-bound containers are executed one at a time on pinned logical CPUs. Pinning fixes placement and removes scheduler noise; for I/O experiments the file set is prepared once and cleaned afterward to avoid warm-cache artifacts. Each run records a start and finish timestamp at microsecond resolution and derives the wall-clock duration. In parallel, the monitoring pipeline supplies per-container power time series. After a run finishes, only the power streams belonging to the participating container are retained, aligned to the containers lifetime, and summarized to a representative mean value.
The co-location stage replays the same workloads in pairs to expose interference effects. Pairs are chosen to cover both homogeneous combinations, where both containers stress the same resource class, and heterogeneous combinations, where their dominant pressure differs. Placement again uses CPU pinning. Some experiments bind pairs to siblings on the same physical core to amplify shared-core effects; others place them on distinct cores to isolate memory bandwidth or storage contention. Each co-located container is measured in exactly the same way as in isolation, producing a matched set of durations and power summaries.

% Table for the experiment design


\begin{table}[htbp]
    \centering
    \caption{Summary of Synthetic Benchmarks Used in Evaluation}
    \label{tab:synthetic-benchmarks}
    \begin{tabular}{@{}lcccc@{}}
        \toprule
        \textbf{Benchmark Label}                                                                                                                          & \textbf{Image / Version} & \textbf{Execution Command} & \textbf{Behavior Type} & \textbf{Comments} \\
        \midrule

        \textbf{CPU}                                                                                                                                      &
        \texttt{stress-ng-cpu:latest}                                                                                                                     &
        \texttt{stress-ng --cpu 1 --cpu-method matrixprod --cpu-ops 100000 --metrics-brief}                                                               &
        CPU-bound, matrix computation kernel.                                                                                                             &
        Used to emulate high arithmetic intensity workloads.                                                                                                                                                                                                   \\

        \textbf{Memory (VM)}                                                                                                                              &
        \texttt{stress-ng-mem-vm:latest}                                                                                                                  &
        \texttt{stress-ng --vm 1 --vm-bytes 18G --vm-ops 1000 --metrics-brief}                                                                            &
        Memory-bound workload.                                                                                                                            &
        Tests VM allocation, memory contention, and NUMA effects.                                                                                                                                                                                              \\

        \textbf{File I/O}                                                                                                                                 &
        \texttt{fio:latest}                                                                                                                               &
        \texttt{fio --name seqread --rw read --bs 1M --size 18G --numjobs 1 --readonly=1 --direct=1 --iodepth=32 --ioengine=io\_uring --group\_reporting} &
        I/O-intensive sequential read.                                                                                                                    &
        Evaluates disk and I/O scheduling performance.                                                                                                                                                                                                         \\
        \bottomrule
    \end{tabular}
\end{table}

% Notation for the experiments.
\[
    \textbf{Isolated and Co-located Metrics.}
\]

For each workload \( i \in \{1,2\} \), let
\[
    t_i^{(iso)}, \; t_i^{(coloc)} \quad \text{denote the isolated and co-located runtimes,}
\]
\[
    P_i^{(iso)}, \; P_i^{(coloc)} \quad \text{denote the average isolated and co-located power consumption.}
\]

\[
    \textbf{Per-workload Slowdown Factors.}
\]

For each workload \( i \in \{1, 2\} \), let
\[
    t_i^{(iso)}, \; t_i^{(coloc)} \quad \text{denote isolated and co-located runtimes,}
\]
\[
    P_i^{(iso)}, \; P_i^{(coloc)} \quad \text{denote isolated and co-located average power consumptions.}
\]

The per-workload slowdowns are defined as
\[
    S_i^{(t)} = \frac{t_i^{(coloc)}}{t_i^{(iso)}},
    \qquad
    S_i^{(P)} = 1 + \log\!\left(
    \max\!\left( \frac{P_i^{(coloc)}}{P_i^{(iso)}}, 1 \right)
    \right),
\]
ensuring that both runtime and power slowdowns are
non-negative and at least one in value.

The mean slowdowns across the workload pair are
\[
    \bar{S}^{(t)} = \frac{S_1^{(t)} + S_2^{(t)}}{2},
    \qquad
    \bar{S}^{(P)} = \frac{S_1^{(P)} + S_2^{(P)}}{2}.
\]

A weighted average combines both effects:
\[
    \bar{S} =
    \alpha\, \bar{S}^{(t)} + (1 - \alpha)\, \bar{S}^{(P)},
    \quad \text{with } \alpha \in [0,1],
\]
where higher \(\alpha\) emphasizes runtime effects,
and lower \(\alpha\) gives more weight to power efficiency.

The final combined slowdown factor is
\[
    \bar{S}_{\text{final}} = \max(1, \bar{S}),
\]
guaranteeing that co-location never yields an apparent
speedup (values \(\ge 1\) indicate slowdown).

\[
    \textbf{Affinity Score.}
\]

The affinity score \(A\) quantifies the degree of
interference between two co-located workloads.

First, compute pairwise affinity ratios:
\[
    A^{(t)} =
    \frac{t_1^{(coloc)} + t_2^{(coloc)}}
    {t_1^{(iso)} + t_2^{(iso)}},
    \qquad
    A^{(P)} =
    1 + \log\!\left(
    \max\!\left(
    \frac{P_1^{(coloc)} + P_2^{(coloc)}}
    {P_1^{(iso)} + P_2^{(iso)}},
    1
    \right)
    \right).
\]

A weighted average combines both effects:
\[
    A_{\text{raw}} =
    \alpha\, A^{(t)} + (1 - \alpha)\, A^{(P)},
    \qquad A_{\text{raw}} \ge 1.
\]

The normalized affinity score is then:
\[
    A =
    \frac{1 - \frac{1}{A_{\text{raw}}}}{\beta},
    \qquad A \in [0, 1],
\]
where \(\beta > 0\) controls scaling sensitivity.
Values of $A \approx 0$ indicate minimal interference,
while \(A \to 1\) signifies strong co-location interference.

% TODO: Write that this yields in an affinity score matrix and insert example.
% TODO: Reference the table with some explanations.

\begin{table}[H]
    \centering
    \caption{Affinity scores between workload types indicating co-location compatibility.}
    \label{tab:affinity_scores}
    \begin{tabularx}{\textwidth}{l l c X}
        \toprule
        \textbf{Workload 1} & \textbf{Workload 2} & \textbf{Affinity Score} & \textbf{Comment}                                                                              \\
        \midrule
        mem                 & cpu                 & 0.543                   & Moderate compatibility; memory and CPU workloads can share resources with limited contention. \\
        mem                 & fileio              & 0.148                   & Very low compatibility; strong interference due to I/O and memory bandwidth pressure.         \\
        fileio              & cpu                 & 0.223                   & Low compatibility; CPU workloads cause contention for shared I/O buffers.                     \\
        cpu                 & cpu                 & 0.444                   & Medium self-affinity; CPU-bound tasks compete for cores but remain schedulable.               \\
        mem                 & mem                 & 0.514                   & Moderate self-affinity; memory contention manageable under shared caching.                    \\
        fileio              & fileio              & 0.346                   & Limited self-affinity; file I/O contention degrades throughput under co-location.             \\
        \bottomrule
    \end{tabularx}
\end{table}

From these measurements, contention is characterized by comparing the co-located outcomes against the isolated baselines for the same workloads. For each pair, the procedure derives how much slower the workloads ran together relative to alone and how their average power changed. To avoid overfitting to any single signal, runtime and power effects are aggregated into a single scalar that captures the overall quality of the pairing. Values above a neutral threshold indicate that the pair plays well together, while values below it signal destructive interference. This single number is what the co-location policies use downstream: it serves both as the supervision signal for learning-based components and as the ground truth to validate scheduling heuristics in simulation.


\begin{enumerate}
    \item Anti-similarity distance. For any pair of tasks i,j, we iterate over their workload types and compute two ingredients: (i) the affinity between the two types; (ii) the correlation between their corresponding peak series computed twice, once per type, to capture both sides of the pairing. We then aggregate these terms so that highly correlated peaks in high-affinity domains increase the distance, whereas low or negative correlations in low-affinity pairs decrease it. The result is a symmetric task distance matrix whose off-diagonal entries quantify how bad it would be to co-locate the two tasks, and whose diagonals are zero by definition.

          % Notation for the distance formula
          \[
              \textbf{Inter-Task Distance and Resource Correlation Model}
          \]

          To quantify the similarity and potential contention between two workloads
          \( i \) and \( j \), we define a composite distance measure
          that integrates both resource affinity and correlation of peak usage.
          Each workload utilizes a set of resources
          \( R = \{ \text{CPU}, \text{Memory}, \text{Disk}\} \),
          yielding in total ten pairwise combinations of resource types across two tasks.
          The distance term combines the precomputed affinity score with the
          correlation of peak resource intensities.

          \[
              D_{i,j}
              = \sum_{R_1, R_2}
              \Bigl(
              (\mathrm{aff\text{-}score}(R_1^i, R_2^j))
              \cdot
              \mathrm{Corr}(\mathrm{peak}\,R_1^i, \mathrm{peak}\,R_1^j)
              \cdot
              \mathrm{Corr}(\mathrm{peak}\,R_2^i, \mathrm{peak}\,R_2^j)
              \Bigr)
              \tag{1}
          \]

          where:
          \begin{align}
              R_1^i, R_2^j & \; \text{denote resource types of workloads } i \text{ and } j,                           \\[4pt]
              \mathrm{Corr}(\mathrm{peak}\,R_1^i, \mathrm{peak}\,R_1^j)
                           & \; \text{is the Pearson correlation between the peak usages of resource } R_1,            \\[4pt]
              \mathrm{aff\text{-}score}(R_1^i, R_2^j)
                           & \in [0, 1] \text{ measures the degree of interference between } R_1^i \text{ and } R_2^j.
          \end{align}

          \[
              \textbf{Interpretation}
          \]

          The intuition behind this distance metric is to
          \emph{promote dissimilar task pairings} for co-location.
          If two workloads exhibit highly correlated peak usage on the same resources,
          their corresponding correlation terms will be large,
          thus increasing \( D_{i,j} \) and discouraging their co-location.
          Conversely, tasks with uncorrelated or complementary resource peaks
          yield smaller distance values and are therefore more suitable to merge.

          The affinity score modulates this behavior:
          smaller values of \( \mathrm{aff\text{-}score} \)
          indicate lower interference between resource pairs,
          which can offset strong peak correlations.

          Finally, clustering proceeds by iteratively merging task clusters
          whose inter-cluster distance satisfies:
          \[
              D_{i,j} < \text{merge\_threshold}.
          \]
          This ensures that only compatible workloads, in terms of both
          resource affinity and temporal peak correlation, are grouped together.
    \item \textbf{Threshold selection}. Because the distance matrix is data-dependent, we estimate a merge threshold directly from its empirical distribution. A quantile for e.g., the 20th percentile on the raw distances acts as an automatic cut-level: any pair below this threshold is safe enough to consider for co-location, while pairs above it are kept apart.
    \item \textbf{Agglomerative clustering with precomputed distances}. We run average-linkage agglomerative clustering on the precomputed distance matrix with the chosen distance threshold and no preset cluster count. This yields variable-sized clusters whose members are mutually non-contentious under our metric. Because we use a threshold rather than a fixed k, the method adapts to each workload mix, producing more or fewer groups as warranted by the observed interference structure.
    \item\textbf{From clusters to co-location candidates}. Each cluster defines a candidate co-location set. To make these cluster-level entities usable by predictive models such as introduced in \ref{sec:statistical_embedding_and_learning}, we construct cluster feature vectors by flattening and concatenating the per-task temporal signatures of all members and summing them dimension-wise. This potentially approximates the combined load shape we would see if the clusters tasks were executed in a co-located manner.
\end{enumerate}

\subsubsection{Preprocessing Raw Time-Series Data for Predictive Modelling}
\label{sec:data_preprocessing_predictive}
% TODO: Add formal notation of the feature matrices
We transform the heterogeneous, time‐stamped monitoring traces into consistent task-level feature/label matrices suitable for statistical analysis. We therefore define the following preprocessing steps:

\begin{enumerate}
    \item Temporal signature construction (feature selection).
          \subitem Sampling and smoothing.
          \subitem Equal-length normalization.
          \subitem Container-wise collation.
    \item Model Input Construction.
          \subitem Normalization and Scaling.
          \subitem Extraction of Input Features and Output Labels.
\end{enumerate}

% Formal notation

\[
    \textbf{Temporal Signature and Model Input Construction.}
\]

We denote by \( \mathcal{T} = \{ T_1, T_2, \dots, T_N \} \) the set of
temporal signatures extracted from the monitored resource-usage profiles
of \( N \) workflow tasks.
Each task \( i \) is characterized by time-varying utilization traces
for the monitored resource dimensions
\[
    R = \{ \text{CPU}, \text{Memory}, \text{Disk}, \text{Network} \}.
\]

\noindent
For each resource \( r \in R \) and task \( i \),
the temporal signature \( T_i^{(r)} \) is defined as a
coarse-grained summary of the normalized resource usage signal
\( x_i^{(r)}(t) \):
\[
    T_i^{(r)} =
    \bigl\langle
    p_{1,i}^{(r)},\;
    p_{2,i}^{(r)},\;
    \dots,\;
    p_{10,i}^{(r)}
    \bigr\rangle,
    \tag{1}
\]
where each component \( p_{k,i}^{(r)} \) represents the mean
usage value within segment \( k \) of the time-normalized
execution window (\( k = 1, \dots, 10 \)).
This yields a ten-dimensional vector describing the temporal pattern of
resource consumption.

\noindent
The feature vector for task \( i \) is obtained by concatenating
the resource-specific signatures:
\[
    x_i =
    \bigl[
    T_i^{(\text{CPU})},\;
    T_i^{(\text{Memory})},\;
    T_i^{(\text{Disk})},\;
    T_i^{(\text{Network})}
    \bigr]
    \in \mathbb{R}^{d_x},
\]
where \( d_x = 4 \times 10 = 40 \) in this example configuration.

\[
    X =
    [x_1^\top, x_2^\top, \dots, x_N^\top]^\top
    \in \mathbb{R}^{N \times d_x}
\]
denotes the complete input matrix used for model training.

Similarly, for each task \( i \), the execution-level targets
(time and mean power consumption) are given by
\[
    y_i = [\,t_i,\, P_i\,] \in \mathbb{R}^2,
    \quad
    Y = [y_1^\top, y_2^\top, \dots, y_N^\top]^\top
    \in \mathbb{R}^{N \times 2}.
\]

\[
    \textbf{Example.}
\]
Consider \( N = 3 \) workflow tasks with simplified
CPU and memory usage signatures
(each consisting of 3 representative pattern points for brevity):
\[
    \begin{array}{lcccccc}
        \toprule
        \text{Task}        &
        p_1^{(\text{CPU})} & p_2^{(\text{CPU})} & p_3^{(\text{CPU})} &
        p_1^{(\text{Mem})} & p_2^{(\text{Mem})} & p_3^{(\text{Mem})}                             \\
        \midrule
        1                  & 0.40               & 0.75               & 0.90 & 0.35 & 0.55 & 0.60 \\
        2                  & 0.20               & 0.50               & 0.70 & 0.25 & 0.40 & 0.50 \\
        3                  & 0.30               & 0.65               & 0.85 & 0.30 & 0.45 & 0.55 \\
        \bottomrule
    \end{array}
\]

Concatenating these signatures yields the input matrix
\[
    X =
    \begin{bmatrix}
        0.40 & 0.75 & 0.90 & 0.35 & 0.55 & 0.60 \\
        0.20 & 0.50 & 0.70 & 0.25 & 0.40 & 0.50 \\
        0.30 & 0.65 & 0.85 & 0.30 & 0.45 & 0.55
    \end{bmatrix},
    \quad
    Y =
    \begin{bmatrix}
        12.4 & 65 \\
        14.1 & 72 \\
        10.8 & 58
    \end{bmatrix}.
\]

Here, each row of \( X \) encodes the temporal resource-usage pattern
of a task, while \( Y \) provides the corresponding runtime and mean
power consumption used for model learning or correlation analysis.

This preprocessing yields: (i) a standardized and fixed-length feature matrix X that preserves per-metric usage distributions and (ii) a label matrix Y capturing runtime and energy

\subsubsection{Modelling and Predicting the Impact of Task Behavior on Runtime and Energy Consumption with KCCA and Random Forest Regression}
\label{sec:prediciton_kcca_rfr}

% Algorithm for prediction of task clusters.
\begin{algorithm}[H]
    \caption{ShaReComp — Prediction of Energy and Performance Behavior of Consolidated Task Clusters}
    \label{alg:sharecomp_prediction}
    \SetKwFunction{Aggregate}{sum\_cluster\_features}
    \SetKwFunction{Build}{build\_feature\_matrix}
    \SetKwFunction{Predict}{predict\_runtime\_and\_energy}

    \KwIn{task clusters $\mathcal{C}$, per-task signatures $\mathrm{sig}$, trained model $\mathcal{M}$ (KCCA or Random Forest)}
    \KwOut{predicted runtime energy pairs $\hat{Y} = \{ (\hat{t}_k, \hat{E}_k) \}$}

    \BlankLine
    \ForEach{cluster $C_k \in \mathcal{C}$}{
        $F_k \gets$ \Aggregate($\{\,\mathrm{sig}[t_i] \mid t_i \in C_k\,\}$) \tcp*[r]{sum task signatures to form cluster feature}
    }
    $X \gets$ \Build($\{F_k\}$) \tcp*[r]{construct consolidated feature matrix}

    \BlankLine
    \ForEach{cluster feature $X_k \in X$}{
        $(\hat{t}_k, \hat{E}_k) \gets$ \Predict($X_k$, $\mathcal{M}$)
    }
    \BlankLine
    \textbf{return } $\hat{Y} = \{ (\hat{t}_k, \hat{E}_k) \}_{k=1}^{|\mathcal{C}|}$
\end{algorithm}

\paragraph{Kernel Canonical Correlation Analysis}
In the next step of the analysis, the collected and preprocessed task data are prepared for statistical learning. The objective is to identify relationships between task-specific features and their corresponding performance and energy characteristics. To achieve this, the dataset is divided into two parts: approximately 70\% of the tasks are used for training the models, while the remaining 30\% are reserved for testing and validation.

Before training, both the feature data (X) and the target data (Y)—representing task runtime and energy consumption—are standardized separately. Each is transformed to have a mean of zero and a standard deviation of one. This step is essential, as the original values cover different numerical scales, which could negatively affect kernel-based learning methods. Standardization ensures that all features contribute equally to the learning process and prevents those with larger magnitudes from biasing the results.
After normalization, a Kernel Canonical Correlation Analysis (KCCA) model is trained to identify shared patterns between the feature data and the target data. KCCA projects both datasets into a common latent space, where it can detect nonlinear relationships between resource usage patterns and their corresponding runtime and energy behavior.

Once trained, the KCCA model is extended into a predictive framework. The latent features learned from the input data are used to fit a regression model that links these representations to runtime and energy targets. This allows the system not only to capture statistical relationships but also to predict performance and energy consumption for unseen tasks.

% Notation
\[
    \textbf{Kernel Canonical Correlation Analysis (KCCA).}
\]
To capture nonlinear dependencies between the resource signatures
and performance power outcomes, we apply Gaussian kernels to both
input and output feature spaces.

KCCA seeks directions \(A\) and \(B\) in the
reproducing kernel spaces of \(K_x\) and \(K_y\)
that maximize the correlation between
\( K_x A \) and \( K_y B \).
This is expressed as the generalized eigenvalue problem:
\[
    \begin{bmatrix}
        0       & K_y K_x \\
        K_x K_y & 0
    \end{bmatrix}
    \begin{bmatrix}
        A \\ B
    \end{bmatrix}
    =
    \lambda
    \begin{bmatrix}
        K_x^2 & 0     \\
        0     & K_y^2
    \end{bmatrix}
    \begin{bmatrix}
        A \\ B
    \end{bmatrix}.
    \tag{2}
\]

Solving (2) yields paired canonical directions
\( (A, B) \) that define latent projections
\[
    X' = K_x A, \qquad Y' = K_y B,
\]
maximally correlated across the two feature spaces.
These projections represent a shared latent space
relating resource utilization dynamics to task performance and energy.

\[
    \textbf{Illustrative Example.}
\]
Consider three workflow tasks \( i = 1, 2, 3 \)
with aggregated temporal signatures over CPU and memory:
\[
    x_1 = [0.40,\, 0.75,\, 0.90,\, 0.35,\, 0.55,\, 0.60], \quad
    x_2 = [0.20,\, 0.50,\, 0.70,\, 0.25,\, 0.40,\, 0.50], \quad
    x_3 = [0.30,\, 0.65,\, 0.85,\, 0.30,\, 0.45,\, 0.55].
\]
Their corresponding runtime power outcomes are:
\[
    y_1 = [12.4,\, 65], \quad
    y_2 = [14.1,\, 72], \quad
    y_3 = [10.8,\, 58].
\]

KCCA maps both the temporal patterns \(x_i\)
and the performance power pairs \(y_i\)
into high-dimensional kernel spaces
and finds projections that maximize their mutual correlation.
In this example, the first canonical mode reveals that
tasks with higher sustained CPU activity
(\(x_1, x_3\))
correspond to lower execution time and reduced power consumption,
while the less efficient task (\(x_2\))
shows a distinct temporal signature characterized by
fluctuating utilization and higher runtime.

\paragraph{Random Forest Regression}
To complement the KCCA model, we trained two non-parametric regressors based on Random Forests—one to predict mean task power and one to predict task runtime from the same preprocessed feature matrix. The power model is trained on mean per-task energy-rate labels, while the runtime model uses task durations as targets. As a sanity check, we established simple baselines by predicting the training-set mean once for power and once for runtime on the test split. These baselines quantify the minimum improvement any learned model must exceed.

% Example on how this looks
\textbf{Example.}
Assume two clusters:
\[
    C_1 = \{t_1, t_2\}, \quad C_2 = \{t_3\},
\]
and each task has a CPU Memory signature with three pattern points:
\[
    t_1 = [0.4, 0.7, 0.9, 0.5, 0.6, 0.7], \quad
    t_2 = [0.3, 0.5, 0.8, 0.4, 0.5, 0.6], \quad
    t_3 = [0.2, 0.4, 0.6, 0.3, 0.4, 0.5].
\]
Cluster features are summed:
\[
    F_1 = t_1 + t_2 = [0.7, 1.2, 1.7, 0.9, 1.1, 1.3], \quad
    F_2 = t_3.
\]
Model predictions yield
\[
    \hat{Y} =
    \begin{bmatrix}
        10.8 & 62.5 \\
        14.2 & 75.1
    \end{bmatrix},
\]
representing predicted runtime (s) and energy (W) per cluster.

\subsection{Simulation of Task Co-location during Workflow Execution}
\label{sec:simulation_environment}

\subsubsection{Outline of Simulation Components}
\label{sec:design_pillars}

The simulator for co-location strategies builds upon three fundamental design pillars that reflect the main optimization opportunities identified in the co-location problem: resource allocation, queue ordering, and job placement.

% Figure with Simulator Design
\begin{figure}[htbp]
    \centering
    \includegraphics[width=\textwidth]{/home/nfomin3/dev/ThesisDocument/fig/04/04-monitoring.pdf} % Adjust the path if needed
    \caption{The monitoring client alongside the needed system components.}
    \label{fig:monitoring_client}
\end{figure}

The design of distributed HPC systems centers on three core aspects: resource allocation, queue ordering and dispatching, and job placement. To apply the concepts introduced in \ref{sec:statistical_embedding_and_learning} and \ref{sec:unsupervised_learning}, the simulation of scientific workflow task co-location adheres to these principles, providing a framework that supports multiple strategies for realistic exploration of this objective.
As seen in \ref{fig:monitoring_client}, we structure the concept of the simulator around these three pillars.
The first pillar concerns resource allocation, which determines whether jobs are executed in exclusive or shared mode. Traditional HPC schedulers allocate full nodes to single jobs, task co-location naturally assumes that multiple tasks can coexist efficiently if they do not saturate the same resources simultaneously. The simulator therefore needs to model node-sharing policies where jobs may share cores or memory bandwidth.
Since the effectiveness of co-location depends on the characteristics of jobs and directly influence the throughput of the processing engine the simulator needs to offer an interface for alternative queue management strategies, ranging from conventional FIFO ordering to heuristic reordering that prioritizes beneficial pairings.
The third pillar, job placement, determines how tasks are assigned to available resources. The simulator needs to support random placement and heuristic strategies that intentionally group complementary tasks.
This setup aims at making it possible to study how the integration of task co-location choices during schedulung and task mapping influence overall system performance and resource efficiency, providing a controlled environment for algorithmic investigations.

\subsubsection{Embedding Task Co-location into Scheduling Heuristics}
\label{sec:heuristic_design}

Our algorithmic approach to define the behavior of the components resource manager, allocator and scheduler introduced in \ref{fig:monitoring_client} is grounded in heuristic principles.
We choose to not formulate the embedding of co-location into the scheduling process as optimization problem but through decision-making within simple, interpretable scheduling and task mapping rules. Heuristic algorithms are well suited for such settings because they rely on deterministic, rule-based traversal of the search space rather than exhaustive or stochastic exploration. They exploit domain-specific knowledge, in our case energy-awareness and structured criteria—such as task priorities, resource affinities, or workload complementarities—to produce acceptable solutions efficiently without guaranteeing global optimality. We divide the task-to-node mapping into two steps. First, a queue of ready tasks is generated according to a defined orderign strategy before the mapping to resources of these ordered tasks is performed. For the scheduling we choose to adpot list-scheduling algorithms, which form one of the most established heuristic approaches for workflow scheduling. They assign a priority or ranking to each task based on topological and performance factors (e.g., critical path length, execution cost, or communication overhead), producing a priority list; second, they iteratively select the highest-priority unscheduled task and map it to a processor. As both phases can be modeled decoupled, there opens a chance to incorporate intelligent decision-making.
Machine learning list-scheduling approaches build on data-driven models that learn decision policies from historical workflow executions or directly during runtime. To complete the foundation of our heuristic design we formally introduce the simulation environment, consisting of the scientific workflow as the application that executes tasks and a system model that we use to represent our algorithmic approach to the co-location problem.

% System Model that was used for the heuristic design.
\paragraph{Simulation System Model}

\textbf{Workflow Properties}

Let the workflow be represented as a directed acyclic graph (DAG)
\[
    G = (T, E)
\]
where
\begin{itemize}
    \item $T = \{t_1, t_2, \dots, t_n\}$ denotes the set of \textbf{tasks}, and
    \item $E \subseteq T \times T$ denotes \textbf{data or control dependencies} between tasks.
\end{itemize}
A directed edge $(t_i, t_j) \in E$ indicates that task $t_j$ can start only after $t_i$ has completed.

\textbf{Task Properties}

Each task \( t_i \in T \) is associated with the following attributes:
\[
    \begin{array}{rcll}
        \text{req\_cores}(t_i) & \in & \mathbb{N}_{\ge 1} & \text{number of CPU cores required}, \\[4pt]
        \text{req\_mem}(t_i)   & \in & \mathbb{R}_{>0}    & \text{memory requirement in bytes.}
    \end{array}
\]

\textbf{Infrastructure Model}

Let \( \mathcal{H} = \{h_1, h_2, \dots, h_m\} \) denote the set of available \textbf{hosts},
where each host \( h_j \) is characterized by
\[
    C_j \in \mathbb{N}_{>0} \text{ (total number of cores)}, \qquad
    c(h_j) \in \mathbb{N}_{\ge 0} \text{ (current number of idle cores).}
\]

A \textbf{virtual machine (VM)} is represented as
\[
    v = (C_v, M_v, h_j),
\]
where \( C_v \) is the number of virtual cores, \( M_v \) the assigned memory,
and \( h_j \) the physical host on which it is instantiated.

The set of currently \textbf{ready tasks} (those whose dependencies are satisfied) is denoted by
\[
    \mathcal{Q} = \{t_1, t_2, \dots, t_k\}.
\]

\textbf{Resource Assignment}

A mapping of tasks to hosts and VMs is represented as
\[
    M = (h, \mathcal{T}_h, \text{colocMap}_h, \Phi_h),
\]
where
\begin{itemize}
    \item $h \in \mathcal{H}$ is the assigned host,
    \item $\mathcal{T}_h \subseteq T$ is the set of tasks mapped to $h$,
    \item $\text{colocMap}_h$ describes task clusters on $h$, and
    \item $\Phi_h$ represents associated file or data locations.
\end{itemize}
The set of mappings for an allocation interval is written as
\[
    \mathcal{M} = \{ M_1, M_2, \dots, M_p \}.
\]

\textbf{Task Co-location}

A \textbf{co-location mapping}, produced by a scheduler is defined as
\[
    \text{colocMap} = \{ (C_i, \mathcal{C}_i) \mid \mathcal{C}_i \subseteq T \},
\]
where $\mathcal{C}_i$ is a \textbf{cluster of tasks} to be executed together within a single virtual machine, ideally selected based on their resource affinity or complementary utilization patterns.

\textbf{Oversubscription}

An \textbf{oversubscription factor} $\alpha \in [0,1]$ allows up to
\[
    N_{\max}(h_j) = \lceil C_j (1 + \alpha) \rceil
\]
tasks to be scheduled concurrently on a VM on host $h_j$.


\textbf{Execution Dynamics}

At runtime:
\begin{itemize}
    \item \textbf{Node Assigners} determine host placement.
    \item \textbf{Schedulers} generate task queues and co-location groupings.
    \item \textbf{Allocators} instantiate and start VMs according to host-task mappings.
    \item The \textbf{Job Manager} executes tasks, monitors VM lifecycles, and updates resource states.
\end{itemize}

To evaluate the effectiveness of the proposed approach, we compare it against a series of progressively refined baseline algorithms that address the co-location problem with increasing complexity. These baselines range from simple scheduling heuristics to more advanced strategies that gradually incorporate awareness of co-location effects. The following table summarizes all baselines considered in this study. It is important to note that while some of them already involve concurrent execution of tasks, this represents uncontrolled co-location rather than informed or optimized placement. For clarity and conciseness, detailed algorithmic designs of these baselines are provided in the appendix, while this section focuses on describing their conceptual behavior.

% Blueprint Algorithm 
\paragraph{Baseline Algorithms}
\label{sec:baseline_algorithms}

\begin{algorithm}[H]
    \caption{ShaReComp Simulation - WRENCH Framework}
    \label{alg:sharecomp_unified}
    \KwIn{workflow \( G=(T,E) \), hosts \( \mathcal{H}=\{h_1,\dots,h_m\} \), scheduling policy \( \pi \), oversubscription factor \( \alpha \), optional co-location API \( \mathcal{S} \)}
    \KwOut{workflow executed with policy-driven scheduling, node assignment, and adaptive resource management}

    \BlankLine
    Initialize system state: host capacities, ready queue \( \mathcal{Q} \), and monitoring layer\\

    \While{workflow \( G \) not completed}{
        Update \( \mathcal{Q} \) with all newly ready tasks\\
        Perform \textbf{task scheduling}: prioritize tasks in \( \mathcal{Q} \) according to policy \( \pi \)\\
        Perform \textbf{node assignment}: select suitable host(s) \( h \in \mathcal{H} \) using \( \pi \)\\
        Perform \textbf{resource allocation}: determine allowed capacity \( n_{\max}=f(c(h),\alpha) \) and reserve resources\\
        \If{policy \( \pi \) supports co-location}{
            Optionally query \( \mathcal{S} \) to group tasks by affinity and launch them on shared VMs
        }\Else{
            Launch one task per VM on assigned host
        }
        Monitor execution and task completions\\
        Release resources and enqueue successors of completed tasks
    }
    \Return{workflow complete}
\end{algorithm}

\paragraph{Algorithmic Examples of Task Mapping with guided Co-location}
\label{sec:co-location_strategies}

% Table with ShaRiff Overview stuff.
\begin{table}[H]
    \centering
    \caption{Overview of ShaRiff and ShaReComp Scheduling Algorithms.}
    \label{tab:shariff_overview}
    \begin{tabularx}{\textwidth}{l l X}
        \toprule
        \textbf{Algorithm} & \textbf{Type}                                         & \textbf{Description}                                                                                                                         \\
        \midrule
        ShaRiff 1          & Contention-Aware Co-location                          & Uses the ShaReComp API to determine affinity-based co-location groups, minimizing interference between tasks.                                \\

        ShaRiff 2          & Adaptive Co-location + Over-Subscription              & Extends ShaRiff 1 by allowing safe CPU over-subscription based on affinity predictions.                                                      \\

        ShaRiff 3          & Round-Robin Node Assignment + Co-location             & Schedules tasks round-robin across hosts while applying affinity-based co-location through the ShaReComp API.                                \\

        ShaReComp          & Adaptive Max-Parallel Co-location + Over-Subscription & Integrates parallel scheduling, affinity-based co-location, and controlled over-subscription for optimized throughput and energy efficiency. \\
        \bottomrule
    \end{tabularx}
\end{table}

% Explain and visualize ShaRiff in it's variants algorihtmically and algorihtms after each paragraph.
% ShaRiff 1

% Algorithm
\begin{algorithm}[H]
    \caption{ShaRiff 1 — Contention-aware Max-Parallel VM Co-Location Scheduling}
    \label{alg:sharecomp}
    \KwIn{workflow \( G=(T,E) \), hosts \( \mathcal{H}=\{h_1,\dots,h_m\} \), idle cores \( c(h_j) \), ShaReComp co-location API \( \mathcal{S} \)}
    \KwOut{all tasks \( t_i\in T \) executed with contention-aware co-location for improved efficiency and utilization}

    \BlankLine
    Initialize idle cores \( c(h_j)\gets C_j \) for all \( h_j\in\mathcal{H} \);
    Initialize ready queue \( \mathcal{Q} \) with all source tasks of \( G \) (FIFO order)

    \BlankLine
    \While{not all tasks \( t_i\in T \) completed}{
        \If{\( \mathcal{Q} \) is empty}{
            Wait until any task \( t_r \) completes;
            Release its cores: \( c(h(t_r)) \gets c(h(t_r)) + \text{req\_cores}(t_r) \);
            For each successor \( t_s \) of \( t_r \), enqueue \( t_s \) into \( \mathcal{Q} \) if all predecessors are completed;
            \textbf{continue}
        }

        Build available-host list \( L=\{\,h\in\mathcal{H}\mid c(h)>0\,\} \), sorted by \( c(h) \) descending;
        Initialize empty host task mapping list \( \mathcal{M} \);

        \BlankLine
        \ForEach{host \( h\in L \) and while \( \mathcal{Q} \) not empty}{
            Select up to \( c(h) \) ready tasks from \( \mathcal{Q} \) into \( \mathcal{T}_h \);
            Compute file-location map \( \Phi(\mathcal{T}_h) \);
            Query ShaReComp API: \( \text{colocMap} \gets \mathcal{S}(\mathcal{T}_h) \) \tcp*[r]{returns co-location groups}
            Add mapping \( (h,\mathcal{T}_h,\Phi(\mathcal{T}_h),\text{colocMap}) \) to \( \mathcal{M} \);
        }

        \BlankLine
        \ForEach{mapping \( (h,\mathcal{T}_h,\Phi,\text{colocMap}) \in \mathcal{M} \)}{
            \ForEach{group \( \mathcal{C}_k \in \text{colocMap} \)}{
                \( C_{\text{req}} \gets \text{sum(req\_cores}(\mathcal{C}_k)) \);
                \( M_{\text{req}} \gets \text{sum(req\_mem}(\mathcal{C}_k)) \);
                Allocate \( v_h = (C_{\text{req}}, M_{\text{req}}, h) \);
                Launch all \( t \in \mathcal{C}_k \) on \( v_h \);
                \( c(h) \gets \max(0, c(h) - C_{\text{req}}) \);
                Remove \( \mathcal{C}_k \) from \( \mathcal{Q} \);
            }
        }

        \BlankLine
        Wait until any task \( t_r \) completes;
        Release its cores: \( c(h(t_r)) \gets c(h(t_r)) + \text{req\_cores}(t_r) \);
        \If{no active tasks remain on its VM}{ Destroy VM }
        For each successor \( t_s \) of \( t_r \): if all predecessors are completed, enqueue \( t_s \) into \( \mathcal{Q} \);
    }
    \Return{workflow complete}

\end{algorithm}
This variant implements ShaRiff (share resources if feasible), which augments the FIFO pipeline with an external co-location adviser and a cluster-aware allocator. Tasks are still dequeued in strict arrival order. Before placement, the scheduler invokes ShaRiff with the current set of ready tasks and receives clusters of jobs that are predicted to co-locate well (i.e., complementary resource profiles / low expected interference). The node-assignment stage then ranks hosts by descending idle-core capacity and fills the largest host first: it forms a batch of up to that hosts idle cores and attaches the ShaRiff cluster map to the batch; if tasks remain, it proceeds to the next host in the ranked list. A small-queue fast path ensures dispatch even when only a few tasks are available.
The allocator realizes the advisers plan one VM per recommended cluster on the chosen host. For each multi-task cluster, it provisions a VM whose vCPU count and memory equal the sum of the clustered tasks declared requirements, starts the VM, and submits the tasks to that same virtual compute service. Singleton clusters are grouped into a shared VM on the host (current variant) to avoid VM fragmentation; each submitted task keeps its own job identity, and the allocator tracks VM lifecycle across all tasks assigned to it, tearing the VM down only after the last co-located task completes.
Conceptually, ShaRiff preserves FIFO ordering and capacity-ranked host filling, but replaces random batching with adviser-driven clustering. The effect is to co-locate tasks that are likely complementary (e.g., CPU-bound with I/O-bound), thereby reducing contention and improving per-host utilization without oversubscription. When the adviser yields singletons, the system still shares if feasible by pooling them into a shared VM, maintaining the same deterministic provisioning and lifecycle rules.

% ShaRiff 2

% Algorithm
\begin{algorithm}[H]
    \caption{ShaRiff 2 — Adaptive Max-Parallel VM Co-Location with Controlled Over-Subscription}
    \label{alg:sharecomp_oversub}
    \KwIn{workflow \( G=(T,E) \), hosts \( \mathcal{H}=\{h_1,\dots,h_m\} \), idle cores \( c(h_j) \), oversubscription factor \( \alpha \), ShaReComp co-location API \( \mathcal{S} \)}
    \KwOut{workflow executed with affinity-based co-location, maximal host parallelism, and safe oversubscription}

    \BlankLine
    Initialize idle cores \( c(h_j)\gets C_j \) for all \( h_j\in\mathcal{H} \)
    Initialize ready queue \( \mathcal{Q} \) with all source tasks of \( G \) (FIFO order)

    \BlankLine
    \While{not all tasks \( t_i\in T \) completed}{
        \If{\( \mathcal{Q} \) is empty}{
            Wait until any task \( t_r \) completes
            Release its cores: \( c(h(t_r)) \gets c(h(t_r)) + \text{req\_cores}(t_r) \)
            For each successor \( t_s \) of \( t_r \), enqueue \( t_s \) into \( \mathcal{Q} \) if all predecessors are completed\\
            \textbf{continue}
        }

        Build available-host list \( L=\{\,h\in\mathcal{H}\mid c(h)>0\,\} \), sorted by \( c(h) \) descending
        Initialize empty host task mapping list \( \mathcal{M} \)

        \BlankLine
        \ForEach{host \( h\in L \) and while \( \mathcal{Q} \) not empty}{
            Compute oversubscription limit \( n_{\max} = \lceil c(h) \times (1 + \alpha) \rceil \)
            Select up to \( n_{\max} \) ready tasks from \( \mathcal{Q} \) into \( \mathcal{T}_h \)
            Compute file-location map \( \Phi(\mathcal{T}_h) \)
            Query ShaReComp API: \( \text{colocMap} \gets \mathcal{S}(\mathcal{T}_h) \) \tcp*[r]{returns co-location groups}
            Add mapping \( (h,\mathcal{T}_h,\Phi(\mathcal{T}_h),\text{colocMap}) \) to \( \mathcal{M} \)
        }

        \BlankLine
        \ForEach{mapping \( (h,\mathcal{T}_h,\Phi,\text{colocMap}) \in \mathcal{M} \)}{
            \ForEach{group \( \mathcal{C}_k \in \text{colocMap} \)}{
                \( C_{\text{req}} \gets \text{sum(req\_cores}(\mathcal{C}_k)) \)
                \( M_{\text{req}} \gets \text{sum(req\_mem}(\mathcal{C}_k)) \)
                \If{\( C_{\text{req}} > c(h) \)}{
                    Allocate \( v_h = (c(h), M_{\text{req}}, h) \); \tcp*[r]{oversubscription active}
                }
                \Else{
                    Allocate \( v_h = (C_{\text{req}}, M_{\text{req}}, h) \)
                }
                Launch all \( t \in \mathcal{C}_k \) on \( v_h \)
                \( c(h) \gets \max(0, c(h) - C_{\text{req}}) \)
                Remove \( \mathcal{C}_k \) from \( \mathcal{Q} \)
            }
        }

        \BlankLine
        Wait until any task \( t_r \) completes
        Release its cores: \( c(h(t_r)) \gets c(h(t_r)) + \text{req\_cores}(t_r) \)
        \If{no active tasks remain on its VM}{ Destroy VM }
        For each successor \( t_s \) of \( t_r \): if all predecessors are completed, enqueue \( t_s \) into \( \mathcal{Q} \)
    }
    \Return{workflow complete}
\end{algorithm}

This variant retains the ShaRiff-augmented FIFO pipeline but enables controlled oversubscription during placement and VM sizing. Tasks are dequeued in arrival order. Before dispatch, the scheduler queries ShaRiff with the current ready set and receives clusters of tasks predicted to co-locate well (complementary resource use / low interference). Hosts are ranked by descending idle-core capacity; the assigner then fills the largest host first with a batch whose size may exceed the hosts free cores by a fixed factor (e.g., +25\%). If tasks remain, it proceeds to the next host and repeats.
The allocator implements the advisers plan one VM per cluster on the chosen host, but with oversubscription semantics. For multi-task clusters, it provisions a VM whose vCPU and memory equal the sum of the clusters requests—even if that exceeds the hosts currently free cores (hard oversub). For singleton clusters collected on the same host, it provisions a shared VM and caps vCPUs at the hosts free cores when necessary (soft cap). In both cases the VM is started once, all tasks in the cluster are submitted to the same virtual compute service, and the VM is torn down only after the last co-located task finishes.
Conceptually, this policy combines adviser-driven co-location with capacity-aware overbooking: FIFO ordering and capacity-ranked host filling are preserved, but batches may intentionally outsize instantaneous capacity to exploit latency hiding and temporal slack (e.g., I/O wait, imbalanced phases). Because ShaRiff groups complementary tasks, oversubscription tends to translate into higher throughput and energy efficiency than naive overbooking; however, when clustered tasks are less complementary, contention can surface, making this variant an explicit trade-off between utilization and interference.

% ShaRiff 3

% Algorithm

\begin{algorithm}[H]
    \caption{ShaRiff 3 — Round-Robin Node Assignment with contention-aware VM Co-Location}
    \label{alg:sharecomp_rr}
    \KwIn{workflow \( G=(T,E) \), hosts \( \mathcal{H}=\{h_1,\dots,h_m\} \), idle cores \( c(h_j) \), ShaReComp co-location API \( \mathcal{S} \)}
    \KwOut{tasks executed using round-robin host selection with affinity-based VM co-location}

    \BlankLine
    Initialize idle cores \( c(h_j)\gets C_j \) for all \( h_j\in\mathcal{H} \)
    Initialize ready queue \( \mathcal{Q} \) with all source tasks of \( G \) (FIFO order)
    Initialize round-robin index \( r \gets 0 \)

    \BlankLine
    \While{not all tasks \( t_i\in T \) completed}{
        \If{\( \mathcal{Q} \) is empty}{
            Wait until any task \( t_r \) completes
            Release its cores: \( c(h(t_r)) \gets c(h(t_r)) + \text{req\_cores}(t_r) \)
            For each successor \( t_s \) of \( t_r \), enqueue \( t_s \) into \( \mathcal{Q} \) if all predecessors are completed\\
            \textbf{continue}
        }

        Select next host \( h = \mathcal{H}[r \bmod |\mathcal{H}|] \)
        Update round-robin index: \( r \gets (r + 1) \bmod |\mathcal{H}| \)
        Retrieve available cores \( C = c(h) \)
        Select up to \( C \) ready tasks from \( \mathcal{Q} \) into \( \mathcal{T}_h \)
        Query ShaReComp API: \( \text{colocMap} \gets \mathcal{S}(\mathcal{T}_h) \) \tcp*[r]{returns co-location groups}

        \BlankLine
        \ForEach{group \( \mathcal{C}_k \in \text{colocMap} \)}{
            \( C_{\text{req}} \gets \text{sum(req\_cores}(\mathcal{C}_k)) \)
            \( M_{\text{req}} \gets \text{sum(req\_mem}(\mathcal{C}_k)) \)
            Allocate \( v_h = (C_{\text{req}}, M_{\text{req}}, h) \)
            Launch all \( t \in \mathcal{C}_k \) on \( v_h \)
            \( c(h) \gets \max(0, c(h) - C_{\text{req}}) \)
            Remove \( \mathcal{C}_k \) from \( \mathcal{Q} \)
        }

        \BlankLine
        Wait until any task \( t_r \) completes
        Release its cores: \( c(h(t_r)) \gets c(h(t_r)) + \text{req\_cores}(t_r) \)
        \If{no active tasks remain on its VM}{ Destroy VM }
        For each successor \( t_s \) of \( t_r \): if all predecessors are completed, enqueue \( t_s \) into \( \mathcal{Q} \)
    }
    \Return{workflow complete}
\end{algorithm}

This variant preserves FIFO dequeuing but combines round-robin first-fit placement with ShaRiff-guided intra-VM co-location. The scheduler releases tasks strictly in arrival order. The node-assignment component scans hosts in round-robin/first-fit fashion and picks the first host reporting at least one idle core. It then pulls up to that hosts idle-core capacity worth of ready tasks and queries ShaRiff for a co-location plan over this batch.
The allocator realizes ShaRiffs plan one VM per suggested cluster on the chosen host. For each multi-task cluster, it sizes the VM by summing vCPU and memory requirements of the clusters tasks, starts the VM, and submits all cluster tasks to the same virtual compute service. Tasks that ShaRiff leaves as singletons are grouped onto an additional shared VM on that host; its size is the aggregate of the singletons requests. VM lifecycle is managed per cluster: each VM stays up until all of its assigned tasks complete, then is shut down and destroyed.
Conceptually, the policy is first-fit host, adviser-driven packing. It preserves FIFO ordering and simple first-fit host selection while letting ShaRiff decide which tasks should share a VM to reduce interference (by favoring complementary profiles). Unlike the max-parallel variants, this strategy does not oversubscribe cores; it fills only the currently free capacity of the first eligible host and relies on ShaRiffs clustering to raise utilization and efficiency through informed co-location.

% TODO: This part needs to be reworked.
% MinMin extension
This scheduler extension augments the ShaRiff-based variants with a MinMin ordering layer, a classical heuristic from list scheduling. In standard list scheduling, tasks are iteratively selected based on their earliest estimated completion time; Min Min specifically chooses, at each step, the task (or cluster) with the minimum predicted runtime among all ready candidates and schedules it first. Here, this principle is applied not to individual tasks but to task clusters produced by ShaRiffs co-location analysis.
For each scheduling interval, the scheduler requests from ShaRiff a co-location partition of the ready tasks, grouping them into clusters that are predicted to interact efficiently when sharing a VM. It then queries a prediction service for each cluster, obtaining runtime estimates through the chosen model (e.g., KCCA). The clusters are ordered by ascending predicted runtime, and this order defines the execution sequence. Node selection and VM provisioning are delegated to the ShaRiff node assignment and allocator components, which handle placement and resource sizing as usual.
Conceptually, this forms a Min Min list scheduler over co-located clusters: it maintains ShaRiffs intelligent co-location strategy while globally minimizing queueing delay and improving average completion time by prioritizing faster clusters. This layer is independent of the underlying allocation or node assignment logic and purely refines execution ordering to exploit performance prediction while preserving all structural and capacity constraints of the ShaRiff framework.
\clearpage
\section{Implementation}
\label{cha:implementation}
% TODO: Insert previous chapter references
This chapter details the technical implementation of the concepts introduced in the previous section. While the Approach chapter established the conceptual and algorithmic foundations of the proposed scheduling framework, the following sections focus on how these ideas were realized in practice. The implementation emphasizes architectural modularity, clear component interfaces, and the integration of machine learning–based decision layers within a simulation-driven environment. Each subsystem—from the monitoring client and statistical modeling backend to the simulation environment—was designed to remain functionally independent while communicating through well-defined data and control flows. This decoupled design not only facilitates reproducibility and maintainability but also enables future extensions, such as the replacement of predictive models or the addition of new scheduling policies, without major structural changes. The remainder of this chapter outlines the overall system architecture, justifies the chosen technologies and design principles, and describes the concrete implementation of the monitoring client, the statistical learning components, and the simulator setup that collectively form the experimental framework.

\subsection{System Architecture}
\label{sec:system_architecture}
% TODO: Provide technical overview picture of the system architecture E2E
% Insert numbers in the picture that are then briefly discussed with references to their section in the approach.

\subsection{Technology and Design Choices}
\label{sec:technology_and_design_choices}
% TODO: Write short text about design choice per component but not much detail.
% Provide a table per number of the previous figure with the technology and design choice and a brief justification.

\begin{table}[H]
    \centering
    \caption{Monitoring features and associated technical components.}
    \label{tab:monitoring_layers}
    \renewcommand{\arraystretch}{1.15}
    \resizebox{\textwidth}{!}{
        \begin{tabular}{
            p{4.5cm}
            >{\centering\arraybackslash}p{2.8cm}
            p{8.2cm}
            }
            \toprule
            \textbf{Component Category} & \textbf{Software Used}                                   & \textbf{Comments / Functionality}                                                                                              \\
            \midrule

            \multicolumn{3}{l}{\textbf{Workflow Execution}}                                                                                                                                                                         \\[3pt]
            Operating System            & Ubuntu 22.04.5 LTS                                       & Base system environment for all components.                                                                                    \\
            Kernel                      & Linux 5.15.0-143-generic                                 & Provides low-level system resource management.                                                                                 \\
            Workflow Management Engine  & Nextflow 25.06.0                                         & Controls workflow DAG execution and task dependency resolution.                                                                \\
            Virtualization Layer        & Docker Engine 28.3.1 (Community)                         & Provides isolated VM environments for task execution.                                                                          \\
            Resource Manager            & Slurm 24.05.2                                            & Allocates CPU and memory resources dynamically across hosts.                                                                   \\

            \midrule
            \multicolumn{3}{l}{\textbf{Monitoring System}}                                                                                                                                                                          \\[3pt]
            Time-Series Database        & Prometheus Server 3.3.1                                  & Central metric collector and time-series database.                                                                             \\
            Monitoring Client           & Go 1.24.1                                                & Collects per-node and per-container resource metrics.                                                                          \\

            \midrule
            \multicolumn{3}{l}{\textbf{Workload Experiments}}                                                                                                                                                                       \\[3pt]
            Benchmark Execution         & stress-ng 0.13.12                                        & Generates controlled CPU, memory, and I/O workloads.                                                                           \\
            Monitoring                  & Deep-Mon (custom fork) with Python 3.10.12 \& BCC 0.35.0 & Captures resource traces during workload execution.                                                                            \\

            \midrule
            \multicolumn{3}{l}{\textbf{Data Analysis}}                                                                                                                                                                              \\[3pt]
            Feature Engineering         & Jupyter Kernel 6.29.5, Python 3.10.12, Poetry 2.1.2      & Derives and normalizes temporal task signatures.                                                                               \\
            Clustering                  & scikit-learn 1.5.2                                       & Groups similar task behaviors into consolidated classes.                                                                       \\
            KCCA                        & CCA-Zoo 2.5.0                                            & Learns correlations between performance and energy domains.                                                                    \\
            Random Forest Regressor     & scikit-learn 1.5.2                                       & Predicts runtime and power consumption from task signatures.                                                                   \\
            Kernel Ridge Regression     & scikit-learn 1.5.2                                       & Baseline model for non-linear regression comparison.                                                                           \\

            \midrule
            \multicolumn{3}{l}{\textbf{Simulation Framework}}                                                                                                                                                                       \\[3pt]
            Workflow Tracing            & Nextflow Tracer (custom fork)                            & Records execution-level metadata for simulation replay.                                                                        \\
            Simulation Engine           & WRENCH 2.7 \& C++ 17                                     &                                                                 & Evaluates scheduling strategies under controlled conditions. \\
            Clustering API              & FastAPI 0.1.0, Python 3.10.12                            & Provides task grouping via ShaReComp integration.                                                                              \\
            Prediction API              & FastAPI 0.1.0, Python 3.10.12                            & Interfaces learned models for runtime and energy estimation.                                                                   \\

            \bottomrule
        \end{tabular}
    }
\end{table}


\subsection{Extensibility through Decoupled Design}
\label{sec:extensibility_through_decoupled_design}
The extensibility of the overall system is achieved through a strictly decoupled design that separates monitoring, modeling, and simulation components while maintaining clear communication interfaces between them. Each part of the system can evolve independently without impacting others, enabling modular experimentation with new data sources, predictive models, or scheduling strategies. This modularity supports reproducibility and future extensibility, as new data collection layers or simulation backends can be added by extending well-defined interfaces rather than rewriting existing code.
\subsubsection{Monitoring Client}
\label{sec:monitoring_client}
The monitoring client exemplifies this philosophy by relying on a flexible configuration-driven architecture. Using a YAML-based configuration file, it dynamically defines which metrics to collect from heterogeneous data sources such as Prometheus exporters, cAdvisor, eBPF probes, or SNMP-based sensors. The configuration specifies not only the metric names and queries but also the identifiers and units, allowing seamless adaptation to different environments or workflow engines. This separation of logic and configuration enables the same client to operate across diverse infrastructures without recompilation. The client’s implementation, built on the Prometheus API, abstracts away the complexity of time-range queries and concurrent metric fetching through lightweight threading and synchronization mechanisms. As a result, developers can extend the monitoring framework simply by adding new data sources or metrics to the configuration file, without altering the underlying collection logic.

% Table to show the config file of the monitoring client
\begin{table}[H]
    \centering
    \caption{Adaptable Monitoring Configuration Overview.}
    \label{tab:monitoring_config_overview}
    \renewcommand{\arraystretch}{1.15}
    \resizebox{\textwidth}{!}{
        \begin{tabular}{
            p{3.5cm}
            >{\centering\arraybackslash}p{2cm}
            p{5cm}
            p{6cm}
            }
            \toprule
            \textbf{Monitoring Target} & \textbf{Enabled} & \textbf{Supported Data Sources}                         & \textbf{Collected Metric Types / Adaptability Notes}                                                                    \\
            \midrule

            Task Metadata              & ✓                & slurm-job-exporter                                      & Collects job metadata (state, runtime, working directory). Can adapt to other job schedulers.                           \\

            CPU                        & ✓                & cAdvisor, ebpf-mon, docker-activity                     & Captures CPU time and cycles from both container and kernel levels. Supports switching sources for varying granularity. \\

            Memory                     & ✓                & cAdvisor, docker-activity                               & Tracks memory utilization at container or process level. Configurable for byte- or percentage-based metrics.            \\

            Disk                       & ✓                & cAdvisor                                                & Monitors block I/O and filesystem throughput. Supports extension with storage exporters.                                \\

            Network                    & ✗                & cAdvisor (optional)                                     & Disabled by default due to noise. Can be enabled for network-intensive workflows.                                       \\

            Energy                     & ✓                & docker-activity, ebpf-mon, ipmi-exporter, snmp-exporter & Multi-layer energy monitoring from container to node level. Adaptable to hardware sensors and external power meters.    \\

            \midrule
            \multicolumn{4}{l}{\textbf{Prometheus Configuration}}                                                                                                                                                                             \\[3pt]
            \multicolumn{4}{p{16.5cm}}{
                The Prometheus backend collects all metrics via configurable scrape intervals and targets. Controller and worker nodes can be flexibly defined, enabling distributed monitoring setups.
            }                                                                                                                                                                                                                                 \\

            \bottomrule
        \end{tabular}
    }
\end{table}

% I think i dont need it here.
% \subsubsection{Resource Contention Measurements}
% \label{sec:resource_contention_measurements}
% % Part of it can also be used for the System Overview because it's way too much detail here.
% The post-processing pipeline operates on two in-memory registries populated during execution: a container-level log of isolated runs and a container-level log of co-located runs. Each record carries a stable workload label, a unique container identifier, the measured wall-clock lifetime derived from Docker start/finish timestamps, and, where available, the mean power traced by the monitoring stack. Pair membership for co-located experiments is encoded via a grouping key that is attached to both containers in a pair. After all runs complete, the aggregator performs a single pass over the co-located registry and folds entries into a pair-centric dictionary. The first container encountered for a given pair initializes the row with its workload identity and its co-located and isolated measurements resolved via a label join against the isolated registry; the second container completes the row with its counterpart’s values. This guarantees that each summary row is self-contained: it holds two isolated baselines and the two co-located measurements produced in the same experiment, keyed by the exact workloads that were paired.
% On top of this consolidated representation, the code derives two scalar indicators that capture interference and complementarity. Slowdown is computed by contrasting, for each workload, its isolated runtime and power with the corresponding co-located values and then averaging both perspectives into a single number, so a pair’s slowdown summarizes the symmetric penalty of sharing. Affinity aggregates the same evidence in a benefit-centric way by comparing the sum of isolated runtime and power against the observed totals under co-location; values closer to one indicate that the pair behaves nearly additively, while higher values signal that the combination degrades either time-to-completion or power efficiency. The implementation only emits these aggregates when a pair row is complete and all numeric fields are present; otherwise the row is left unscored so that downstream statistics and learning are trained on consistent, gap-free observations.
% Runtime and power measurements are anchored directly in the container engine to avoid clock skew and sampling artifacts. For every container, the start and finish timestamps are parsed from Docker’s RFC-3339 strings with microsecond precision and converted into lifetimes by subtraction. Power is attached after the run by scanning the monitoring output directory tree, pruning unrelated subfolders, and computing the per-container mean over the retained time series. Because the registries store both the container identifiers and the stable workload labels, the join between isolated baselines and co-located outcomes is deterministic even when multiple instances of the same workload are present.
% To characterize contention mechanisms more precisely, the harness also runs host-level and cross-resource microbenchmarks and captures their native outputs. CPU pressure is exercised with stress-ng using matrix multiplication workers; memory pressure is generated via stress-ng’s virtual memory allocators; storage pressure is created with fio in direct-I/O mode, including io_uring engines and tuned queue depths. Each benchmark is launched as a Docker container with explicit CPU pinning through cpuset masks, and privileged mode is enabled where kernel features (e.g., io_uring) require it. For these runs, the wrapper blocks until completion, reloads container metadata, computes lifetimes from start/finish timestamps, and then retrieves and stores the raw benchmark logs. This yields both a normalized lifetime metric used by the aggregator and the original tool readouts for auditing and drill-down.
% The co-location experiments cover complementary and competing pairings by binding containers either to the same physical core’s sibling threads or to disjoint CPU ranges representing different NUMA domains. CPU–memory, memory–I/O, and CPU–I/O pairs are executed in both directions, as well as homogeneous pairs, to expose asymmetries in interference. Because every container record carries its pairing key, the summarizer can reconstruct per-pair outcomes regardless of execution order and without relying on wall-clock correlation. The result is a technically minimal but robust measurement stack: lifetimes are computed from engine timestamps, power is reduced from per-container traces, logs are persisted for verification, and all metrics are funneled into pair-centric rows from which slowdown and affinity are derived consistently.

\subsubsection{Bridged Gap between Simulation and Statistical Models}
\label{sec:statistical_modeling}
The statistical modeling component, implemented as a standalone FastAPI service, follows a similar modular design. It exposes a clean, language-agnostic HTTP API that separates the inference logic from data ingestion and model management. The service maintains state for clustering and prediction requests, delegating core computational tasks to dedicated helper functions. This decoupling makes it straightforward to replace or add new predictive models, such as neural architectures or alternative regression approaches, without modifying the API contract. The clustering and prediction endpoints can interact with any external workflow manager or simulator via standardized JSON payloads, ensuring flexibility in integrating new pipelines or retraining procedures. This independence between model serving and data processing pipelines also simplifies scalability, allowing the modeling service to be containerized and deployed independently for distributed or cloud-based setups.

% Graphic with API Endpoints showing the implemented capabilities:

\begin{table}[H]
    \centering
    \caption{Overview of REST API Endpoints Exposed by the ShaReComp Service.}
    \label{tab:api_endpoints}
    \renewcommand{\arraystretch}{1.15}
    \resizebox{\textwidth}{!}{
        \begin{tabular}{
            p{0.8cm}
            p{5.3cm}
            >{\centering\arraybackslash}p{2cm}
            p{9cm}
            }
            \toprule
            \textbf{\#} & \textbf{Endpoint}               & \textbf{Method} & \textbf{Description / Response Schema}                   \\
            \midrule

            1           & \texttt{/clusterize\_jobs}      & POST            &
            Clusters nf-core jobs based on historical data. \newline
            \textit{Request:} \texttt{ClusterizeJobsRequest} (list of job names). \newline
            \textit{Response:} \texttt{ClusterizeJobsResponse} (cluster mapping with run ID).                                          \\

            2           & \texttt{/predict}               & POST            &
            Predicts runtime and power consumption for consolidated job clusters. \newline
            \textit{Request:} \texttt{PredictRequest} (cluster IDs, model types). \newline
            \textit{Response:} \texttt{PredictionResponse} (predicted values for each model).                                          \\

            \midrule
            \multicolumn{4}{l}{\textbf{Component Schemas}}                                                                             \\[3pt]

            --          & \texttt{ClusterizeJobsRequest}  & Object          &
            \textbf{Fields:} \texttt{job\_names} (array of strings). Required.                                                         \\

            --          & \texttt{ClusterizeJobsResponse} & Object          &
            \textbf{Fields:} \texttt{run\_id} (string), \texttt{clusters} (map of arrays). Required.                                   \\

            --          & \texttt{PredictRequest}         & Object          &
            \textbf{Fields:} \texttt{cluster\_ids} (array), \texttt{prediction\_models} (array of \texttt{PredictionModel}). Required. \\

            --          & \texttt{PredictionModel}        & Object          &
            \textbf{Fields:} \texttt{model\_type} (array of strings). Required.                                                        \\

            --          & \texttt{PredictionResponse}     & Object          &
            \textbf{Fields:} \texttt{run\_id} (string), \texttt{predictions} (nested map of numeric values). Required.                 \\


            \bottomrule
        \end{tabular}
    }
\end{table}

\subsubsection{Simulator Setup}
\label{sec:simulator_setup}
The simulator setup further demonstrates the benefits of this decoupled design. The resource management layer of the simulator exposes generic interfaces for allocators, schedulers, and node assigners, allowing any of them to be replaced or combined dynamically at runtime. This separation allows the same simulator to execute both baseline and experimental resource allocation strategies—including oversubscription, co-location, or ShaRiff-based scheduling—without modifying the controller logic. Through this design, the simulator can execute diverse workflow types, including various nf-core pipelines, by merely switching configuration parameters or class bindings. Moreover, the integration of energy and performance tracing through independent services ensures that extending the simulator with new measurement capabilities does not interfere with the scheduling or execution logic.

% Show cpp code of base directory

\clearpage
\section{Evaluation}
\label{cha:evaluation}

\subsection{Evaluation Setup}
\label{sec:evaluation_setup}

\subsubsection{Infrastructure}
\label{sec:evaluation_infrastructure}
% Table with different nodes and their characteristics, as well as slurm

\subsubsection{Workflows}
\label{sec:evaluation_workflows}
% Table with different workflows and their characteristics

\subsubsection{Monitoring Configuration}
\label{sec:evaluation_monitoring_configuration}
% Table with different combinations

\subsubsection{Statistical Learning Methods}
\label{sec:evaluation_statistical_learning_methods}
% Just mention that the Models are accessible on the same host via API and the parameters they were run with

\subsubsection{Simulation Configuration}
\label{sec:evaluation_simulation_configuration}
% XML file integration 

\subsection{Experiment Results}
\label{sec:experiment_results}

\subsubsection{Monitoring Statistics}
\label{sec:monitoring_statistics}

\subsubsection{Resource Contention Analysis}
\label{sec:resource_contention_analysis}
% Only summarize quickly what was done and focus on the results and their interpretation while for the approach refering to the according chapter.
% Put tables and figures in here.

\subsubsection{Statistical Learning Methods}
\label{sec:evaluation_statistical_learning_methods}

\paragraph{Task Consolidation}
\label{sec:evaluation_task_consolidation}
% Spider diagram, example task time-series that are different and clustered

\paragraph{Task Cluster Runtime and Energy Prediction}
\label{sec:evaluation_task_cluster_runtime_and_energy_prediction}
% Table with prediction results based on basic metrics

\subsubsection{Workflow Makespan and Energy Consumption}
\label{sec:workflow_makespan_and_energy_consumption}

\clearpage

\section{Discussion}
\label{cha:discussion}
We now conclude by reflecting on the evaluation results.
% Topics
% Contention Experiments
Beginning with the experiments on resource contention, we have already discussed the observed outcomes and the underlying reasons for these behaviors. We argue that repeating these measurements on Intel-based hardware would likely yield more consistent and intuitive power consumption results for co-located benchmarks, given the better stability and accessibility of Intel's RAPL interfaces compared to AMD's.
Another factor influencing the results is the type of co-location applied is that in our setup, memory access is shared across cores via NUMA. As we confirmed the processor-to-core mapping on the test system with two CPUs available, we experimented with mapping co-located benchmark pairs according to their NUMA layout—such as pairing core 0 with 24—and pinning both tasks to the same physical CPU. Although both strategies produced broadly similar results, a more detailed investigation of the benchmark's internal performance metrics, rather than relying solely on runtime and power consumption, could yield a more accurate understanding of contention effects.
Incorporating detailed benchmark-level metrics could reduce the dependency on power-based measurements and allow the use of smaller weighting factor alpha when combining runtime and energy in the affinity computation. This refinement would likely produce more reliable and interpretable affinity scores, better reflecting the true resource interaction between co-located tasks.

% Preprocessing for clustering and prediction
The last point of discussion also relates to the results obtained from the predictor training. Overall, the evaluation indicates clear potential in using predictive models to estimate the performance and energy consumption of consolidated workflow tasks. However, several challenges remain, particularly those related to data dimensionality. The structure and representation of time-series features strongly influence how well a model can learn meaningful relationships between task behavior, runtime, and energy usage. Determining how these temporal features should be arranged, aggregated, or reduced to accurately reflect task behavior is therefore a key area for future research.
The flexibility of our monitoring approach, which records a large number of low-level features per task, also introduces complexity in feature selection. Deciding which metrics to retain and understanding how each contributes to prediction accuracy remains an open question. This uncertainty likely contributes to the observed behavior of the KCCA model, which successfully identified correlations between time-series features and performance metrics but exhibited strong overfitting.
As mentioned earlier, inconsistencies in the measured power data—stemming from the Deep-Mon fork used on AMD hardware—may further explain the instability observed in the models. Inaccurate or noisy energy readings can easily lead to model confusion, reducing generalization ability. Future work should therefore include more reliable measurement sources, improved feature preprocessing, and systematic dimensionality studies to establish a robust and interpretable prediction framework for workflow performance and energy behavior.

% WRENCH
Lastly, we discuss the results from simulating the integration of \textit{ShaReComp} into the workflow execution framework through the \textit{ShaRiff} algorithms. Although the \textit{ShaRiff} variants generally perform well and outperform the baselines, their success depends strongly on the quantity and quality of available monitoring data. As mentioned earlier, there is a direct interdependence between the effectiveness of the clustering algorithm used in \textit{ShaRiff} and the completeness of the monitoring data. This explains why, in several cases shown in the appendix, the baselines on which \textit{ShaRiff} builds outperform it when detailed monitoring information is missing for many tasks.
Our initial assumption was that combining the best-performing baseline—baseline 3—with the co-location awareness provided by \textit{ShaReComp} would consistently yield superior results. While this expectation holds for certain workflows, it does not generalize across all of them. We therefore see potential in extending the evaluation to a wider range of workflows with different task characteristics to better understand when and why \textit{ShaRiff} provides the largest benefit.
When comparing the performance of all baselines and \textit{ShaRiff} algorithms across workflows, their lower performance bound appears to remain within a similar magnitude. Several factors could explain this. The first two baselines, which do not employ co-location, clearly perform worse, as expected. However, the differences among the node assignment strategies are smaller than anticipated. For example, we expected that always selecting the largest host would result in faster completion for workflows with fewer tasks, compared to round-robin or parallel assignment strategies. Although such trends are visible, the differences are not as pronounced as predicted.
A key reason lies in the simulator design and the way baselines are implemented. During many scheduling intervals, the task queue available for allocation or co-location contained only a few tasks. To address this, we introduced a minimum queue size requirement before enabling co-location, which improved results. Nevertheless, considering task dependencies directly from the workflow DAG could further increase throughput and prevent underutilization caused by limited queue size.
We also encountered limitations in the virtual cluster compute service of WRENCH. Once tasks are assigned to a virtual machine, the VM cannot be resized dynamically; even if one task completes early, the allocated core remains idle until all tasks within that VM finish. Introducing VM resizing capabilities could significantly improve resource utilization. Similarly, instead of placing all singleton tasks that do not belong to clusters into a single VM, spawning separate VMs for each could potentially reduce makespan and energy consumption. Future work should also integrate the idle-time metric provided by SimGrid into WRENCH to quantify unused resources more precisely.
Moreover, WRENCH supports direct access to workflow DAG information from the workflow management system, which could be leveraged to improve scheduling decisions. Implementing additional random co-location and scheduling strategies would also provide more comparative baselines for assessing energy efficiency.
Overall, while \textit{ShaRiff} already demonstrates improved runtime and energy consumption in many cases, its full potential has yet to be explored. Future work should focus on integrating alternative cluster resource management strategies, expanding the host pool, and refining the platform model. In particular, moving beyond the current assumption of linear energy consumption relative to core usage toward a more fine-grained and realistic energy model would provide a more accurate evaluation of how \textit{ShaReComp} enhances ShaRiff's performance.

\clearpage

\section{Conclusion and Future Work}
\label{cha:conclusionfuturework}
This thesis explores the field of co-locating scientific workflow tasks to reduce resource contention and improve energy-awareness. We identified the relevant research areas and introduced a fine-grained task monitoring system to capture detailed task behavior over time. This system records low-level execution time series with minimal overhead and processes them to enable statistical analysis.
Furthermore, we proposed a novel online task clustering approach that extends and adapts an existing formulation of the co-location problem by framing it as a consolidation process to group dissimilar workloads. We formalized this problem through a distance-based formulation, defined a threshold mechanism, and derived complementary task clusters. We demonstrated how these clusters and their low-level temporal resource signatures can be used to predict task behavior through modeling. Specifically, we applied KCCA together with a regression model to identify maximum correlations between task metrics and performance, and used Random Forest models to predict runtime and energy consumption independently.
Finally, we investigated how co-location can be integrated into workflow scheduling by designing a simulation framework capable of embedding such mechanisms. We implemented two naive baselines and five co-location heuristics, along with three algorithms that use the dissimilarity-based clustering approach, and evaluated them on nine workflows from the nf-core repository.
The results show promising potential for adopting complementary task co-location in online scheduling environments. The proposed methods enable more informed scheduling decisions that can serve multiple objectives, such as makespan reduction, energy efficiency, and potentially reduced carbon footprint.
Through the extensible framework developed in this work, we provide a foundation for further research by defining clear interfaces and offering a formal basis for extending the co-location modeling and scheduling capabilities.

Future work should focus on several directions to extend and refine the proposed framework. First, the monitoring system can be improved to achieve broader task coverage, reduced overhead, and more accurate energy estimation, for instance by enhancing eBPF-based measurements and incorporating vendor-independent predictive energy models based on external powermeters. Second, the processing of time-series data should be deepened to generate more expressive and representative feature vectors, supported by dimensionality reduction and feature selection techniques that prioritize features with the highest explanatory power.
Third, the current affinity score calculation could be advanced through more detailed, low-level interference measurements, while also improving the handling of static or less dynamic time-series data in distance computations.
These key directions will enhance the framework and deepen the understanding of co-location-based workflow scheduling. In particular, improving the quality and coverage of monitoring data will be essential for the successful operation of the \textit{ShaRiff} algorithms as they depend strongly on the availability of accurate and complete task metrics. As discussed, the performance of \textit{ShaRiff} is also interdependent with the clustering algorithm's quality because insufficient monitoring detail limits its effectiveness. Expanding the evaluation to more diverse workflows with different task characteristics will help clarify under which conditions \textit{ShaRiff} achieves the greatest benefits.
Additionally, improvements to the simulation environment hold potential. The current design limits resource utilization, as virtual machines cannot be resized dynamically. As a consequence idle cores remain unused until all tasks within a VM complete. Supporting VM resizing and refining task allocation such as spawning separate VMs for singleton tasks could further reduce makespan and energy usage. Leveraging workflow DAG information available in WRENCH alongside SimGrid's detailed idle-core time metrics could further improve scheduling precision and provide more accurate insights into resource utilization.
Finally, incorporating more realistic energy models that move beyond the current linear assumptions and integrating additional random co-location and scheduling strategies will strengthen the framework's analytical scope. We plan to extend our simulation by expanding the host pool and adopting alternative resource management strategies to provide richer insights into how co-location and \textit{ShaReComp} improve the performance of workflow execution systems.
From a systems perspective, integrating explicit cost functions into the co-location strategies would allow the formulation of optimization problems that support both single- and multi-objective trade-offs. The simulation framework should also be extended and calibrated with realistic energy sources to better reflect actual data center behavior. Finally, future research could reformulate the co-location problem itself from task consolidation towards predicting degradation through resource contention directly thereby moving from analytical modeling to real-time decision-making.


\clearpage

\appendix
\section{Plots}
\section{Algorithms}

\newpage
\printbibliography[heading=bibintoc,title={Bibliography}]
\end{document}