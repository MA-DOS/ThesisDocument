\section{Conclusion and Future Work}
\label{cha:conclusionfuturework}
This thesis explores the field of co-locating scientific workflow tasks with the aim of reducing resource contention and improving energy awareness. We identified the relevant research areas and introduced a fine-grained task monitoring system to capture detailed task behavior over time. This system records low-level execution time series with minimal overhead and processes them to enable statistical analysis.

Furthermore, we proposed a novel online task clustering approach by extending and modifying an existing formulation of the co-location problem, treating it as a consolidation task that groups dissimilar workloads together. We formalized this problem through a distance-based formulation, defined a threshold mechanism, and derived complementary task clusters. We demonstrated how these clusters and their low-level temporal resource signatures can be used to predict task behavior through modeling. Specifically, we applied Kernel Canonical Correlation Analysis (KCCA) to identify maximum correlations between task metrics and performance, and used Random Forest models to predict runtime and energy consumption independently.
Finally, we investigated how co-location can be integrated into workflow scheduling by designing a simulation framework capable of embedding such mechanisms. We implemented two naive baselines and five co-location heuristics, along with three algorithms that use the dissimilarity-based clustering approach, and evaluated them on nine workflows from the nf-core repository.
The results show promising potential for adopting complementary task co-location in online scheduling environments. The proposed methods enable more informed scheduling decisions that can serve multiple objectives, such as makespan reduction, energy efficiency, and potentially reduced carbon footprint.
Through the extensible framework developed in this work, we provide a foundation for further research by defining clear interfaces and offering a formal basis for extending the co-location modeling and scheduling capabilities.
Future work will focus on several key directions to enhance the framework and deepen the understanding of co-location-based workflow scheduling. In particular, improving the quality and coverage of monitoring data will be essential, as the success of the ShaRiff algorithms depends strongly on the availability of accurate and complete task metrics. As discussed, the performance of ShaRiff is also interdependent with the clustering algorithm's quality—insufficient monitoring detail limits its effectiveness. Expanding the evaluation to more diverse workflows with different task characteristics will help clarify under which conditions ShaRiff achieves the greatest benefits.
Additionally, improvements to the simulation environment are needed. The current design limits resource utilization, as virtual machines cannot be resized dynamically; idle cores remain unused until all tasks within a VM complete. Supporting VM resizing and refining task allocation—such as spawning separate VMs for singleton tasks—could significantly reduce makespan and energy usage. Leveraging workflow DAG information available in WRENCH, together with SimGrid's idle-time metrics, will further enhance scheduling precision and resource accounting.
Finally, incorporating more realistic energy models that move beyond the current linear assumptions and integrating additional random co-location and scheduling strategies will strengthen the framework's analytical scope. Expanding the host pool and adopting alternative resource management strategies will provide richer insights into how co-location and ShaReComp improve the performance of workflow execution systems.
% \subsection{Final Remarks}
% \label{sec:final_remarks}

% \subsection{Outlook}
% \label{sec:outlook}

% Choose a couple systems and ideas, match them to my current state of the work and write that how and why it would improve the current state of the work by referencing the assumptions part in the approach section.
\begin{enumerate}
    \item Monitoring Improvement, higher task coverage, better ebpf, lower overhead, better energy models
    \item More in-depth time-series treatment for better feature-vectors
    \item Dimensionality reduction, feature selection based on highest explanation
    \item More sophisticated affinity score calculation based on lower-level interference measurements
    \item Treatment of static time-series for distance calculation
    \item Refinement of regression approach within KCCA by comparing to other means
    \item Extending by more suitable or sophisticated models.
    \item Integrating cost-functions into the co-location strategies so that optimization problems can be formulated and account for single or multi-obectives.
    \item Extending and calibrating the simulation framework with energy-sources for more realistic simulation results
    \item Reformulating the co-location problem from consolidation problem into degredataion prediction

\end{enumerate}