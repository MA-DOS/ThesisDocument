\section{Conclusion and Future Work}
\label{cha:conclusionfuturework}
This thesis explores the field of co-locating scientific workflow tasks to reduce resource contention and improve energy-awareness. We identified the relevant research areas and introduced a fine-grained task monitoring system to capture detailed task behavior over time. This system records low-level execution time series with minimal overhead and processes them to enable statistical analysis.
Furthermore, we proposed a novel online task clustering approach that extends and adapts an existing formulation of the co-location problem by framing it as a consolidation process to group dissimilar workloads. We formalized this problem through a distance-based formulation, defined a threshold mechanism, and derived complementary task clusters. We demonstrated how these clusters and their low-level temporal resource signatures can be used to predict task behavior through modeling. Specifically, we applied KCCA together with a regression model to identify maximum correlations between task metrics and performance, and used Random Forest models to predict runtime and energy consumption independently.
Finally, we investigated how co-location can be integrated into workflow scheduling by designing a simulation framework capable of embedding such mechanisms. We implemented two naive baselines and five co-location heuristics, along with three algorithms that use the dissimilarity-based clustering approach, and evaluated them on nine workflows from the nf-core repository.
The results show promising potential for adopting complementary task co-location in online scheduling environments. The proposed methods enable more informed scheduling decisions that can serve multiple objectives, such as makespan reduction, energy efficiency, and potentially reduced carbon footprint.
Through the extensible framework developed in this work, we provide a foundation for further research by defining clear interfaces and offering a formal basis for extending the co-location modeling and scheduling capabilities.

Future work should focus on several directions to extend and refine the proposed framework. First, the monitoring system can be improved to achieve broader task coverage, reduced overhead, and more accurate energy estimation, for instance by enhancing eBPF-based measurements and incorporating vendor-independent predictive energy models based on external powermeters. Second, the processing of time-series data should be deepened to generate more expressive and representative feature vectors, supported by dimensionality reduction and feature selection techniques that prioritize features with the highest explanatory power.
Third, the current affinity score calculation could be advanced through more detailed, low-level interference measurements, while also improving the handling of static or less dynamic time-series data in distance computations.
These key directions will enhance the framework and deepen the understanding of co-location-based workflow scheduling. In particular, improving the quality and coverage of monitoring data will be essential for the successful operation of the \textit{ShaRiff} algorithms as they depend strongly on the availability of accurate and complete task metrics. As discussed, the performance of \textit{ShaRiff} is also interdependent with the clustering algorithm's quality because insufficient monitoring detail limits its effectiveness. Expanding the evaluation to more diverse workflows with different task characteristics will help clarify under which conditions \textit{ShaRiff} achieves the greatest benefits.
Additionally, improvements to the simulation environment hold potential. The current design limits resource utilization, as virtual machines cannot be resized dynamically. As a consequence idle cores remain unused until all tasks within a VM complete. Supporting VM resizing and refining task allocation such as spawning separate VMs for singleton tasks could further reduce makespan and energy usage. Leveraging workflow DAG information available in WRENCH alongside SimGrid's detailed idle-core time metrics could further improve scheduling precision and provide more accurate insights into resource utilization.
Finally, incorporating more realistic energy models that move beyond the current linear assumptions and integrating additional random co-location and scheduling strategies will strengthen the framework's analytical scope. We plan to extend our simulation by expanding the host pool and adopting alternative resource management strategies to provide richer insights into how co-location and \textit{ShaReComp} improve the performance of workflow execution systems.
From a systems perspective, integrating explicit cost functions into the co-location strategies would allow the formulation of optimization problems that support both single- and multi-objective trade-offs. The simulation framework should also be extended and calibrated with realistic energy sources to better reflect actual data center behavior. Finally, future research could reformulate the co-location problem itself from task consolidation towards predicting degradation through resource contention directly thereby moving from analytical modeling to real-time decision-making.

