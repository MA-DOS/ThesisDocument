\section{Introduction}
\label{cha:introduction}

% TODO: Put significant graphic in here for problem motivation
\subsection{Problem Motivation \& Description}
\label{subse:problem_motivation_description}
% Energy-aware Workflow Execution, Thamsen, Ch. 1,3,5
% TODO: Add a well descriptive intuition example on the problem motivation just like in the MA's
The carbon footprint of information and communication technologies (ICT) continues to increase, despite the urgent imperative to decarbonize society and remain within planetary boundaries [citation]. A key driver of this trend is the exponential growth in data collection, storage, and processing across scientific disciplines [citation]. Scientific Workflow Management Systems (SWMSs) have become essential tools for managing this complexity, enabling researchers to exploit computing clusters for large-scale data analysis in domains such as remote sensing, astronomy, and bioinformatics [citation]. However, workflows executed through SWMSs are often long-running, resource-intensive, and computationally demanding, which translates into high energy consumption and substantial greenhouse gas emissions [citation]. Techniques to mitigate these impacts include energy-efficient code generation for workflow tasks and energy-aware scheduling strategies [citation]. Nevertheless, practitioners face challenges in assessing which approaches are most applicable, effective, and feasible to implement without excessive effort \cite{thamsen2025energyawareworkflowexecutionoverview}.

Scientific workflows have become a central paradigm for automating computational workloads on parallel and distributed platforms. With the rapid growth in data volumes and processing requirements over the last two decades, workflow applications have grown in complexity, while computing infrastructures have advanced in processing capacity and workload management capabilities. A critical component of this evolution is energy management, where scheduling and resource provisioning strategies are designed to maximize throughput while reducing or constraining energy consumption [citation]. In recent years, energy management in scientific workflows has gained particular importance, not only because of rising energy costs and sustainability concerns but also due to the pivotal role workflows play in enabling major scientific breakthroughs \cite{Coleman_2021}.

% A review on the decarbonization of high-performance computing centers, CA Silva
High-performance computing (HPC) refers to the pursuit of maximizing computational capabilities through advanced technologies, methodologies, and applications, enabling the solution of complex scientific and societal problems [citation]. HPC systems consist of large numbers of interconnected compute and storage nodes, often numbering in the thousands, and rely on job schedulers to allocate resources, manage queues, and monitor execution. While these infrastructures provide the backbone for highly demanding applications, their intensive power draw—arising not only from computation but also from networking, cooling, and auxiliary equipment—makes them major consumers of electricity and significant contributors to climate change [citation]. The demand for HPC continues to rise across both public and private sectors, fueled by emerging computationally intensive domains such as artificial intelligence, Internet of Things, cryptocurrencies, and 5G networks [citation]. The transition from petascale to exascale performance has amplified sustainability concerns, as operational costs approach parity with capital investment and energy efficiency becomes a limiting factor [citation]. Recent exascale systems exemplify this trend: while achieving unprecedented performance, they consume tens of megawatts of power and highlight the urgency of energy-aware design. To increase efficiency, modern HPC architectures increasingly integrate heterogeneous hardware, combining multicore CPUs with specialized accelerators such as GPUs, enabling more operations per second but also driving higher power densities at the node and rack level. This creates additional challenges for energy management and cooling, compounded by scheduling constraints and the demand for near-continuous system availability. Addressing these issues is essential to ensure that future HPC developments meet performance goals without exacerbating their environmental footprint \cite{Silva_2024}.

% FONDA 2
As scientific workflows grow in scale and complexity, coarse models of resource usage are no longer sufficient for ensuring efficient and sustainable execution. Tasks within data-driven workflows often appear in multiple instances and may vary substantially depending on input data, parameters, or execution environments. This results in highly dynamic patterns of resource demand, energy consumption, and carbon emissions, which fluctuate during runtime rather than remaining constant. At the same time, the carbon intensity of the underlying infrastructure changes over time, reflecting variations in energy availability and network conditions across sites. To address these challenges, there is a need for fine-grained, time-dependent task models that capture detailed resource usage profiles, incorporate infrastructure energy characteristics, and adapt dynamically during execution. Such models would enable more accurate task-to-machine mappings, informed scheduling decisions across multiple sites, and strategies for co-locating complementary tasks to reduce interference, energy waste, and communication overhead. Developing these models requires continuous monitoring of tasks, predictive techniques to handle incomplete prior knowledge, and adaptive mechanisms that respond in real time to evolving infrastructure conditions. Ultimately, this approach provides the foundation for carbon-aware workflow execution, where tasks are scheduled and allocated in ways that minimize energy consumption and associated emissions while maintaining performance and scalability.
Task mapping addresses the challenge of assigning tasks to machines in a way that not only satisfies resource requirements but also minimizes energy consumption and carbon emissions. Traditional approaches largely focused on matching resource demands with machine capabilities, but they overlooked the variability in energy efficiency and carbon intensity over time. By leveraging fine-grained task models and infrastructure profiles, mappings can be extended to consider spatio-temporal variations in energy supply and demand. A key component of this is task co-location: strategically placing tasks on the same machine, within the same cluster, or in proximity across sites. When executed with awareness of complementary resource usage patterns, co-location reduces interference, avoids idle resource blocking, and lowers communication overhead, thereby improving both performance and energy efficiency. Together, advanced task mapping and co-location strategies provide a pathway toward reducing the carbon footprint of computational workflows while maintaining scalability and reliability.

% pHPCe: a hybrid power conservation approach for containerized HPC environment
Reducing power consumption while maintaining acceptable levels of performance remains a central challenge in containerized High Performance Computing (HPC). Existing approaches to Dynamic Power Management (DPM) typically rely on profile-guided prediction techniques to balance energy efficiency and computational throughput. However, in multi-tenant containerized HPC environments, this balance becomes significantly more complex due to heterogeneous user demands and contention over shared resources. These factors increase the difficulty of accurately predicting and managing power-performance trade-offs. Furthermore, while containerization frameworks such as Docker are widely adopted, they were not originally designed with HPC workloads in mind, leading to limited research and insufficient software-level mechanisms for monitoring and controlling power consumption in such environments \cite{Kuity_2023}.

Task clustering is a technique aimed at improving workflow efficiency by aggregating fine-grained computational tasks into larger units, commonly referred to as jobs. This aggregation reduces the overhead associated with scheduling numerous small tasks individually, which in turn lowers energy consumption and shortens the overall makespan of the workflow. By combining coarse-grained and fine-grained tasks into clustered jobs, resource utilization becomes more balanced, and the performance of workflow execution can be significantly enhanced \cite{Saadi_2023}.

This paper focuses on effective energy and resource costs management for scientific workflows. Our work is driven by the observation that tasks of a given workflow can have substantially different resource requirements, and even the resource requirements of a particular task can vary over time. Mapping each workflow task to a different server can be energy inefficient, as servers with a very low load also consume more than 50\% of the peak power [8]. Thus, server consolidation, i.e. allowing workflow tasks to be consolidated onto a smaller number of servers, can be a promising approach for reducing resource and energy costs.

We apply consolidation to tasks of a scientific workflow, with the goal of minimizing the total power consumption and resource costs, without a substantial degradation in performance. Particularly, the consolidation is performed on a single workflow with multiple tasks. Effective consolidation, however, poses several challenges. First, we must carefully decide which workloads can be combined together, as the workload resource usage, performance, and power consumption are not additive. Interference of combined workloads, particularly those hosted in virtualized machines, and the resulting power consumption and application performance need to be carefully understood. Second, due to the time-varying resource requirements, resources should be provisioned at runtime among consolidated workloads.

We have developed pSciMapper, a power-aware consolidation framework to perform consolidation to scientific workflow tasks in virtualized environments. We first study how the resource usage impacts the total power consumption, particularly taking virtualization into account. Then, we investigate the correlation between workloads with different resource usage profiles, and how power and performance are impacted by their interference. Our algorithm derives from the insights gained from these experiments. We first summarize the key temporal features of the CPU, memory, disk, and network activity associated with each workflow task. Based on this profile, consolidation is viewed as a hierarchical clustering problem, with a distance metric capturing the interference between the tasks. We also consider the possibility that the servers may be heterogeneous, and an optimization method is used to map each consolidated set (cluster) onto a server. As an enhancement to our static method, we also perform time varying resource provisioning at runtime \cite{5644899}.

\subsection{Research Question \& Core Contributions}
\label{subse:research_question_core_contributions}

The central questions this thesis seeks to address are:

% TODO: refine research questions
\begin{enumerate}[label=RQ\arabic*]
    \item How can fine-grained, time-dependent models of workflow tasks be developed to capture fluctuating patterns of computational resource usage and energy consumption during execution?
    \item How can the co-location of workflow tasks be modeled so that their interference is minimized and the resulting shared usage of resources leads to lower overall energy consumption and carbon emissions while performance is maintained?
    \item How can co-location models and time-dependent task characterizations be integrated into resource management systems and workflow scheduling frameworks to enable adaptive, energy-aware execution at scale?
\end{enumerate}

The resulting core contributions of this thesis are:
\begin{itemize}
    \item An architectural mapping that defines which layers of workflow execution need to be monitored and systematically assigns suitable data exporters to them
    \item The implementation of a monitoring client, capable of serving the relevant monitoring layers for scientific workflow execution which collects fine-grained, time-dependent resource usage data for workflow tasks during execution that was used for data collection on running 10 nf-core pipelines.
    \item An analysis of co-location effects on the core and node-level that are later on used for implementing the proposed co-location approach.
    \item The design and implementation of a novel co-location approach that utilizes time series data to compute for any given set of tasks clusters where the resource usage patterns are complementary.
    \item The application of 2 multivariate, statistical learning methods on the time series data of the workflow tasks: Kernel Canonical Correlation Analysis (KCCA) and Random Forest Regressor (RFR).
    \item The development of an evaluation test-bed in the WRENCH simulation framework that integrates the proposed co-location approach and allows for the simulation of scientific workflow execution with and without co-location.
    \item The evaluation of the proposed co-location approach using 10 nf-core workflows, demonstrating its effectiveness in reducing energy consumption  while maintaining performance.
\end{itemize}

\subsection{Structure of the Thesis}
\label{subse:structure_of_the_thesis}
The remainder of this thesis is structured as follows: Chapter 2 presents the fundamentals of this work, covering scientific workflow systems, the co-location problem, cluster resource management, machine learning and more. Chapter 3 introduces related work for monitoring scientific workflows, modeling the co-location of tasks and enabling energy-awareness through minimizing resource contention. In addition, the main state-of-the-art approaches for the co-location problem are presented. Furthermore, Chapter 4 depicts the approach to the problem in both a theoretical and practical manner and ultimately defines requirements for the realization of the approach. The corresponding implementation is elaborated in Chapter 5. Moreover, Chapter 6 evaluates the models compared to the state-of-the-art approaches using a simulation. Chapter 7 interprets the key findings and discusses some limitations of this work. Finally, Chapter 8 concludes this thesis and gives an outlook on the impact of this work for the future.